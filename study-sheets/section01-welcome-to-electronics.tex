\documentclass[a4paper,9pt]{article}

% Packages
\usepackage[margin=0.4in]{geometry}
\usepackage{amsmath}
\usepackage{amssymb}
\usepackage{enumitem}
\usepackage{xcolor}
\usepackage{tcolorbox}
\usepackage{titlesec}
\usepackage{fancyhdr}
\usepackage[hidelinks]{hyperref}
\usepackage{tocloft}
\usepackage{graphicx}

% ====================================================================
% CONFIGURATION - Section 01: Welcome to Electronics I
% ====================================================================
\newcommand{\topicname}{Introduction to Electronics}
\newcommand{\sectionnumber}{Section 01}
\newcommand{\sectionname}{Welcome to Electronics I}
% ====================================================================

% Color definitions
\definecolor{headercolor}{RGB}{13,71,161}        % Deep Blue - Main header
\definecolor{sectioncolor}{RGB}{26,115,232}      % Bright Blue - Section borders
\definecolor{tldrcolor}{RGB}{255,245,230}        % Light Orange - TL;DR background
\definecolor{tldrborder}{RGB}{255,152,0}         % Orange - TL;DR border
\definecolor{detailcolor}{RGB}{232,245,233}      % Light Green - Detail background
\definecolor{detailborder}{RGB}{56,142,60}       % Green - Detail border
\definecolor{examplecolor}{RGB}{255,243,224}     % Light Amber - Example background
\definecolor{exampleborder}{RGB}{245,124,0}      % Deep Orange - Example border
\definecolor{keypointcolor}{RGB}{227,242,253}    % Light Blue - Key Points background
\definecolor{keypointborder}{RGB}{25,118,210}    % Blue - Key Points border
\definecolor{boxcolor}{RGB}{240,248,255}         % Light blue for decorative boxes

% Custom box styles
\tcbuselibrary{skins,breakable}
\newtcolorbox{tldrbox}{
    colback=tldrcolor,
    colframe=tldrborder,
    fonttitle=\bfseries\large,
    title=1. TL;DR (The Gist),
    breakable,
    rounded corners,
    boxrule=2pt,
    top=6pt,
    bottom=6pt,
    left=8pt,
    right=8pt
}

\newtcolorbox{detailbox}{
    colback=detailcolor,
    colframe=detailborder,
    fonttitle=\bfseries\large,
    title=2. Detailed Explanation,
    breakable,
    rounded corners,
    boxrule=2pt,
    top=6pt,
    bottom=6pt,
    left=8pt,
    right=8pt
}

\newtcolorbox{examplebox}{
    colback=examplecolor,
    colframe=exampleborder,
    fonttitle=\bfseries\large,
    title=3. Practical Example \& Numerical,
    breakable,
    rounded corners,
    boxrule=2pt,
    top=6pt,
    bottom=6pt,
    left=8pt,
    right=8pt
}

\newtcolorbox{keypointsbox}{
    colback=keypointcolor,
    colframe=keypointborder,
    fonttitle=\bfseries\large,
    title=4. Key Points (Interview Focus),
    breakable,
    rounded corners,
    boxrule=2pt,
    top=6pt,
    bottom=6pt,
    left=8pt,
    right=8pt
}

% Section formatting
\titleformat{\section}
  {\normalfont\Large\bfseries\color{sectioncolor}}
  {\thesection}{1em}{}

\titleformat{\subsection}
  {\normalfont\large\bfseries\color{sectioncolor}}
  {\thesubsection}{1em}{}

% Remove page numbers
\pagestyle{empty}

% Compact lists
\setlist{nosep, leftmargin=15pt, itemsep=2pt}

% Table of Contents formatting
\renewcommand{\cftsecleader}{\cftdotfill{\cftdotsep}}
\setlength{\cftbeforesecskip}{5pt}
\renewcommand{\cftsecfont}{\normalsize\bfseries\color{sectioncolor}}
\renewcommand{\cftsecpagefont}{\normalsize\color{sectioncolor}}

% Document start
\begin{document}

% ====================================================================
% TITLE PAGE
% ====================================================================
\begin{titlepage}
    \centering
    \vspace*{2cm}
    
    % Title
    {\Huge\bfseries\color{headercolor} ELECTRONICS\\[0.3cm] STUDY SHEET\par}
    \vspace{1cm}
    
    % Decorative line
    {\color{sectioncolor}\rule{0.8\textwidth}{2pt}\par}
    \vspace{1.5cm}
    
    % Topic and Section
    {\LARGE\color{headercolor}\topicname\par}
    \vspace{0.5cm}
    {\Large\color{sectioncolor}\sectionnumber~--~\sectionname\par}
    \vspace{2cm}
    
    % Description box
    \begin{tcolorbox}[
        colback=boxcolor,
        colframe=sectioncolor,
        width=0.8\textwidth,
        arc=3mm,
        boxrule=1.5pt
    ]
    \centering
    \large
    \textbf{Comprehensive One-Page Study Guide}\\[0.3cm]
    \normalsize
    Covering fundamental concepts, formulas, examples,\\
    and interview-focused key points
    \end{tcolorbox}
    
    \vfill
    
    % Bottom information
    {\large\color{headercolor}
    Electronics Engineering Reference Series\\[0.2cm]
    \normalsize
    \today
    \par}
    
\end{titlepage}

% Reset geometry for content page
\newgeometry{margin=0.4in}

% ====================================================================
% TABLE OF CONTENTS
% ====================================================================
\begin{center}
    {\LARGE\bfseries\color{headercolor} TABLE OF CONTENTS}
    \vspace{0.3cm}
    \hrule height 2pt
\end{center}
\vspace{0.3cm}

\tableofcontents
\vspace{0.5cm}

\begin{center}
\hrule height 1pt
\end{center}

\newpage

% ====================================================================
% MAIN CONTENT PAGE
% ====================================================================

% Suppress section numbering
\setcounter{secnumdepth}{0}

% Header
\begin{center}
    {\Huge\bfseries\color{headercolor} ELECTRONICS STUDY SHEET}
    \vspace{0.2cm}
    \hrule height 2pt
    \vspace{0.3cm}
\end{center}

% Topic and Section Information
\noindent
\begin{tabular}{@{}ll}
    \textbf{\large TOPIC:} & \textcolor{headercolor}{\large \topicname} \\[0.1cm]
    \textbf{\large SECTION:} & \textcolor{headercolor}{\large \sectionnumber~--~\sectionname} \\
\end{tabular}

\vspace{0.25cm}

% 1. TL;DR Section
\section{TL;DR (The Gist)}
\begin{tldrbox}
\begin{itemize}
    \item Electronics is the field that manipulates electrical current to perform useful and interesting tasks, going beyond simple electrical devices that just convert energy to heat, light, or motion
    \item The Edison Effect in 1883 marked the birth of electronics when Edison discovered current flow from a heated filament, creating the world's first electronic device
    \item Electronic circuits use components (resistors, capacitors, diodes, transistors, ICs) mounted on circuit boards to control and manipulate electric current flow in meaningful ways
\end{itemize}
\end{tldrbox}

\vspace{0.25cm}

% 2. Detailed Explanation Section
\section{Detailed Explanation}
\begin{detailbox}

\textbf{Working Principle:}
\begin{itemize}
    \item Electronics manipulates electrical current by adding meaningful information (audio, video, control signals) rather than just converting electrical energy into other forms
    \item Electric current is the flow of free electrons through conductors (like copper) when motivated by voltage
    \item Electronic components work together on circuit boards to bend, twist, restrict, strengthen, or direct current flow to achieve specific functions
\end{itemize}

\vspace{0.15cm}

\textbf{Key Components \& Their Roles:}
\begin{itemize}
    \item \textbf{Atoms \& Electrons:} Foundation of electricity - atoms (especially copper) have loosely bound valence electrons that can move freely
    \item \textbf{Voltage:} The pushing force (like pressure in a water pipe) that motivates electrons to flow
    \item \textbf{Electronic Components:} Resistors (restrict flow like speed bumps), capacitors (smooth variations/store energy), diodes (one-way flow), transistors (amplify/switch), ICs (complex functions)
    \item \textbf{Circuit Boards (PCBs):} Provide pathways (copper/silver traces) connecting components in specific order
\end{itemize}

\vspace{0.15cm}

\textbf{Step-by-Step Operation:}
\begin{enumerate}
    \item Voltage source applies pressure to free electrons in conductor material
    \item Electrons begin to flow from negative to positive terminal, creating electric current
    \item Current flows through circuit board traces to various electronic components
    \item Each component manipulates the current in specific ways (restricting, amplifying, directing, storing)
    \item Components work in sequence to add meaningful information or perform control functions
    \item Final output delivers useful result (sound, images, control signals, computation)
\end{enumerate}

\vspace{0.15cm}

\textbf{Important Formulas \& Theory:}

\textit{Atomic Structure:}
\begin{itemize}
    \item Nucleus (protons + neutrons) + orbiting electrons
    \item Atomic Number = number of protons (defines element)
    \item Copper: 29 protons, 29 electrons, 1 valence electron
\end{itemize}

\textit{Electrical Units:}
\begin{align*}
    \text{Voltage:} \quad & \text{Measured in Volts (V)} \\
    \text{Current:} \quad & \text{Measured in Amperes (A)} \\
    \text{Power:} \quad & \text{Measured in Watts (W)}
\end{align*}

\textbf{Key Relationships:}
\begin{itemize}
    \item Higher voltage $\rightarrow$ more electrons can flow (like higher water pressure)
    \item Free valence electrons in conductors enable current flow
    \item Electronic devices manipulate current vs. electrical devices convert energy
\end{itemize}

\end{detailbox}

\vspace{0.25cm}

% 3. Practical Example & Numerical Section
\section{Practical Example \& Numerical}
\begin{examplebox}

\textbf{Practical Application - Edison's First Electronic Device (1883):}
\\[0.1cm]
Edison's team noticed carbon particles from the heated filament darkened one end of the light bulb. They inserted a third wire to test if electric charge was flowing from the filament. This discovery led to a voltage monitoring and regulation device - the first practical electronic circuit that could automatically adjust voltage levels.

\vspace{0.2cm}

\textbf{Circuit Board Analysis:}
\\[0.1cm]
When examining a typical circuit board:
\begin{itemize}
    \item \textbf{Component Side:} Contains resistors, capacitors, diodes, transistors, ICs arranged like ``little buildings in a city''
    \item \textbf{Trace Side:} Silver/copper conductive pathways connect components like ``streets connecting buildings''
    \item \textbf{Function:} Components manipulate current while traces route it in correct sequence
\end{itemize}

\vspace{0.2cm}

\textbf{Numerical Analysis - Copper as Conductor:}

\begin{align*}
    \text{Atomic Number:} \quad & 29 \text{ protons} \\
    \text{Electrons:} \quad & 29 \text{ (equal to protons in neutral atom)} \\
    \text{Valence Electrons:} \quad & 1 \text{ (loosely bound in outer shell)}
\end{align*}

\textbf{Why Copper?} That single valence electron is very easy to move, making copper an excellent conductor.

\vspace{0.15cm}

\textbf{Voltage Comparison:}
\begin{align*}
    \text{Household electricity:} \quad & 120\,\text{V} \\
    \text{Flashlight battery:} \quad & 1.5\,\text{V} \\
    \text{Car battery:} \quad & 12\,\text{V}
\end{align*}

\textbf{Power Ratio Calculation:}
\begin{equation*}
    \text{Ratio} = \frac{120\,\text{V}}{1.5\,\text{V}} = \boxed{80\times \text{ more voltage}}
\end{equation*}

\end{examplebox}

\vspace{0.25cm}

% 4. Key Points Section
\section{Key Points (Interview Focus)}
\begin{keypointsbox}

\textbf{Essential Facts to Remember:}
\begin{enumerate}
    \item \textbf{Electronics vs. Electrical:} Electrical devices convert energy (light, heat, motion); Electronic devices manipulate current to add information and perform complex functions
    \item \textbf{Edison Effect (1883):} Birth of electronics - discovered current flow from heated filament to third wire, creating first electronic device for voltage regulation
    \item \textbf{Electric Current:} Flow of free electrons through conductors, motivated by voltage pressure
    \item \textbf{Atomic Foundation:} Valence electrons (outer orbit) can escape atoms and become free electrons for conduction
    \item \textbf{Circuit Board Components:} Resistors (restrict), capacitors (smooth/store), diodes (one-way), transistors (amplify/switch), ICs (complex tasks)
    \item \textbf{Voltage as Motivation:} Like water pressure - higher voltage pushes more electrons through conductor
    \item \textbf{Safety Warning:} Capacitors can store lethal electrical energy even when device is unplugged
\end{enumerate}

\vspace{0.15cm}

\textbf{Common Interview Questions:}
\begin{itemize}
    \item \textbf{Q1:} What is the fundamental difference between electrical and electronic devices? \\
    \textit{A:} Electrical devices convert electrical energy into other forms (heat, light, motion). Electronic devices manipulate the electrical current itself to add information and perform complex functions.
    
    \item \textbf{Q2:} What was the Edison Effect and why is it significant? \\
    \textit{A:} In 1883, Edison discovered current flow from a heated filament to a third wire in a light bulb. This is considered the birth of electronics and led to the world's first electronic device.
    
    \item \textbf{Q3:} What is electric current at the atomic level? \\
    \textit{A:} Electric current is the flow of free valence electrons through conductor materials, motivated by voltage (electrical pressure).
    
    \item \textbf{Q4:} Why is copper commonly used as a conductor? \\
    \textit{A:} Copper atoms have one loosely bound valence electron in their outer shell that is very easy to move, making it an excellent conductor.
\end{itemize}

\vspace{0.15cm}

\textbf{Typical Applications:}
\begin{itemize}
    \item Audio electronics: Adding sound information to current (phones, music players, speakers)
    \item Video electronics: Adding image information to current (TVs, cameras, displays)
    \item Control electronics: Monitoring and adjusting voltage/current levels automatically
    \item Communication: Telegraph (1830s), transatlantic cable (1866), modern cell phones
\end{itemize}

\vspace{0.15cm}

\textbf{Limitations \& Safety:}
\begin{itemize}
    \item Electrical safety: Can deliver fatal shocks (100+ years of electric chair use)
    \item Capacitors store dangerous energy even when unplugged - major safety hazard
    \item Component sensitivity: Damage from heat, moisture, static electricity
    \item Complexity: Requires understanding of voltage, current, watts, amps relationships
\end{itemize}

\end{keypointsbox}

% Footer
\vspace{0.15cm}
\hrule height 1pt
\begin{center}
    \small\textit{Electronics Study Sheet | Section 01 | Generated on \today}
\end{center}

\end{document}
