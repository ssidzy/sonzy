\documentclass[a4paper,9pt]{article}

% Packages
\usepackage[margin=0.4in]{geometry}
\usepackage{amsmath}
\usepackage{amssymb}
\usepackage{enumitem}
\usepackage{xcolor}
\usepackage{tcolorbox}
\usepackage{titlesec}
\usepackage{fancyhdr}
\usepackage[hidelinks]{hyperref}
\usepackage{tocloft}
\usepackage{graphicx}

% ====================================================================
% CONFIGURATION - Section 04: Fundamentals
% ====================================================================
\newcommand{\topicname}{Ohm's Law}
\newcommand{\sectionnumber}{Section 04}
\newcommand{\sectionname}{FUNDAMENTALS}
% ====================================================================

% Color definitions
\definecolor{headercolor}{RGB}{13,71,161}        % Deep Blue - Main header
\definecolor{sectioncolor}{RGB}{26,115,232}      % Bright Blue - Section borders
\definecolor{tldrcolor}{RGB}{255,245,230}        % Light Orange - TL;DR background
\definecolor{tldrborder}{RGB}{255,152,0}         % Orange - TL;DR border
\definecolor{detailcolor}{RGB}{232,245,233}      % Light Green - Detail background
\definecolor{detailborder}{RGB}{56,142,60}       % Green - Detail border
\definecolor{examplecolor}{RGB}{255,243,224}     % Light Amber - Example background
\definecolor{exampleborder}{RGB}{245,124,0}      % Deep Orange - Example border
\definecolor{keypointcolor}{RGB}{227,242,253}    % Light Blue - Key Points background
\definecolor{keypointborder}{RGB}{25,118,210}    % Blue - Key Points border
\definecolor{boxcolor}{RGB}{240,248,255}         % Light blue for decorative boxes

% Custom box styles
\tcbuselibrary{skins,breakable}
\newtcolorbox{tldrbox}{
    colback=tldrcolor,
    colframe=tldrborder,
    fonttitle=\bfseries\large,
    title=1. TL;DR (The Gist),
    breakable,
    rounded corners,
    boxrule=2pt,
    top=6pt,
    bottom=6pt,
    left=8pt,
    right=8pt
}

\newtcolorbox{detailbox}{
    colback=detailcolor,
    colframe=detailborder,
    fonttitle=\bfseries\large,
    title=2. Detailed Explanation,
    breakable,
    rounded corners,
    boxrule=2pt,
    top=6pt,
    bottom=6pt,
    left=8pt,
    right=8pt
}

\newtcolorbox{examplebox}{
    colback=examplecolor,
    colframe=exampleborder,
    fonttitle=\bfseries\large,
    title=3. Practical Example \& Numerical,
    breakable,
    rounded corners,
    boxrule=2pt,
    top=6pt,
    bottom=6pt,
    left=8pt,
    right=8pt
}

\newtcolorbox{keypointsbox}{
    colback=keypointcolor,
    colframe=keypointborder,
    fonttitle=\bfseries\large,
    title=4. Key Points (Interview Focus),
    breakable,
    rounded corners,
    boxrule=2pt,
    top=6pt,
    bottom=6pt,
    left=8pt,
    right=8pt
}

% Section formatting
\titleformat{\section}
  {\normalfont\Large\bfseries\color{sectioncolor}}
  {\thesection}{1em}{}

\titleformat{\subsection}
  {\normalfont\large\bfseries\color{sectioncolor}}
  {\thesubsection}{1em}{}

% Remove page numbers
\pagestyle{empty}

% Compact lists
\setlist{nosep, leftmargin=15pt, itemsep=2pt}

% Table of Contents formatting
\renewcommand{\cftsecleader}{\cftdotfill{\cftdotsep}}
\setlength{\cftbeforesecskip}{5pt}
\renewcommand{\cftsecfont}{\normalsize\bfseries\color{sectioncolor}}
\renewcommand{\cftsecpagefont}{\normalsize\color{sectioncolor}}

% Document start
\begin{document}

% ====================================================================
% TITLE PAGE
% ====================================================================
\begin{titlepage}
    \centering
    \vspace*{2cm}
    
    % Title
    {\Huge\bfseries\color{headercolor} ELECTRONICS\\[0.3cm] STUDY SHEET\par}
    \vspace{1cm}
    
    % Decorative line
    {\color{sectioncolor}\rule{0.8\textwidth}{2pt}\par}
    \vspace{1.5cm}
    
    % Topic and Section
    {\LARGE\color{headercolor}\topicname\par}
    \vspace{0.5cm}
    {\Large\color{sectioncolor}\sectionnumber~--~\sectionname\par}
    \vspace{2cm}
    
    % Description box
    \begin{tcolorbox}[
        colback=boxcolor,
        colframe=sectioncolor,
        width=0.8\textwidth,
        arc=3mm,
        boxrule=1.5pt
    ]
    \centering
    \large
    \textbf{Comprehensive One-Page Study Guide}\\[0.3cm]
    \normalsize
    Covering fundamental concepts, formulas, examples,\\
    and interview-focused key points
    \end{tcolorbox}
    
    \vfill
    
    % Bottom information
    {\large\color{headercolor}
    Electronics Engineering Reference Series\\[0.2cm]
    \normalsize
    \today
    \par}
    
\end{titlepage}

% Reset geometry for content page
\newgeometry{margin=0.4in}

% ====================================================================
% TABLE OF CONTENTS
% ====================================================================
\begin{center}
    {\LARGE\bfseries\color{headercolor} TABLE OF CONTENTS}
    \vspace{0.3cm}
    \hrule height 2pt
\end{center}
\vspace{0.3cm}

\tableofcontents
\vspace{0.5cm}

\begin{center}
\hrule height 1pt
\end{center}

\newpage

% ====================================================================
% MAIN CONTENT PAGE
% ====================================================================

% Suppress section numbering
\setcounter{secnumdepth}{0}

% Header
\begin{center}
    {\Huge\bfseries\color{headercolor} ELECTRONICS STUDY SHEET}
    \vspace{0.2cm}
    \hrule height 2pt
    \vspace{0.3cm}
\end{center}

% Topic and Section Information
\noindent
\begin{tabular}{@{}ll}
    \textbf{\large TOPIC:} & \textcolor{headercolor}{\large \topicname} \\[0.1cm]
    \textbf{\large SECTION:} & \textcolor{headercolor}{\large \sectionnumber~--~\sectionname} \\
\end{tabular}

\vspace{0.25cm}

% 1. TL;DR Section
\section{TL;DR (The Gist)}
\begin{tldrbox}
\begin{itemize}
    \item Ohm's Law states the relationship between voltage, current, and resistance: $V = I \times R$
    \item It's the fundamental principle for analyzing most electrical circuits
    \item Works for linear, ohmic materials under constant temperature
\end{itemize}
\end{tldrbox}

\vspace{0.25cm}

% 2. Detailed Explanation Section
\section{Detailed Explanation}
\begin{detailbox}

\textbf{Working Principle:}
\begin{itemize}
    \item Voltage (V) across a conductor is directly proportional to current (I) flowing through it
    \item Resistance (R) is the constant of proportionality between voltage and current
    \item Higher resistance means more voltage needed to push the same current
\end{itemize}

\vspace{0.15cm}

\textbf{Key Components \& Their Roles:}
\begin{itemize}
    \item \textbf{Voltage (V):} Electrical pressure measured in Volts (V) - the driving force
    \item \textbf{Current (I):} Flow of electrons measured in Amperes (A) - the charge flow
    \item \textbf{Resistance (R):} Opposition to current flow measured in Ohms ($\Omega$) - the restrictor
\end{itemize}

\vspace{0.15cm}

\textbf{Step-by-Step Operation:}
\begin{enumerate}
    \item Voltage source creates electric potential difference across conductor
    \item Electric field established causes free electrons to drift
    \item Resistance in material impedes electron flow
    \item Steady-state current established based on $V = I \times R$ relationship
\end{enumerate}

\vspace{0.15cm}

\textbf{Important Formulas \& Theory:}
\begin{align*}
    \text{Ohm's Law:} \quad & V = I \times R \\
    \text{Current:} \quad & I = \frac{V}{R} \\
    \text{Resistance:} \quad & R = \frac{V}{I} \\
    \text{Power:} \quad & P = V \times I = I^2 R = \frac{V^2}{R}
\end{align*}

\textbf{Key relationships:}
\begin{itemize}
    \item Voltage and current are directly proportional (constant R)
    \item Current and resistance are inversely proportional (constant V)
\end{itemize}

\end{detailbox}

\vspace{0.25cm}

% 3. Practical Example & Numerical Section
\section{Practical Example \& Numerical}
\begin{examplebox}

\textbf{Practical Application:}
\\[0.1cm]
LED circuit design: Calculating the current-limiting resistor needed to protect an LED when powered by a battery.

\vspace{0.2cm}

\textbf{Numerical Problem:}
\\[0.1cm]
\textit{Problem Statement:} 
A 12V battery is connected across a resistor, and 2A of current flows through it. Calculate the resistance value and power dissipated.

\vspace{0.15cm}

\textbf{Solution:}
\begin{align*}
    \text{Given:} \quad & V = 12\,\text{V} \\
                        & I = 2\,\text{A} \\
    \text{Find:} \quad  & R = ?\,, \quad P = ?
\end{align*}

\textbf{Step 1:} Apply Ohm's Law to find resistance
\begin{equation*}
    R = \frac{V}{I} = \frac{12\,\text{V}}{2\,\text{A}} = 6\,\Omega
\end{equation*}

\textbf{Step 2:} Calculate power using $P = V \times I$
\begin{equation*}
    P = V \times I = 12\,\text{V} \times 2\,\text{A} = 24\,\text{W}
\end{equation*}

\textbf{Step 3:} Verify using alternate power formula
\begin{equation*}
    P = I^2 R = (2)^2 \times 6 = 24\,\text{W} \quad \checkmark
\end{equation*}

\textbf{Answer:} \boxed{R = 6\,\Omega, \quad P = 24\,\text{W}}

\end{examplebox}

\vspace{0.25cm}

% 4. Key Points Section
\section{Key Points (Interview Focus)}
\begin{keypointsbox}

\textbf{Essential Facts to Remember:}
\begin{enumerate}
    \item Ohm's Law: $V = I \times R$ - fundamental circuit analysis equation
    \item Only applies to ohmic materials (linear V-I relationship)
    \item Temperature affects resistance; most materials increase R with temperature
    \item Can be rearranged: $I = V/R$ or $R = V/I$ depending on what you're solving for
    \item Combined with power formulas: $P = VI = I^2R = V^2/R$
    \item Used in series/parallel circuit analysis with Kirchhoff's laws
    \item Essential for component selection (resistor values, current ratings)
\end{enumerate}

\vspace{0.15cm}

\textbf{Common Interview Questions:}
\begin{itemize}
    \item \textbf{Q1:} What is Ohm's Law and when does it apply?
    \item \textbf{Q2:} What happens to current if resistance doubles while voltage stays constant?
    \item \textbf{Q3:} Why doesn't Ohm's Law apply to semiconductors like diodes?
    \item \textbf{Q4:} How do you calculate power dissipation using Ohm's Law?
\end{itemize}

\vspace{0.15cm}

\textbf{Typical Applications:}
\begin{itemize}
    \item Current-limiting resistor calculations for LEDs
    \item Voltage divider circuits
    \item Power supply design and load calculations
    \item Sensor interface circuits
\end{itemize}

\vspace{0.15cm}

\textbf{Limitations:}
\begin{itemize}
    \item Doesn't apply to non-ohmic devices (diodes, transistors, LEDs)
    \item Assumes constant temperature (resistance changes with heat)
    \item Not valid for AC reactive components (capacitors, inductors) without complex impedance
\end{itemize}

\end{keypointsbox}

% Footer
\vspace{0.15cm}
\hrule height 1pt
\begin{center}
    \small\textit{Electronics Study Sheet | Generated on \today}
\end{center}

\end{document}
