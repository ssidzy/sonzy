% ====================================================================
% SECTION 06: Electric Power P[W] Fundamentals
% ====================================================================

\section{Section 06 -- Electric Power P[W] Fundamentals}

% --------------------------------------------------------------------
\subsection{Electric Power - Introduction}

\noindent\textbf{\color{accentcolor} TL;DR (The Gist)}
\begin{tldrbox}
\begin{itemize}
    \item \textbf{Power:} Rate of energy transfer over time
    \item \textbf{Electric Power:} How fast electrical energy is converted to other forms
    \item Energy cannot be created/destroyed - only transformed
    \item Producers supply power (batteries), consumers use power (LEDs, motors)
\end{itemize}
\end{tldrbox}

\vspace{0.2cm}

\noindent\textbf{\color{accentcolor} Detailed Explanation}
\begin{detailbox}
\textbf{Why Power Matters:}

\textit{Two fundamental reasons:}
\begin{itemize}
    \item \textbf{Cost:} Energy costs money - batteries and electricity aren't free
    \item \textbf{Safety:} Energy can be harmful (heat, radiation, sound, nuclear)
\end{itemize}

\textit{Power measures:}
\begin{itemize}
    \item How fast pennies drain from your wallet
    \item How fast energy is being used or produced
    \item Potential for component damage (overheating, smoking resistors)
\end{itemize}

\vspace{0.15cm}

\textbf{What is Electric Power?}

\textit{General definition:}
\begin{itemize}
    \item Power = rate at which energy is transferred or transformed
    \item "Rate" means per unit time
    \item More power = faster energy transfer
\end{itemize}

\textit{Energy basics:}
\begin{itemize}
    \item Energy = ability to move something or make change happen
    \item Energy cannot be created or destroyed (conservation law)
    \item Energy can only be \textbf{transformed} from one form to another
\end{itemize}

\vspace{0.15cm}

\textbf{Forms of Energy:}

\textit{Many types exist:}
\begin{itemize}
    \item \textbf{Mechanical:} Motion, kinetic energy
    \item \textbf{Electrical:} Moving electrons, current flow
    \item \textbf{Chemical:} Batteries, fuel
    \item \textbf{Electromagnetic:} Light, radio waves
    \item \textbf{Thermal:} Heat energy
    \item \textbf{Nuclear:} Atomic reactions
    \item \textbf{Sound:} Acoustic energy
\end{itemize}

\vspace{0.15cm}

\textbf{Energy Transformations in Electronics:}

\textit{Electronics converts energy to/from electrical form:}

\textbf{Electric to Light (LED):}
\begin{itemize}
    \item Electrical energy $\rightarrow$ Electromagnetic energy
    \item LED consumes electric power, produces light
    \item Some energy lost as heat (byproduct)
\end{itemize}

\textbf{Electric to Motion (Motor):}
\begin{itemize}
    \item Electrical energy $\rightarrow$ Mechanical energy
    \item Motor consumes electric power, produces rotation
    \item Also produces heat and sound (losses)
\end{itemize}

\textbf{Electric to Sound (Buzzer):}
\begin{itemize}
    \item Electrical energy $\rightarrow$ Acoustic energy
    \item Buzzer consumes electric power, produces sound waves
\end{itemize}

\textbf{Chemical to Electric (Battery):}
\begin{itemize}
    \item Chemical energy $\rightarrow$ Electrical energy
    \item Battery produces electric power from chemical reactions
    \item 9V alkaline battery converts chemical potential to voltage
\end{itemize}

\vspace{0.15cm}

\textbf{Electric Energy Flow:}

\textit{How electric energy works:}
\begin{enumerate}
    \item Starts as \textbf{electric potential energy} (voltage)
    \item Electrons flow through potential difference
    \item Potential energy converts to \textbf{electric energy} (current $\times$ voltage)
    \item Electric energy transforms to other useful forms (light, heat, motion)
\end{enumerate}

\vspace{0.15cm}

\textbf{Power Consumers vs Producers:}

\textit{Consumers:}
\begin{itemize}
    \item Transform electric energy INTO another form
    \item Examples: Resistors (heat), LEDs (light), motors (motion)
    \item \textbf{Consume} or \textbf{dissipate} power
    \item Power is positive (energy leaving electrical form)
\end{itemize}

\textit{Producers:}
\begin{itemize}
    \item Transform energy FROM another form into electric
    \item Examples: Batteries (chemical$\rightarrow$electric), solar cells (light$\rightarrow$electric)
    \item \textbf{Supply} or \textbf{generate} power
    \item Power is supplied to circuit
\end{itemize}

\vspace{0.15cm}

\textbf{Power Measures Two Things:}

\textit{Electric power combines:}
\begin{enumerate}
    \item \textbf{How much:} Amount of electric energy transferred (Joules)
    \item \textbf{How fast:} Rate of transfer (per second)
\end{enumerate}

\textit{Result:}
\begin{itemize}
    \item Power = Energy / Time
    \item Units: Joules per second
    \item Standard name: Watt (W)
\end{itemize}

\vspace{0.15cm}

\textbf{Conservation in Circuits:}

\textit{Energy conservation law:}
\begin{itemize}
    \item Power supplied = Power consumed (in steady state)
    \item Battery power out = Sum of all component powers
    \item No energy created or destroyed
    \item All supplied energy eventually becomes heat (usually)
\end{itemize}

\textit{Example circuit:}
\begin{itemize}
    \item Battery supplies 10W
    \item LED consumes 2W (1.8W light + 0.2W heat)
    \item Resistor consumes 8W (all becomes heat)
    \item Total consumed: 2W + 8W = 10W $\checkmark$
\end{itemize}
\end{detailbox}

\vspace{0.2cm}

\noindent\textbf{\color{accentcolor} Practical Example \& Numerical}
\begin{examplebox}
\textbf{Example 1: Energy Transformations}

\textit{LED circuit:}
\begin{itemize}
    \item 5V battery powers LED through resistor
    \item Battery: Chemical $\rightarrow$ Electric (produces power)
    \item LED: Electric $\rightarrow$ Light + Heat (consumes power)
    \item Resistor: Electric $\rightarrow$ Heat (consumes power)
\end{itemize}

\textbf{Energy flow:}
\begin{equation*}
    \text{Chemical} \xrightarrow{\text{Battery}} \text{Electric} \xrightarrow{\text{LED+Resistor}} \text{Light + Heat}
\end{equation*}

\vspace{0.2cm}

\textbf{Example 2: Power Scale Examples}

\textit{Different power levels in real devices:}

\textbf{Arduino (Microcontroller):}
\begin{itemize}
    \item Operating power: 100-500 mW (milliwatts)
    \item Low power sleep mode: 1-10 mW
    \item Range: Microwatts to milliwatts
\end{itemize}

\textbf{LED Lighting:}
\begin{itemize}
    \item Small indicator LED: 20-100 mW
    \item Standard LED bulb: 5-15 W
    \item High-power LED: 1-10 W
\end{itemize}

\textbf{Laptop Computer:}
\begin{itemize}
    \item Typical operation: 30-65 W
    \item Heavy load: 100-150 W
    \item Range: Standard watts
\end{itemize}

\textbf{Household:}
\begin{itemize}
    \item Average house: 1-5 kW (kilowatts)
    \item Air conditioner: 2-5 kW
    \item Electric stove: 2-3 kW
\end{itemize}

\textbf{Large Stadium:}
\begin{itemize}
    \item Lighting and facilities: 1-10 MW (megawatts)
    \item Major sporting event: 5-20 MW
\end{itemize}

\textbf{Power Station:}
\begin{itemize}
    \item Large coal plant: 500-2000 MW
    \item Nuclear plant: 1-2 GW (gigawatts)
    \item Hydroelectric dam: 1-10 GW
\end{itemize}

\vspace{0.2cm}

\textbf{Example 3: Motor Energy Conversion}

\textit{DC motor powered by 12V battery:}

\textbf{Energy transformations:}
\begin{itemize}
    \item Input: 12V $\times$ 500mA = 6W electrical
    \item Output: Mechanical rotation (shaft work)
    \item Losses: Heat (70\%), Sound (5\%)
    \item Useful mechanical: ~25\% = 1.5W
\end{itemize}

\textbf{Energy accounting:}
\begin{itemize}
    \item 6W supplied (chemical$\rightarrow$electric in battery)
    \item 1.5W useful mechanical work
    \item 4.2W lost as heat
    \item 0.3W lost as sound
    \item Total: 6W consumed = 6W supplied $\checkmark$
\end{itemize}

\vspace{0.2cm}

\textbf{Example 4: Battery Lifetime Estimation}

\textit{Given:}
\begin{itemize}
    \item 9V battery capacity: 500 mAh (milliamp-hours)
    \item Circuit draws: 50 mA constant
\end{itemize}

\textbf{Battery life:}
\begin{equation*}
    \text{Time} = \frac{\text{Capacity}}{\text{Current}} = \frac{500\,\text{mAh}}{50\,\text{mA}} = 10\text{ hours}
\end{equation*}

\textbf{Energy used:}
\begin{itemize}
    \item Power: $P = 9V \times 50mA = 450mW = 0.45W$
    \item Time: 10 hours = 36,000 seconds
    \item Energy: $E = P \times t = 0.45 \times 36000 = 16,200\text{ J}$
\end{itemize}

\textbf{Cost (if from wall outlet):}
\begin{itemize}
    \item Energy: 0.45W $\times$ 10h = 4.5 Wh = 0.0045 kWh
    \item At \$0.12/kWh: 0.0045 $\times$ 0.12 = \$0.00054 (\textasciitilde{}0.05 cents)
\end{itemize}

\vspace{0.2cm}

\textbf{Example 5: Power Prefix Practice}

\textit{Convert between units:}

\textbf{Microwatts to watts:}
\begin{equation*}
    500\,\mu W = 500 \times 10^{-6}W = 0.0005W
\end{equation*}

\textbf{Milliwatts to watts:}
\begin{equation*}
    250\,mW = 250 \times 10^{-3}W = 0.25W
\end{equation*}

\textbf{Kilowatts to watts:}
\begin{equation*}
    3.5\,kW = 3.5 \times 10^{3}W = 3500W
\end{equation*}

\textbf{Megawatts to watts:}
\begin{equation*}
    2\,MW = 2 \times 10^{6}W = 2,000,000W
\end{equation*}

\textbf{Gigawatts to watts:}
\begin{equation*}
    1.21\,GW = 1.21 \times 10^{9}W = 1,210,000,000W
\end{equation*}

(Yes, that's a Back to the Future reference - DeLorean time machine!)
\end{examplebox}

\vspace{0.2cm}

\noindent\textbf{\color{accentcolor} Key Points (Interview Focus)}
\begin{keypointsbox}
\begin{enumerate}
    \item \textbf{Power:} Rate of energy transfer (Energy/Time)
    \item \textbf{Conservation:} Energy cannot be created/destroyed, only transformed
    \item \textbf{Consumers:} Convert electric$\rightarrow$other (LEDs, motors, resistors)
    \item \textbf{Producers:} Convert other$\rightarrow$electric (batteries, solar cells)
    \item \textbf{Forms:} Electrical, mechanical, chemical, electromagnetic, thermal, sound
    \item \textbf{Why care:} Cost (money) and safety (heat, damage)
    \item \textbf{Circuit balance:} Power supplied = Power consumed
    \item \textbf{Transformations:} Always involve some loss (usually heat)
\end{enumerate}

\textbf{Interview Questions:}
\begin{itemize}
    \item \textbf{Q:} What is electric power? \\
    \textit{A:} Rate at which electrical energy is transferred or transformed.
    
    \item \textbf{Q:} Can energy be created or destroyed? \\
    \textit{A:} No - only transformed from one form to another (conservation law).
    
    \item \textbf{Q:} What energy transformation happens in an LED? \\
    \textit{A:} Electrical energy $\rightarrow$ Electromagnetic (light) + Heat.
    
    \item \textbf{Q:} Is a battery a power consumer or producer? \\
    \textit{A:} Producer - converts chemical energy to electrical.
    
    \item \textbf{Q:} Is a resistor a power consumer or producer? \\
    \textit{A:} Consumer - converts electrical energy to heat.
    
    \item \textbf{Q:} Why do we care about power in circuits? \\
    \textit{A:} Cost (energy costs money) and safety (too much power causes heat/damage).
    
    \item \textbf{Q:} What happens to energy supplied by battery? \\
    \textit{A:} Consumed by components and transformed to other forms (light, heat, motion).
\end{itemize}

\textbf{Energy Transformation Examples:}
\begin{itemize}
    \item LED: Electric $\rightarrow$ Light + Heat
    \item Motor: Electric $\rightarrow$ Mechanical + Heat + Sound
    \item Buzzer: Electric $\rightarrow$ Sound + Heat
    \item Resistor: Electric $\rightarrow$ Heat (100\%)
    \item Battery: Chemical $\rightarrow$ Electric
    \item Solar cell: Light $\rightarrow$ Electric
\end{itemize}

\textbf{Power Scale Reference:}
\begin{itemize}
    \item $\mu$W (microwatt): $10^{-6}$ W - Low-power sensors
    \item mW (milliwatt): $10^{-3}$ W - Arduino, small LEDs
    \item W (watt): $10^{0}$ W - Laptops, light bulbs
    \item kW (kilowatt): $10^{3}$ W - Household appliances
    \item MW (megawatt): $10^{6}$ W - Large buildings, stadiums
    \item GW (gigawatt): $10^{9}$ W - Power stations
\end{itemize}

\textbf{Common Misconceptions:}
\begin{itemize}
    \item Power is NOT voltage (voltage is potential)
    \item Power is NOT current (current is flow rate)
    \item Current doesn't "get used up" (energy does)
    \item More voltage doesn't always mean more power (depends on current too)
\end{itemize}
\end{keypointsbox}

% --------------------------------------------------------------------
\subsection{Wattage}

\noindent\textbf{\color{accentcolor} TL;DR (The Gist)}
\begin{tldrbox}
\begin{itemize}
    \item \textbf{Watt (W):} Unit of power = Joules per second
    \item Energy measured in Joules; Power = Energy/Time
    \item Symbol: W (uppercase, not to confuse with "w" for work)
    \item Prefixes: $\mu$W, mW, W, kW, MW, GW (depends on application)
\end{itemize}
\end{tldrbox}

\vspace{0.2cm}

\noindent\textbf{\color{accentcolor} Detailed Explanation}
\begin{detailbox}
\textbf{The Watt - Official Unit of Power:}

\vspace{0.15cm}

\textbf{Definition:}

\textit{Power measures energy transfer rate:}
\begin{itemize}
    \item Energy measured in \textbf{Joules (J)}
    \item Time measured in \textbf{seconds (s)}
    \item Power = Energy / Time = Joules/second
\end{itemize}

\textit{International System of Units (SI):}
\begin{itemize}
    \item Official name: \textbf{Watt}
    \item Symbol: \textbf{W}
    \item Equivalence: 1 W = 1 J/s
    \item Named after James Watt (Scottish inventor)
\end{itemize}

\vspace{0.15cm}

\textbf{Mathematical Expression:}

\begin{equation*}
    P = \frac{E}{t}
\end{equation*}

Where:
\begin{itemize}
    \item $P$ = Power (Watts)
    \item $E$ = Energy (Joules)
    \item $t$ = Time (seconds)
\end{itemize}

\textit{Rearranging:}
\begin{align*}
    E &= P \times t \quad \text{(Energy from power and time)} \\
    t &= \frac{E}{P} \quad \text{(Time from energy and power)}
\end{align*}

\vspace{0.15cm}

\textbf{Prefixes for Different Scales:}

\textit{Very common in electronics and power systems:}

\textbf{Microwatt ($\mu$W):}
\begin{itemize}
    \item $1\,\mu W = 10^{-6}$ W = 0.000001 W
    \item Ultra-low power devices
    \item Examples: Watches, pacemakers, wireless sensors
\end{itemize}

\textbf{Milliwatt (mW):}
\begin{itemize}
    \item $1\,mW = 10^{-3}$ W = 0.001 W
    \item Low power electronics
    \item Examples: Arduino, LEDs, small circuits
\end{itemize}

\textbf{Watt (W):}
\begin{itemize}
    \item Base unit = 1 W
    \item Common consumer electronics
    \item Examples: Laptops, light bulbs, phone chargers
\end{itemize}

\textbf{Kilowatt (kW):}
\begin{itemize}
    \item $1\,kW = 10^{3}$ W = 1,000 W
    \item Household and automotive power
    \item Examples: Appliances, electric vehicles, HVAC
\end{itemize}

\textbf{Megawatt (MW):}
\begin{itemize}
    \item $1\,MW = 10^{6}$ W = 1,000,000 W
    \item Industrial and large facilities
    \item Examples: Factories, data centers, large buildings
\end{itemize}

\textbf{Gigawatt (GW):}
\begin{itemize}
    \item $1\,GW = 10^{9}$ W = 1,000,000,000 W
    \item Power generation and distribution
    \item Examples: Power plants, electrical grids, time machines
\end{itemize}

\vspace{0.15cm}

\textbf{Application-Specific Ranges:}

\textbf{Microcontrollers ($\mu$W to mW):}
\begin{itemize}
    \item Arduino Uno: 200-500 mW (active)
    \item Arduino sleep mode: 1-10 mW
    \item ESP32 WiFi: 100-300 mW
    \item ATtiny sleep: 1-100 $\mu$W
\end{itemize}

\textbf{Consumer Electronics (W range):}
\begin{itemize}
    \item LED bulb: 5-15 W
    \item Laptop: 30-100 W
    \item Desktop PC: 100-500 W
    \item Phone charger: 5-20 W
\end{itemize}

\textbf{Household (kW range):}
\begin{itemize}
    \item Refrigerator: 100-800 W (0.1-0.8 kW)
    \item Microwave: 600-1200 W (0.6-1.2 kW)
    \item Air conditioner: 2-5 kW
    \item Electric water heater: 3-5 kW
    \item Whole house: 1-5 kW average
\end{itemize}

\textbf{Large Scale (MW to GW):}
\begin{itemize}
    \item Sports stadium: 5-20 MW
    \item Small town: 10-50 MW
    \item Large city: 1-10 GW
    \item Coal power plant: 500-2000 MW
    \item Nuclear plant: 1-2 GW
\end{itemize}

\vspace{0.15cm}

\textbf{Energy vs Power - Critical Distinction:}

\textit{Energy (Joules):}
\begin{itemize}
    \item \textbf{Amount} of work that can be done
    \item Total capacity
    \item Stored in batteries, capacitors, fuel
    \item Example: "Battery stores 10,000 J"
\end{itemize}

\textit{Power (Watts):}
\begin{itemize}
    \item \textbf{Rate} at which work is done
    \item How fast energy is used/produced
    \item Example: "Circuit consumes 5 W"
\end{itemize}

\textit{Analogy:}
\begin{itemize}
    \item Energy = Water in bucket (amount)
    \item Power = Flow rate from bucket (how fast)
    \item 10 liters flowing at 1 L/s $\rightarrow$ 10 seconds to empty
    \item 10,000 J at 5 W $\rightarrow$ 2,000 seconds to deplete
\end{itemize}

\vspace{0.15cm}

\textbf{Watt-Hour (Wh) - Energy Unit:}

\textit{Common in batteries and utilities:}
\begin{itemize}
    \item Watt-hour (Wh) = Power $\times$ Time
    \item 1 Wh = 1 W for 1 hour = 3,600 J
    \item Kilowatt-hour (kWh) = 1,000 Wh
\end{itemize}

\textit{Why use Wh instead of J?}
\begin{itemize}
    \item More convenient for everyday use
    \item Easier to understand (watts $\times$ hours)
    \item Standard on electricity bills (kWh)
    \item Battery capacity often in mAh or Wh
\end{itemize}

\textit{Example:}
\begin{itemize}
    \item 100W bulb for 10 hours = 1,000 Wh = 1 kWh
    \item At \$0.12/kWh $\rightarrow$ costs \$0.12
\end{itemize}
\end{detailbox}

\vspace{0.2cm}

\noindent\textbf{\color{accentcolor} Practical Example \& Numerical}
\begin{examplebox}
\textbf{Example 1: Energy to Power Conversion}

\textit{Given:}
\begin{itemize}
    \item Device uses 5,000 J of energy
    \item Runs for 50 seconds
\end{itemize}

\textbf{Calculate power:}
\begin{equation*}
    P = \frac{E}{t} = \frac{5000\,J}{50\,s} = \boxed{100\,W}
\end{equation*}

Device operates at 100 watts.

\vspace{0.2cm}

\textbf{Example 2: Power to Energy Conversion}

\textit{Given:}
\begin{itemize}
    \item LED consumes 50 mW
    \item Operates for 2 hours
\end{itemize}

\textbf{Calculate energy:}
\begin{align*}
    P &= 50\,mW = 0.05\,W \\
    t &= 2\,hours = 2 \times 3600 = 7200\,s \\
    E &= P \times t = 0.05 \times 7200 = \boxed{360\,J}
\end{align*}

Also in watt-hours:
\begin{equation*}
    E = 0.05\,W \times 2\,h = 0.1\,Wh
\end{equation*}

\vspace{0.2cm}

\textbf{Example 3: Unit Conversion Practice}

\textbf{Convert 2,500 mW to watts:}
\begin{equation*}
    2500\,mW = 2500 \times 10^{-3} = 2.5\,W
\end{equation*}

\textbf{Convert 0.003 W to microwatts:}
\begin{equation*}
    0.003\,W = 3 \times 10^{-3} = 3\,mW = 3000\,\mu W
\end{equation*}

\textbf{Convert 5 kW to watts:}
\begin{equation*}
    5\,kW = 5 \times 10^{3} = 5000\,W
\end{equation*}

\textbf{Convert 1.5 MW to kW:}
\begin{equation*}
    1.5\,MW = 1.5 \times 10^{6}\,W = 1500\,kW
\end{equation*}

\vspace{0.2cm}

\textbf{Example 4: Electricity Bill Calculation}

\textit{Given:}
\begin{itemize}
    \item 2 kW air conditioner runs 6 hours/day for 30 days
    \item Electricity rate: \$0.12 per kWh
\end{itemize}

\textbf{Energy consumed:}
\begin{align*}
    E &= P \times t \\
    &= 2\,kW \times (6\,h/day \times 30\,days) \\
    &= 2 \times 180 = 360\,kWh
\end{align*}

\textbf{Cost:}
\begin{equation*}
    \text{Cost} = 360\,kWh \times \$0.12/kWh = \boxed{\$43.20}
\end{equation*}

\vspace{0.2cm}

\textbf{Example 5: Battery Runtime Calculation}

\textit{Given:}
\begin{itemize}
    \item Battery capacity: 2,000 mAh at 3.7V
    \item Device power consumption: 500 mW
\end{itemize}

\textbf{Battery energy:}
\begin{align*}
    E &= V \times Q \\
    Q &= 2000\,mAh = 2\,Ah = 2 \times 3600 = 7200\,C \\
    E &= 3.7 \times 7200 = 26{,}640\,J
\end{align*}

Or in watt-hours:
\begin{equation*}
    E = 3.7\,V \times 2\,Ah = 7.4\,Wh
\end{equation*}

\textbf{Runtime:}
\begin{align*}
    t &= \frac{E}{P} = \frac{7.4\,Wh}{0.5\,W} = 14.8\,hours
\end{align*}

Or in Joules:
\begin{equation*}
    t = \frac{26640\,J}{0.5\,W} = 53{,}280\,s = \boxed{14.8\,hours}
\end{equation*}

\vspace{0.2cm}

\textbf{Example 6: Comparing Power Levels}

\textit{Which uses more power?}

\textbf{Option A:} 500 mW for 10 hours
\begin{equation*}
    E_A = 0.5\,W \times 10\,h = 5\,Wh
\end{equation*}

\textbf{Option B:} 2 W for 2 hours
\begin{equation*}
    E_B = 2\,W \times 2\,h = 4\,Wh
\end{equation*}

\textbf{Result:} Option A uses more \textit{energy} (5 Wh vs 4 Wh), but Option B has higher \textit{power} (2 W vs 0.5 W).

Key distinction: Power is rate, energy is total amount!
\end{examplebox}

\vspace{0.2cm}

\noindent\textbf{\color{accentcolor} Key Points (Interview Focus)}
\begin{keypointsbox}
\begin{enumerate}
    \item \textbf{Watt (W):} Unit of power = 1 Joule/second
    \item \textbf{Formula:} $P = E/t$ (Power = Energy / Time)
    \item \textbf{Common prefixes:} $\mu$W, mW, W, kW, MW, GW
    \item \textbf{Energy vs Power:} Energy = amount; Power = rate
    \item \textbf{Watt-hour (Wh):} Energy unit = Power $\times$ Time
    \item \textbf{Electricity billing:} kWh (kilowatt-hours)
    \item \textbf{Named after:} James Watt (Scottish inventor)
    \item \textbf{Symbol:} W (uppercase)
\end{enumerate}

\textbf{Interview Questions:}
\begin{itemize}
    \item \textbf{Q:} What is a watt? \\
    \textit{A:} Unit of power equal to one joule per second.
    
    \item \textbf{Q:} What's the difference between Joules and Watts? \\
    \textit{A:} Joules measure energy (amount), Watts measure power (rate of energy transfer).
    
    \item \textbf{Q:} How many watts in 1 kilowatt? \\
    \textit{A:} 1,000 watts.
    
    \item \textbf{Q:} What is a kilowatt-hour? \\
    \textit{A:} Energy used by 1 kW device for 1 hour = 3.6 million Joules.
    
    \item \textbf{Q:} Device uses 200 mW for 5 hours. Energy consumed? \\
    \textit{A:} E = 0.2 W $\times$ 5 h = 1 Wh = 3,600 J.
    
    \item \textbf{Q:} Typical power range for Arduino? \\
    \textit{A:} 100-500 mW (milliwatt range).
\end{itemize}

\textbf{Power Range Guide:}
\begin{itemize}
    \item $\mu$W: Ultra-low power (watches, sensors)
    \item mW: Low power electronics (Arduino, LEDs)
    \item W: Consumer devices (laptops, bulbs)
    \item kW: Appliances, vehicles
    \item MW: Industrial, large buildings
    \item GW: Power plants, grids
\end{itemize}

\textbf{Unit Conversion Quick Reference:}
\begin{itemize}
    \item 1 W = 1,000 mW = 1,000,000 $\mu$W
    \item 1 kW = 1,000 W
    \item 1 MW = 1,000 kW = 1,000,000 W
    \item 1 GW = 1,000 MW = 1,000,000,000 W
\end{itemize}

\textbf{Common Calculations:}
\begin{itemize}
    \item Power from energy and time: $P = E/t$
    \item Energy from power and time: $E = P \times t$
    \item Time from energy and power: $t = E/P$
    \item Watt-hours to Joules: Wh $\times$ 3,600 = J
    \item Joules to Watt-hours: J / 3,600 = Wh
\end{itemize}
\end{keypointsbox}

% --------------------------------------------------------------------
\subsection{Calculating Power}

\noindent\textbf{\color{accentcolor} TL;DR (The Gist)}
\begin{tldrbox}
\begin{itemize}
    \item \textbf{Basic formula:} $P = V \times I$ (Power = Voltage $\times$ Current)
    \item \textbf{With resistance:} $P = I^2 R$ or $P = V^2 / R$ (using Ohm's Law)
    \item Need any 2 of 3 values (V, I, R) to calculate power
    \item Power in component = voltage drop $\times$ current through it
\end{itemize}
\end{tldrbox}

\vspace{0.2cm}

\noindent\textbf{\color{accentcolor} Detailed Explanation}
\begin{detailbox}
\textbf{Fundamental Power Equation:}

\vspace{0.15cm}

\textbf{Deriving Power from Basic Definitions:}

\textit{What we know:}
\begin{itemize}
    \item \textbf{Voltage (V):} Potential energy per unit charge = Joules/Coulomb
    \item \textbf{Current (I):} Charge flow per unit time = Coulombs/second
    \item \textbf{Power (P):} Energy transfer per unit time = Joules/second
\end{itemize}

\textit{Combining voltage and current:}
\begin{align*}
    P &= \frac{\text{Energy}}{\text{Time}} \\
    &= \frac{\text{Joules}}{\text{second}} \\
    &= \frac{\text{Joules}}{\text{Coulomb}} \times \frac{\text{Coulombs}}{\text{second}} \\
    &= V \times I
\end{align*}

\textbf{Result - Primary Power Formula:}
\begin{equation*}
    \boxed{P = V \times I}
\end{equation*}

Where:
\begin{itemize}
    \item $P$ = Power (Watts)
    \item $V$ = Voltage (Volts)
    \item $I$ = Current (Amperes)
\end{itemize}

\vspace{0.15cm}

\textbf{Using Ohm's Law to Create Alternative Formulas:}

\textit{Ohm's Law reminder:}
\begin{equation*}
    V = I \times R \quad \text{or} \quad I = \frac{V}{R}
\end{equation*}

\vspace{0.15cm}

\textbf{Power Formula 1: Using Voltage and Resistance}

\textit{Start with:} $P = V \times I$

\textit{Substitute:} $I = V/R$

\begin{align*}
    P &= V \times \frac{V}{R} \\
    P &= \frac{V^2}{R}
\end{align*}

\textbf{Result:}
\begin{equation*}
    \boxed{P = \frac{V^2}{R}}
\end{equation*}

\textit{Use when:}
\begin{itemize}
    \item Know voltage across component
    \item Know resistance of component
    \item Don't know current (don't need to calculate it)
\end{itemize}

\vspace{0.15cm}

\textbf{Power Formula 2: Using Current and Resistance}

\textit{Start with:} $P = V \times I$

\textit{Substitute:} $V = I \times R$

\begin{align*}
    P &= (I \times R) \times I \\
    P &= I^2 \times R
\end{align*}

\textbf{Result:}
\begin{equation*}
    \boxed{P = I^2 R}
\end{equation*}

\textit{Use when:}
\begin{itemize}
    \item Know current through component
    \item Know resistance of component
    \item Don't know voltage (don't need to calculate it)
\end{itemize}

\vspace{0.15cm}

\textbf{Three Power Formulas - Summary:}

\begin{center}
\begin{tabular}{|c|c|c|}
\hline
\textbf{Formula} & \textbf{When to Use} & \textbf{Known Values} \\
\hline
$P = V \times I$ & V and I known & Voltage, Current \\
$P = \frac{V^2}{R}$ & V and R known & Voltage, Resistance \\
$P = I^2 R$ & I and R known & Current, Resistance \\
\hline
\end{tabular}
\end{center}

\textit{Key insight:}
\begin{itemize}
    \item Need ANY TWO of three values (V, I, R)
    \item Can always calculate power
    \item Choose formula based on what you know
\end{itemize}

\vspace{0.15cm}

\textbf{Step-by-Step Power Calculation:}

\textbf{Example approach:}
\begin{enumerate}
    \item Identify the component (resistor, LED, etc.)
    \item Find voltage drop ACROSS component
    \item Find current THROUGH component
    \item Multiply: $P = V \times I$
\end{enumerate}

\textit{Alternative if resistance known:}
\begin{enumerate}
    \item Know any two: V, I, or R
    \item Choose appropriate power formula
    \item Calculate power
    \item Check if within component rating
\end{enumerate}

\vspace{0.15cm}

\textbf{Practical Circuit Example:}

\textit{Given:}
\begin{itemize}
    \item 9V battery
    \item 10$\Omega$ resistor
    \item Series circuit
\end{itemize}

\textbf{Step 1 - Find current (Ohm's Law):}
\begin{equation*}
    I = \frac{V}{R} = \frac{9}{10} = 0.9\,A = 900\,mA
\end{equation*}

\textbf{Step 2 - Calculate power (Method 1: V and I):}
\begin{equation*}
    P = V \times I = 9 \times 0.9 = 8.1\,W
\end{equation*}

\textbf{Or Step 2 - Calculate power (Method 2: $V^2/R$):}
\begin{equation*}
    P = \frac{V^2}{R} = \frac{9^2}{10} = \frac{81}{10} = 8.1\,W
\end{equation*}

\textbf{Or Step 2 - Calculate power (Method 3: $I^2R$):}
\begin{equation*}
    P = I^2 R = (0.9)^2 \times 10 = 0.81 \times 10 = 8.1\,W
\end{equation*}

All three methods give same answer!

\vspace{0.15cm}

\textbf{Physical Interpretation:}

\textit{What does 8.1W mean?}
\begin{itemize}
    \item Resistor dissipates 8.1 Joules per second
    \item Electrical energy $\rightarrow$ Heat energy
    \item 8.1W of heat produced continuously
    \item After 1 minute: $8.1 \times 60 = 486$ Joules as heat
    \item After 10 minutes: $8.1 \times 600 = 4,860$ Joules
\end{itemize}

\vspace{0.15cm}

\textbf{When Each Formula is Most Useful:}

\textbf{$P = V \times I$ (Traditional):}
\begin{itemize}
    \item Most direct and intuitive
    \item Use when both V and I are known/measured
    \item Works for any component (not just resistors)
    \item Best for components with varying resistance
\end{itemize}

\textbf{$P = V^2/R$ (Voltage-based):}
\begin{itemize}
    \item Use when voltage is fixed/known
    \item Don't need to calculate current first
    \item Common for components across power supply
    \item Shows: Double voltage = 4$\times$ power (quadratic relationship)
\end{itemize}

\textbf{$P = I^2 R$ (Current-based):}
\begin{itemize}
    \item Use when current is fixed/known
    \item Common in series circuits (same current everywhere)
    \item Don't need to calculate voltage first
    \item Shows: Double current = 4$\times$ power (quadratic relationship)
\end{itemize}

\vspace{0.15cm}

\textbf{Power in Complete Circuit:}

\textit{Conservation principle:}
\begin{itemize}
    \item Power supplied = Power consumed (steady state)
    \item Battery/source power = Sum of all component powers
    \item $P_{source} = P_{R1} + P_{R2} + P_{R3} + \ldots$
\end{itemize}

\textit{Example verification:}
\begin{itemize}
    \item 12V battery, 1A current $\rightarrow$ $P_{source} = 12 \times 1 = 12W$
    \item Three resistors in series: 4W, 5W, 3W
    \item Total consumed: $4 + 5 + 3 = 12W$ $\checkmark$
\end{itemize}
\end{detailbox}

\vspace{0.2cm}

\noindent\textbf{\color{accentcolor} Practical Example \& Numerical}
\begin{examplebox}
\textbf{Example 1: All Three Methods}

\textit{Given:} $R = 100\Omega$, $V = 10V$ across resistor

\textbf{Method 1 - Find I, then P:}
\begin{align*}
    I &= \frac{V}{R} = \frac{10}{100} = 0.1\,A \\
    P &= V \times I = 10 \times 0.1 = \boxed{1\,W}
\end{align*}

\textbf{Method 2 - Use $V^2/R$:}
\begin{align*}
    P &= \frac{V^2}{R} = \frac{10^2}{100} = \frac{100}{100} = \boxed{1\,W}
\end{align*}

\textbf{Method 3 - Use $I^2R$:}
\begin{align*}
    I &= 0.1\,A \text{ (from Method 1)} \\
    P &= I^2 R = (0.1)^2 \times 100 = 0.01 \times 100 = \boxed{1\,W}
\end{align*}

All methods confirm: 1W dissipated!

\vspace{0.2cm}

\textbf{Example 2: Series Circuit Power Distribution}

\textit{Given:}
\begin{itemize}
    \item 12V battery
    \item Three resistors in series: $R_1 = 10\Omega$, $R_2 = 20\Omega$, $R_3 = 30\Omega$
\end{itemize}

\textbf{Total resistance:}
\begin{equation*}
    R_{total} = 10 + 20 + 30 = 60\Omega
\end{equation*}

\textbf{Circuit current:}
\begin{equation*}
    I = \frac{12}{60} = 0.2\,A
\end{equation*}

\textbf{Power in each resistor (using $P = I^2R$):}
\begin{align*}
    P_{R1} &= I^2 R_1 = (0.2)^2 \times 10 = 0.04 \times 10 = 0.4\,W \\
    P_{R2} &= I^2 R_2 = (0.2)^2 \times 20 = 0.04 \times 20 = 0.8\,W \\
    P_{R3} &= I^2 R_3 = (0.2)^2 \times 30 = 0.04 \times 30 = 1.2\,W
\end{align*}

\textbf{Total power:}
\begin{equation*}
    P_{total} = 0.4 + 0.8 + 1.2 = 2.4\,W
\end{equation*}

\textbf{Verify with source power:}
\begin{equation*}
    P_{source} = V \times I = 12 \times 0.2 = 2.4\,W \quad \checkmark
\end{equation*}

\textit{Observation:} Larger resistor dissipates more power in series!

\vspace{0.2cm}

\textbf{Example 3: LED Current Limiting Resistor Power}

\textit{Given:}
\begin{itemize}
    \item 5V supply
    \item LED: $V_f = 2V$, $I_f = 20mA$
    \item Current limiting resistor: $R = ?$
\end{itemize}

\textbf{Find resistor value:}
\begin{align*}
    V_R &= V_{supply} - V_{LED} = 5 - 2 = 3V \\
    R &= \frac{V_R}{I} = \frac{3}{0.02} = 150\Omega
\end{align*}

\textbf{Power in resistor:}
\begin{align*}
    P_R &= V_R \times I = 3 \times 0.02 = \boxed{0.06\,W = 60\,mW}
\end{align*}

Or using $I^2R$:
\begin{equation*}
    P_R = (0.02)^2 \times 150 = 0.0004 \times 150 = 0.06\,W
\end{equation*}

\textbf{Resistor rating needed:} Standard 1/8W (125mW) is sufficient.

\vspace{0.2cm}

\textbf{Example 4: Voltage Doubling Effect on Power}

\textit{Given:} $R = 50\Omega$ resistor

\textbf{At 5V:}
\begin{equation*}
    P_1 = \frac{V^2}{R} = \frac{5^2}{50} = \frac{25}{50} = 0.5\,W
\end{equation*}

\textbf{At 10V (doubled):}
\begin{equation*}
    P_2 = \frac{V^2}{R} = \frac{10^2}{50} = \frac{100}{50} = 2\,W
\end{equation*}

\textbf{Ratio:}
\begin{equation*}
    \frac{P_2}{P_1} = \frac{2}{0.5} = 4
\end{equation*}

\textit{Conclusion:} Doubling voltage quadruples power! (Quadratic relationship)

\vspace{0.2cm}

\textbf{Example 5: Multi-Component Circuit}

\textit{Circuit:}
\begin{itemize}
    \item 15V battery
    \item $R_1 = 100\Omega$ in series with parallel combination
    \item $R_2 = 200\Omega$ and $R_3 = 200\Omega$ in parallel
\end{itemize}

\textbf{Parallel combination:}
\begin{equation*}
    R_{23} = \frac{R_2 \times R_3}{R_2 + R_3} = \frac{200 \times 200}{200 + 200} = 100\Omega
\end{equation*}

\textbf{Total resistance:}
\begin{equation*}
    R_{total} = R_1 + R_{23} = 100 + 100 = 200\Omega
\end{equation*}

\textbf{Main current:}
\begin{equation*}
    I_1 = \frac{15}{200} = 0.075\,A = 75\,mA
\end{equation*}

\textbf{Power in $R_1$:}
\begin{equation*}
    P_1 = I_1^2 R_1 = (0.075)^2 \times 100 = 0.5625\,W
\end{equation*}

\textbf{Voltage across parallel resistors:}
\begin{equation*}
    V_{23} = I_1 \times R_{23} = 0.075 \times 100 = 7.5V
\end{equation*}

\textbf{Current through each parallel resistor:}
\begin{equation*}
    I_2 = I_3 = \frac{7.5}{200} = 0.0375\,A = 37.5\,mA
\end{equation*}

\textbf{Power in $R_2$ and $R_3$:}
\begin{equation*}
    P_2 = P_3 = V \times I = 7.5 \times 0.0375 = 0.28125\,W
\end{equation*}

\textbf{Total power:}
\begin{equation*}
    P_{total} = 0.5625 + 0.28125 + 0.28125 = 1.125\,W
\end{equation*}

\textbf{Verify:}
\begin{equation*}
    P_{source} = 15 \times 0.075 = 1.125\,W \quad \checkmark
\end{equation*}

\vspace{0.2cm}

\textbf{Example 6: Finding Unknown Resistance from Power}

\textit{Given:}
\begin{itemize}
    \item Resistor dissipates 2W
    \item Current through it: 100mA
\end{itemize}

\textbf{Find resistance:}

From $P = I^2 R$:
\begin{align*}
    R &= \frac{P}{I^2} = \frac{2}{(0.1)^2} = \frac{2}{0.01} = \boxed{200\Omega}
\end{align*}

\textbf{Verify with voltage:}
\begin{align*}
    V &= I \times R = 0.1 \times 200 = 20V \\
    P &= V \times I = 20 \times 0.1 = 2W \quad \checkmark
\end{align*}
\end{examplebox}

\vspace{0.2cm}

\noindent\textbf{\color{accentcolor} Key Points (Interview Focus)}
\begin{keypointsbox}
\begin{enumerate}
    \item \textbf{Basic formula:} $P = V \times I$ (most fundamental)
    \item \textbf{With resistance:} $P = V^2/R$ or $P = I^2 R$
    \item \textbf{Need 2 of 3:} Knowing any two (V, I, R) allows power calculation
    \item \textbf{Quadratic relationship:} Double V or I $\rightarrow$ 4$\times$ power
    \item \textbf{Conservation:} $P_{source} = \sum P_{components}$
    \item \textbf{All formulas equivalent:} Choose based on known values
    \item \textbf{Units:} V in volts, I in amps, R in ohms $\rightarrow$ P in watts
    \item \textbf{Energy over time:} $E = P \times t$ (power $\times$ duration)
\end{enumerate}

\textbf{Interview Questions:}
\begin{itemize}
    \item \textbf{Q:} State the three power formulas. \\
    \textit{A:} $P = VI$, $P = V^2/R$, $P = I^2R$.
    
    \item \textbf{Q:} 10V across 100$\Omega$ resistor. Power dissipated? \\
    \textit{A:} $P = V^2/R = 100/100 = 1W$.
    
    \item \textbf{Q:} If voltage doubles, how does power change? \\
    \textit{A:} Quadruples (4$\times$ power), since $P \propto V^2$.
    
    \item \textbf{Q:} Which resistor dissipates more power in series - larger or smaller? \\
    \textit{A:} Larger resistance (same current, $P = I^2R$, so larger R $\rightarrow$ larger P).
    
    \item \textbf{Q:} 12V, 2A through circuit. Total power? \\
    \textit{A:} $P = VI = 12 \times 2 = 24W$.
    
    \item \textbf{Q:} Derive $P = V^2/R$ from $P = VI$. \\
    \textit{A:} Substitute $I = V/R$ into $P = VI$: $P = V(V/R) = V^2/R$.
\end{itemize}

\textbf{Power Formula Selection Guide:}
\begin{itemize}
    \item Known V and I $\rightarrow$ Use $P = VI$
    \item Known V and R $\rightarrow$ Use $P = V^2/R$
    \item Known I and R $\rightarrow$ Use $P = I^2R$
    \item Any two values sufficient!
\end{itemize}

\textbf{Applications:}
\begin{itemize}
    \item Resistor power rating selection
    \item LED current limiting resistor sizing
    \item Circuit power consumption calculation
    \item Battery life estimation
    \item Thermal management (heat dissipation)
    \item Energy cost calculation
\end{itemize}

\textbf{Common Mistakes:}
\begin{itemize}
    \item Using total voltage instead of voltage drop across component
    \item Forgetting to square V or I in formulas
    \item Wrong units (must use Volts, Amps, Ohms)
    \item Confusing power (rate) with energy (total)
\end{itemize}

\textbf{Key Relationships:}
\begin{itemize}
    \item $P \propto V^2$ (at constant R)
    \item $P \propto I^2$ (at constant R)
    \item $P \propto 1/R$ (at constant V)
    \item $P \propto R$ (at constant I)
\end{itemize}
\end{keypointsbox}

% --------------------------------------------------------------------
\subsection{Resistor Power Ratings}

\noindent\textbf{\color{accentcolor} TL;DR (The Gist)}
\begin{tldrbox}
\begin{itemize}
    \item \textbf{Power rating:} Maximum power component can safely dissipate
    \item Common resistor ratings: 1/8W, 1/4W, 1/2W, 1W, 2W
    \item Exceeding rating $\rightarrow$ overheating $\rightarrow$ "magic smoke" (component failure)
    \item Always select resistor with rating > calculated power (safety margin)
\end{itemize}
\end{tldrbox}

\vspace{0.2cm}

\noindent\textbf{\color{accentcolor} Detailed Explanation}
\begin{detailbox}
\textbf{Why Power Ratings Matter:}

\vspace{0.15cm}

\textbf{Energy Transformation and Heat:}

\textit{What happens in components:}
\begin{itemize}
    \item All components transform energy from one type to another
    \item \textbf{Desired transformations:} LEDs$\rightarrow$light, motors$\rightarrow$motion, batteries$\rightarrow$charging
    \item \textbf{Undesired transformations:} Energy losses, usually as heat
    \item Heat is unavoidable byproduct in most circuits
\end{itemize}

\textit{Even "useful" components produce heat:}
\begin{itemize}
    \item LEDs: Mostly light, but some heat
    \item Motors: Mostly motion, but winding resistance creates heat
    \item Batteries: Charging creates heat
    \item All have internal resistance $\rightarrow$ power loss
\end{itemize}

\vspace{0.15cm}

\textbf{Heat and Component Damage:}

\textit{Too much heat is dangerous:}
\begin{itemize}
    \item Components have maximum temperature limits
    \item Exceeding limits causes:
    \begin{itemize}
        \item Physical damage (melting, burning)
        \item Performance degradation
        \item Parameter drift
        \item Complete failure ("letting the magic smoke out")
    \end{itemize}
\end{itemize}

\textit{Power rating definition:}
\begin{itemize}
    \item Maximum power component can dissipate safely
    \item Usually specified at room temperature (25°C)
    \item Higher ambient temperature $\rightarrow$ must derate (use less power)
    \item Exceeding rating $\rightarrow$ overheating $\rightarrow$ damage
\end{itemize}

\vspace{0.15cm}

\textbf{Resistors - Notorious Power Consumers:}

\textit{Why resistors are special concern:}
\begin{itemize}
    \item 100\% of electrical energy $\rightarrow$ heat (no useful output)
    \item Drop voltage AND pass current $\rightarrow$ $P = VI$
    \item More voltage = more current = MORE power (quadratic)
    \item Easy to exceed rating if not careful
\end{itemize}

\textit{Example of dangerous scenario:}
\begin{itemize}
    \item 9V across 10$\Omega$ resistor
    \item Current: $I = 9/10 = 0.9A$
    \item Power: $P = 9 \times 0.9 = 8.1W$
    \item Standard 1/2W (0.5W) resistor rated for only 0.5W
    \item This is 16$\times$ over rating! $\rightarrow$ Immediate damage
\end{itemize}

\vspace{0.15cm}

\textbf{Common Resistor Power Ratings:}

\textbf{Standard values:}
\begin{itemize}
    \item \textbf{1/8W (0.125W):} Very common, small size, cheap
    \item \textbf{1/4W (0.25W):} Most common general purpose
    \item \textbf{1/2W (0.5W):} Common, slightly larger
    \item \textbf{1W:} Moderate power, larger physical size
    \item \textbf{2W:} Higher power, significantly larger
    \item \textbf{5W, 10W, 25W+:} Power resistors, large with heatsinks
\end{itemize}

\textbf{Size correlation:}
\begin{itemize}
    \item Higher rating = physically larger resistor
    \item Larger surface area $\rightarrow$ better heat dissipation
    \item Power resistors may have heatsink mounting
    \item Ceramic body for high-power types
\end{itemize}

\vspace{0.15cm}

\textbf{Selecting Proper Power Rating:}

\textbf{Design rule:}
\begin{itemize}
    \item Calculate actual power dissipation
    \item Select rating at least 2$\times$ calculated power (safety margin)
    \item For critical applications: 50\% derating (use 2W for 1W dissipation)
\end{itemize}

\textit{Why safety margin?}
\begin{itemize}
    \item Component tolerances (resistance may vary $\pm$5\% or $\pm$10\%)
    \item Voltage supply variations
    \item Ambient temperature higher than 25°C
    \item Aging effects
    \item Prevent continuous operation at maximum rating
\end{itemize}

\vspace{0.15cm}

\textbf{Power Resistors:}

\textit{When standard resistors aren't enough:}
\begin{itemize}
    \item Specifically designed for high power dissipation
    \item Ratings: 5W, 10W, 25W, 50W, 100W+
    \item Construction: Ceramic body, wirewound
    \item Often have mounting holes for heatsinks
    \item More expensive than standard resistors
\end{itemize}

\textit{Applications:}
\begin{itemize}
    \item Current sensing (shunt resistors)
    \item Load banks (testing power supplies)
    \item Braking resistors (motors)
    \item Dummy loads
    \item High-power voltage dividers
\end{itemize}

\vspace{0.15cm}

\textbf{Practical Example - LED Current Limiting:}

\textit{Scenario:}
\begin{itemize}
    \item 10mm super-bright red LED
    \item Maximum current: 80mA
    \item Forward voltage: $V_f = 2.2V$
    \item Power supply: 9V battery
    \item Goal: Maximum brightness
\end{itemize}

\textbf{Calculate resistor:}
\begin{align*}
    V_R &= V_{supply} - V_f = 9 - 2.2 = 6.8V \\
    R &= \frac{V_R}{I} = \frac{6.8}{0.08} = 85\Omega \text{ (use 82$\Omega$ standard)}
\end{align*}

\textbf{Calculate power:}
\begin{align*}
    P_R &= V_R \times I = 6.8 \times 0.08 = 0.544W
\end{align*}

\textbf{Resistor selection:}
\begin{itemize}
    \item Calculated power: 0.544W
    \item 1/2W (0.5W) rating: NOT enough (would overheat)
    \item 1W rating: Acceptable, but tight
    \item \textbf{Best choice: 1W or 2W resistor}
\end{itemize}

\textit{Why 1/2W is bad:}
\begin{itemize}
    \item 0.544W > 0.5W rating
    \item Resistor will get very hot
    \item May not immediately fail, but stressed
    \item Shortened lifespan, possible intermittent failure
\end{itemize}

\vspace{0.15cm}

\textbf{Minimizing Power Loss:}

\textit{Design strategies:}
\begin{itemize}
    \item Use switching regulators instead of linear (less heat)
    \item Reduce voltage drop across resistors where possible
    \item Use current sources instead of resistors (for LEDs)
    \item Choose lower resistance values when appropriate
    \item Use PWM for LED brightness control (not resistor)
\end{itemize}

\textit{When power loss is desired:}
\begin{itemize}
    \item Heating elements (intentional heat generation)
    \item Load resistors (testing power supplies)
    \item Discharge resistors (bleeding capacitors)
\end{itemize}

\vspace{0.15cm}

\textbf{Other Components with Power Ratings:}

\textit{Power ratings apply to many components:}

\textbf{Voltage regulators:}
\begin{itemize}
    \item Linear regulators dissipate significant power
    \item $P = (V_{in} - V_{out}) \times I$
    \item Need heatsinks for higher currents
\end{itemize}

\textbf{Diodes:}
\begin{itemize}
    \item Forward voltage drop $\times$ current = power
    \item Rectifier diodes in power supplies
    \item Schottky diodes have lower forward drop (less power)
\end{itemize}

\textbf{MOSFETs and transistors:}
\begin{itemize}
    \item On-state: $P = I^2 R_{DS(on)}$
    \item Switching: Transition losses
    \item High-power applications need thermal management
\end{itemize}

\textbf{Motor drivers:}
\begin{itemize}
    \item H-bridges dissipate power in switching elements
    \item Current sensing resistors
    \item Thermal shutdown protection common
\end{itemize}

\textbf{Amplifiers:}
\begin{itemize}
    \item Power dissipation in output stage
    \item Class A worst (always conducting)
    \item Class D best efficiency (switching)
\end{itemize}
\end{detailbox}

\vspace{0.2cm}

\noindent\textbf{\color{accentcolor} Practical Example \& Numerical}
\begin{examplebox}
\textbf{Example 1: Standard LED Current Limiting}

\textit{Given:}
\begin{itemize}
    \item 5V supply
    \item LED: $V_f = 2V$, $I_f = 20mA$
\end{itemize}

\textbf{Resistor value:}
\begin{equation*}
    R = \frac{5-2}{0.02} = \frac{3}{0.02} = 150\Omega
\end{equation*}

\textbf{Power dissipation:}
\begin{equation*}
    P = 3 \times 0.02 = 0.06W = 60mW
\end{equation*}

\textbf{Rating selection:}
\begin{itemize}
    \item 60mW actual dissipation
    \item 1/8W (125mW) rating: 2$\times$ margin $\checkmark$
    \item \textbf{Use: 1/8W or 1/4W resistor (both OK)}
\end{itemize}

\vspace{0.2cm}

\textbf{Example 2: High-Power LED (From Problem Description)}

\textit{Given:}
\begin{itemize}
    \item 9V battery
    \item 10mm super-bright LED: $V_f = 2.2V$, $I_{max} = 80mA$
\end{itemize}

\textbf{Design:}
\begin{align*}
    V_R &= 9 - 2.2 = 6.8V \\
    R &= \frac{6.8}{0.08} = 85\Omega \text{ $\rightarrow$ use 82$\Omega$ standard}
\end{align*}

\textbf{Actual current with 82$\Omega$:}
\begin{equation*}
    I = \frac{6.8}{82} = 0.0829A = 82.9mA
\end{equation*}

\textbf{Power:}
\begin{equation*}
    P = 6.8 \times 0.083 = 0.564W
\end{equation*}

\textbf{Rating needed:}
\begin{itemize}
    \item 0.564W dissipation
    \item 1/2W (0.5W): Too small! $\times$
    \item 1W rating: Acceptable (77\% of rating)
    \item \textbf{Recommended: 1W resistor} (or 2W for cool operation)
\end{itemize}

\vspace{0.2cm}

\textbf{Example 3: Voltage Divider Power}

\textit{Given:}
\begin{itemize}
    \item 12V input
    \item Voltage divider: $R_1 = R_2 = 10k\Omega$
    \item Output: 6V (half of 12V)
\end{itemize}

\textbf{Current through divider:}
\begin{equation*}
    I = \frac{12}{20k} = 0.6mA
\end{equation*}

\textbf{Power in each resistor:}
\begin{equation*}
    P = I^2 R = (0.0006)^2 \times 10000 = 0.0036W = 3.6mW
\end{equation*}

\textbf{Rating:}
\begin{itemize}
    \item 3.6mW per resistor
    \item 1/8W (125mW) has 35$\times$ margin
    \item \textbf{Standard 1/4W resistors work fine}
\end{itemize}

\vspace{0.2cm}

\textbf{Example 4: When 1/2W Isn't Enough}

\textit{Given:}
\begin{itemize}
    \item 24V supply
    \item Need to drop to 12V at 500mA load
    \item Using series resistor (linear regulation)
\end{itemize}

\textbf{Resistor value:}
\begin{equation*}
    R = \frac{24-12}{0.5} = \frac{12}{0.5} = 24\Omega
\end{equation*}

\textbf{Power dissipation:}
\begin{equation*}
    P = 12 \times 0.5 = 6W
\end{equation*}

\textbf{Rating selection:}
\begin{itemize}
    \item 6W dissipation
    \item Need at least 6W rated resistor
    \item \textbf{Use: 10W or 15W power resistor with heatsink}
\end{itemize}

\textit{Better solution:} Use switching regulator (much more efficient!)

\vspace{0.2cm}

\textbf{Example 5: Multiple Resistors vs Power Resistor}

\textit{Problem:} Need 100$\Omega$ at 2W dissipation

\textbf{Option 1: Single power resistor}
\begin{itemize}
    \item 100$\Omega$, 5W power resistor
    \item Cost: Higher
    \item Size: Large
\end{itemize}

\textbf{Option 2: Four 1/2W resistors in series/parallel}
\begin{itemize}
    \item Four 200$\Omega$, 1/2W resistors in parallel
    \item $R_{eq} = 200/4 = 50\Omega$ $\times$ (wrong value)
\end{itemize}

\textbf{Option 3: Two resistors in series}
\begin{itemize}
    \item Two 50$\Omega$, 1W resistors in series
    \item $R_{eq} = 50 + 50 = 100\Omega$ $\checkmark$
    \item Each dissipates 1W (within rating)
    \item Total: 2W
    \item \textbf{This works!}
\end{itemize}

\vspace{0.2cm}

\textbf{Example 6: Temperature Derating}

\textit{Given:}
\begin{itemize}
    \item 1W resistor rated at 25°C
    \item Operating in 70°C environment
    \item Derating: 2.5mW/°C
\end{itemize}

\textbf{Temperature difference:}
\begin{equation*}
    \Delta T = 70 - 25 = 45°C
\end{equation*}

\textbf{Power reduction:}
\begin{equation*}
    P_{reduce} = 45 \times 2.5mW = 112.5mW
\end{equation*}

\textbf{Derated power rating:}
\begin{equation*}
    P_{derated} = 1000mW - 112.5mW = 887.5mW \approx 0.89W
\end{equation*}

\textit{Safe operating power at 70°C: ~0.9W instead of 1W}
\end{examplebox}

\vspace{0.2cm}

\noindent\textbf{\color{accentcolor} Key Points (Interview Focus)}
\begin{keypointsbox}
\begin{enumerate}
    \item \textbf{Power rating:} Maximum safe power dissipation (usually at 25°C)
    \item \textbf{Common ratings:} 1/8W, 1/4W, 1/2W, 1W, 2W, 5W, 10W+
    \item \textbf{Selection rule:} Use rating $\geq$ 2$\times$ calculated power
    \item \textbf{Exceed rating:} Component overheats, fails ("magic smoke")
    \item \textbf{Size matters:} Higher rating = larger physical size
    \item \textbf{Resistors worst:} 100\% energy $\rightarrow$ heat
    \item \textbf{Power resistors:} Special high-power types (5W+)
    \item \textbf{Derating:} Reduce rating at high ambient temperatures
\end{enumerate}

\textbf{Interview Questions:}
\begin{itemize}
    \item \textbf{Q:} What is a power rating? \\
    \textit{A:} Maximum power component can safely dissipate without damage.
    
    \item \textbf{Q:} Common resistor power ratings? \\
    \textit{A:} 1/8W, 1/4W, 1/2W, 1W, 2W (and 5W+ for power resistors).
    
    \item \textbf{Q:} Resistor dissipates 0.3W. What rating to use? \\
    \textit{A:} 1/2W or 1W (need safety margin above 0.3W).
    
    \item \textbf{Q:} What happens if you exceed power rating? \\
    \textit{A:} Component overheats, may burn/smoke, fails.
    
    \item \textbf{Q:} Why do resistors need power ratings? \\
    \textit{A:} They convert 100\% electrical energy to heat - too much heat damages them.
    
    \item \textbf{Q:} 9V across 10$\Omega$ - can you use 1/2W resistor? \\
    \textit{A:} No! Power = 81/10 = 8.1W, far exceeds 0.5W rating. Need 10W+ power resistor.
\end{itemize}

\textbf{Standard Power Ratings:}
\begin{itemize}
    \item 1/8W = 0.125W (very common, small)
    \item 1/4W = 0.25W (most common general purpose)
    \item 1/2W = 0.5W (common, slightly larger)
    \item 1W = 1.0W (moderate power)
    \item 2W, 5W, 10W+ (power resistors, large)
\end{itemize}

\textbf{Design Guidelines:}
\begin{itemize}
    \item Calculate actual power dissipation
    \item Select rating 2$\times$ actual (minimum)
    \item For critical: 50\% derating (2W for 1W actual)
    \item Account for ambient temperature
    \item Larger is safer (better heat dissipation)
\end{itemize}

\textbf{Applications:}
\begin{itemize}
    \item LED current limiting (1/8W to 1W typically)
    \item Voltage dividers (usually 1/4W sufficient)
    \item Current sensing shunts (power resistors)
    \item Load banks (high-power resistors)
    \item Pull-up/pull-down (1/8W or 1/4W)
\end{itemize}

\textbf{Warning Signs of Overheating:}
\begin{itemize}
    \item Resistor very hot to touch
    \item Discoloration (browning, blackening)
    \item Smoke or burning smell ("magic smoke")
    \item Value drift (resistance changes)
    \item Solder melting around component
    \item PCB discoloration under resistor
\end{itemize}

\textbf{Common Mistakes:}
\begin{itemize}
    \item Using standard resistor for high-power application
    \item Not accounting for voltage drop across resistor
    \item Forgetting safety margin
    \item Ignoring ambient temperature effects
    \item Not considering thermal resistance of PCB
\end{itemize}
\end{keypointsbox}

