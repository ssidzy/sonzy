\section{Section 17: Miscellaneous Topics}

% Topic 1: Relays and RC Differentiator
\subsection{Relays and RC Differentiator}

\vspace{0.2cm}

\noindent\textbf{\color{accentcolor} TL;DR}
\begin{tldrbox}
\textbf{Mechanical relay:} Electrically operated switch using electromagnetic coil to control high current with low current. \textbf{Components:} Coil (electromagnet), armature (movable part), contacts (switch terminals). \textbf{Operation:} Current through coil $\rightarrow$ magnetic field $\rightarrow$ attracts armature $\rightarrow$ closes contacts $\rightarrow$ completes circuit. No current $\rightarrow$ spring returns armature $\rightarrow$ opens contacts. \textbf{Types:} SPST (Single Pole Single Throw, 2 contacts), SPDT (Single Pole Double Throw, 3 contacts + common), DPST (Double Pole Single Throw, 4 contacts), DPDT (Double Pole Double Throw, 6 contacts). \textbf{Key specs:} Rated coil voltage (nominal), set/operate voltage (minimum to close), switching capacity (max current through contacts), power rating (coil consumption in mW).

\textbf{MOSFET relay (Solid-State):} Semiconductor relay using LED, photodiode, and MOSFET instead of mechanical contacts. \textbf{Operation:} Input current $\rightarrow$ LED lights $\rightarrow$ photodiode receives light $\rightarrow$ generates voltage $\rightarrow$ drives MOSFET gate $\rightarrow$ MOSFET switches load. \textbf{Advantages:} No moving parts (silent, long life), compact, no maintenance, faster switching. \textbf{Disadvantages:} Higher cost, voltage drop when ON, leakage when OFF.

\textbf{RC Differentiator:} High-pass filter that outputs rate of change of input signal. \textbf{Circuit:} Capacitor series, resistor to ground, output across R. \textbf{Equation:} $V_{out} = RC \frac{dV_{in}}{dt}$ (derivative of input weighted by RC time constant). \textbf{Square wave input:} Produces positive spike on rising edge, negative spike on falling edge (DC blocked by capacitor). \textbf{Triangle wave input:} Produces square wave (constant slope $\rightarrow$ constant output). \textbf{Design rule:} Small time constant ($\tau = RC \ll T_{pulse}$, typically $\tau < 0.1T$) for sharp spikes. Large $\tau$ ($> 10T$) resembles original square wave.

\textbf{Key equations:} Relay control: Low current coil controls high current load; Differentiator: $V_{out} = RC \frac{dV_{in}}{dt}$; Time constant: $\tau = RC$
\end{tldrbox}

\vspace{0.2cm}

\noindent\textbf{\color{accentcolor} Detailed Explanation}
\begin{detailbox}
\textbf{1. Mechanical Relay Fundamentals:}

Relay is electrically operated switch allowing small current to control much larger current.

\textbf{Relay Analogy:}
\begin{itemize}
    \item Like relay race: Team members pass baton to complete race
    \item Electrical relay: Receives signal from one circuit, passes control to another
    \item Example: TV remote button $\rightarrow$ sends signal to relay $\rightarrow$ relay switches main power ON
    \item Isolation: Input circuit (low voltage control) electrically separated from output (high voltage load)
\end{itemize}

\textbf{Structure Components:}
\begin{itemize}
    \item \textbf{Coil (Electromagnet):} Wire wound around iron core, receives control current
    \item Creates magnetic field when current flows (DC current for steady field)
    \item Strength proportional to current: More current $\rightarrow$ stronger field
    \item \textbf{Armature:} Movable iron piece attracted by electromagnet
    \item Spring-loaded to return to rest position when coil de-energized
    \item \textbf{Contacts:} Metal terminals that close/open under armature movement
    \item Fixed contact (stationary) and moving contact (attached to armature)
    \item \textbf{Return Spring:} Pulls armature back when magnetic force removed
\end{itemize}

\textbf{Operating Principle:}
\begin{itemize}
    \item \textbf{Switch CLOSED (energized):}
    \item Current flows through coil $\rightarrow$ magnetizes iron core
    \item Magnetic force attracts armature to core
    \item Moving contact touches fixed contact $\rightarrow$ relay ON
    \item Load circuit completes $\rightarrow$ lamp/motor/device powered
    \item \textbf{Switch OPEN (de-energized):}
    \item No current through coil $\rightarrow$ no magnetic field
    \item Spring pulls armature back to rest position
    \item Contacts separate $\rightarrow$ relay OFF
    \item Load circuit breaks $\rightarrow$ device unpowered
\end{itemize}

\textbf{Key Characteristic (Isolation):}
\begin{itemize}
    \item Physical spacing between coil and contacts
    \item Input circuit (coil side) electrically isolated from output (contact side)
    \item Allows low-voltage DC to control high-voltage AC safely
    \item Example: 5~V DC Arduino controls 120~V AC lamp via relay
    \item No electrical connection, only magnetic coupling
\end{itemize}

\textbf{2. Relay Types (Contact Configuration):}

Classification based on number of poles (circuits) and throws (positions).

\textbf{SPST (Single Pole Single Throw):}
\begin{itemize}
    \item \textbf{Poles:} 1 circuit controlled
    \item \textbf{Throws:} 1 position (ON or OFF)
    \item \textbf{Terminals:} 2 contacts + 2 for coil = 4 total
    \item Function: Simple ON/OFF switch
    \item Use: Turn single load ON/OFF (lamp, motor)
\end{itemize}

\textbf{SPDT (Single Pole Double Throw):}
\begin{itemize}
    \item \textbf{Poles:} 1 circuit controlled
    \item \textbf{Throws:} 2 positions (common connects to either terminal)
    \item \textbf{Terminals:} 3 contacts (common + 2 positions) + 2 for coil = 5 total
    \item Function: Changeover switch (select between two circuits)
    \item Use: Switch load between two sources, reversing motor direction
    \item Common terminal alternates between two fixed contacts
\end{itemize}

\textbf{DPST (Double Pole Single Throw):}
\begin{itemize}
    \item \textbf{Poles:} 2 circuits controlled simultaneously
    \item \textbf{Throws:} 1 position per pole (both ON or both OFF)
    \item \textbf{Terminals:} 4 contacts (2 pairs) + 2 for coil = 6 total
    \item Function: Two independent SPST switches actuated together
    \item Use: Switch both hot and neutral in AC circuit, control two separate loads together
\end{itemize}

\textbf{DPDT (Double Pole Double Throw):}
\begin{itemize}
    \item \textbf{Poles:} 2 circuits controlled simultaneously
    \item \textbf{Throws:} 2 positions per pole
    \item \textbf{Terminals:} 6 contacts (2 commons + 4 positions) + 2 for coil = 8 total
    \item Function: Two independent SPDT switches actuated together
    \item Use: Reversing polarity, complex switching (motor forward/reverse with two circuits)
\end{itemize}

\textbf{3. Relay Specifications:}

Critical parameters for selecting and using relays.

\textbf{Rated Coil Voltage:}
\begin{itemize}
    \item Nominal voltage relay coil designed for
    \item Example: 5~V, 12~V, 24~V DC relays common
    \item Operating coil at rated voltage ensures proper operation
    \item Too low voltage $\rightarrow$ may not close (weak magnetic field)
    \item Too high voltage $\rightarrow$ overheating, coil damage
\end{itemize}

\textbf{Set/Operate Voltage:}
\begin{itemize}
    \item Minimum voltage needed to close contacts reliably
    \item Typically 70--80\% of rated voltage
    \item Example: 12~V relay may close at 9~V (75\%)
    \item Below operate voltage: Contacts may chatter or not close
\end{itemize}

\textbf{Release Voltage:}
\begin{itemize}
    \item Voltage below which contacts open (armature releases)
    \item Hysteresis: Operate voltage > Release voltage
    \item Prevents chattering when voltage fluctuates
\end{itemize}

\textbf{Coil Power Rating:}
\begin{itemize}
    \item Power consumed by coil when energized (mW or W)
    \item Example: 12~V relay, 50~mA coil $\rightarrow$ $P = 12 \times 0.05 = 0.6$~W
    \item Important for calculating driver circuit power dissipation
    \item Low-power relays: 100--500~mW typical
\end{itemize}

\textbf{Switching Capacity (Contact Rating):}
\begin{itemize}
    \item Maximum current contacts can switch safely
    \item Specified for resistive, inductive, capacitive loads separately
    \item Example: 10~A at 250~V AC (resistive), 3~A at 30~V DC (inductive)
    \item \textbf{Inductive loads:} Motors take surge current at startup (2--5$\times$ running current)
    \item Design rule: Relay rating $\geq$ 2$\times$ motor running current for safety
    \item Contact material: Silver alloy (low resistance, arc-resistant)
\end{itemize}

\textbf{4. Relay Applications:}

Using small current to control large loads.

\textbf{Example 1: DC Control of AC Load:}
\begin{itemize}
    \item Control circuit: 5~V DC (microcontroller, Arduino)
    \item Load circuit: 120~V AC lamp
    \item Relay coil: 5~V DC, 70~mA (350~mW)
    \item Contacts: 10~A at 250~V AC rating
    \item Operation: MCU output HIGH (5~V) $\rightarrow$ energizes coil $\rightarrow$ contacts close $\rightarrow$ lamp ON
    \item MCU output LOW (0~V) $\rightarrow$ coil de-energized $\rightarrow$ contacts open $\rightarrow$ lamp OFF
    \item Safety: Complete electrical isolation between DC and AC circuits
\end{itemize}

\textbf{Example 2: Low Current Controlling High Current:}
\begin{itemize}
    \item Control: 50~mA through coil (low current, safe for electronics)
    \item Load: 5~A motor (100$\times$ coil current)
    \item Power amplification: Small control power switches large load power
    \item Without relay: Would need transistor rated for full 5~A (expensive, complex)
\end{itemize}

\textbf{Example 3: Multiple Independent Circuits:}
\begin{itemize}
    \item One coil can control multiple contact sets (DPDT, multi-pole relays)
    \item Single input signal switches several independent loads simultaneously
    \item Example: Turn ON heater + fan + indicator lamp together
\end{itemize}

\textbf{5. MOSFET Relay (Solid-State Relay):}

Semiconductor-based relay with no moving parts.

\textbf{Structure:}
\begin{itemize}
    \item \textbf{Input:} LED (light-emitting diode)
    \item \textbf{Coupling:} Optical isolation (light transmission)
    \item \textbf{Detector:} Photodiode array
    \item \textbf{Output:} Power MOSFET (switches load)
\end{itemize}

\textbf{Operating Principle:}
\begin{itemize}
    \item Input current applied $\rightarrow$ LED lights up
    \item LED emits light (infrared or visible)
    \item Photodiode receives light $\rightarrow$ generates voltage (photovoltaic effect)
    \item Photodiode voltage drives MOSFET gate
    \item MOSFET turns ON $\rightarrow$ conducts between drain-source $\rightarrow$ load powered
    \item No input current $\rightarrow$ LED OFF $\rightarrow$ photodiode generates no voltage $\rightarrow$ MOSFET OFF
\end{itemize}

\textbf{Advantages of MOSFET Relay:}
\begin{itemize}
    \item \textbf{No mechanical parts:} No wear, unlimited switching cycles (vs 100k--1M for mechanical)
    \item \textbf{Silent operation:} No audible click (mechanical relays make noise)
    \item \textbf{Fast switching:} µs switching time (vs ms for mechanical)
    \item \textbf{Compact:} Smaller footprint (no coil, armature, spring)
    \item \textbf{No maintenance:} Contacts don't wear, corrode, or arc
    \item \textbf{Long life:} 10--100$\times$ longer than mechanical (no contact degradation)
\end{itemize}

\textbf{Disadvantages:}
\begin{itemize}
    \item \textbf{Higher cost:} More expensive than equivalent mechanical relay
    \item \textbf{Voltage drop:} MOSFET has $R_{DS(on)}$ (10--100~m$\Omega$), small voltage drop when ON
    \item Mechanical relay: Contact resistance ~10--50~m$\Omega$, similar but better
    \item \textbf{Leakage current:} Small OFF-state current (µA--mA) through MOSFET
    \item Mechanical relay: Zero current when OFF (true open circuit)
    \item \textbf{Heat dissipation:} Power MOSFET generates heat ($P = I^2 R_{DS(on)}$)
\end{itemize}

\textbf{6. RC Differentiator (High-Pass as Rate-of-Change Detector):}

RC high-pass filter outputs derivative of input signal (rate of change).

\textbf{Circuit Configuration:}
\begin{itemize}
    \item Capacitor in series (input to capacitor)
    \item Resistor to ground (capacitor to resistor junction)
    \item Output taken across resistor (not capacitor)
    \item High-pass filter becomes differentiator for non-sinusoidal inputs
\end{itemize}

\textbf{Why Called Differentiator:}
\begin{itemize}
    \item Effect similar to mathematical differentiation
    \item Differentiation: Finding rate of change of quantity
    \item Output proportional to $\frac{dV_{in}}{dt}$ (how fast input changes)
    \item Rapid input change $\rightarrow$ large output voltage
    \item Slow input change $\rightarrow$ small output voltage
    \item No change (DC) $\rightarrow$ zero output (capacitor blocks DC)
\end{itemize}

\textbf{Differentiator Equation Derivation:}
\begin{itemize}
    \item Capacitor current: $I_C = C \frac{dV_C}{dt}$ (current proportional to voltage rate of change)
    \item Voltage across capacitor cannot change instantly (needs time to charge)
    \item Current through C must equal current through R (series circuit): $I_C = I_R$
    \item Voltage across R: $V_R = I_R \times R = I_C \times R$
    \item Substitute: $V_{out} = V_R = RC \frac{dV_{in}}{dt}$
    \item Output is derivative of input weighted by time constant $\tau = RC$
\end{itemize}

\textbf{Square Wave Input Response:}
\begin{itemize}
    \item \textbf{Rising edge:} Input voltage jumps positive instantly
    \item Rate of change very high ($\frac{dV}{dt} \to \infty$ ideally)
    \item Output: Positive spike (narrow pulse)
    \item \textbf{Falling edge:} Input voltage jumps negative instantly
    \item Output: Negative spike (narrow pulse, opposite polarity)
    \item \textbf{Flat top:} Input constant (no change)
    \item $\frac{dV}{dt} = 0$ $\rightarrow$ output zero (capacitor fully charged, no current)
    \item Result: Positive + negative spikes at edges only
    \item DC level removed (capacitor blocks DC component)
\end{itemize}

\textbf{Triangle Wave Input Response:}
\begin{itemize}
    \item \textbf{Rising slope:} Constant positive rate of change
    \item $\frac{dV}{dt} = +k$ (constant) $\rightarrow$ output constant positive voltage
    \item \textbf{Falling slope:} Constant negative rate of change
    \item $\frac{dV}{dt} = -k$ (constant) $\rightarrow$ output constant negative voltage
    \item Result: Square wave output (derivative of triangle is square)
    \item Perfect mathematical relationship: Triangle slope $\rightarrow$ square level
\end{itemize}

\textbf{Time Constant Effect:}

Shape of output depends on ratio of pulse width $T$ to time constant $\tau = RC$.

\begin{itemize}
    \item \textbf{Large $\tau$ ($\tau > 10T$):} Output resembles input square wave
    \item Capacitor charges/discharges slowly, follows input
    \item High-pass characteristic weak, passes low frequencies
    \item \textbf{Small $\tau$ ($\tau < 0.1T$):} Output is sharp spikes
    \item Capacitor charges/discharges quickly, only edges create voltage
    \item True differentiation: Only fast changes produce output
    \item \textbf{Intermediate $\tau$:} Range of waveforms between spikes and square
    \item Design rule: Use $\tau \ll T$ for good differentiation (sharp pulses)
\end{itemize}

\textbf{Applications:}
\begin{itemize}
    \item Edge detection: Trigger on rising/falling edges of digital signals
    \item Wave shaping: Convert square to spikes for timing circuits
    \item High-pass coupling: Remove DC, pass AC changes
    \item Pulse generation: Create narrow trigger pulses from slow transitions
\end{itemize}
\end{detailbox}

\vspace{0.2cm}

\noindent\textbf{\color{accentcolor} Practical Examples \& Numerical}
\begin{examplebox}
\textbf{Example 1: Relay Coil Power Calculation}

\textbf{Given:} 12~V DC relay, coil resistance 240~$\Omega$

\textbf{Calculate coil current and power:}
\begin{itemize}
    \item Coil current: $I = \frac{V}{R} = \frac{12}{240} = 0.05$~A = 50~mA
    \item Coil power: $P = VI = 12 \times 0.05 = 0.6$~W = 600~mW
    \item Or: $P = \frac{V^2}{R} = \frac{144}{240} = 0.6$~W $\checkmark$
\end{itemize}

\textbf{Driver circuit considerations:}
\begin{itemize}
    \item Need transistor/driver capable of sinking 50~mA
    \item 2N2222 transistor: Max 600~mA collector current $\checkmark$ (12$\times$ safety margin)
    \item Power dissipation in transistor: $P = V_{CE(sat)} \times I_C \approx 0.3 \times 0.05 = 15$~mW (negligible)
    \item Flyback diode required across coil (inductive load, voltage spike when turned OFF)
\end{itemize}

\vspace{0.15cm}

\textbf{Example 2: Motor Relay Selection}

\textbf{Requirement:} Control DC motor, running current 2~A, startup surge 8~A

\textbf{Relay selection criteria:}
\begin{itemize}
    \item Running current: 2~A
    \item Surge current: 8~A (4$\times$ running, typical for motors)
    \item Design rule: Relay rating $\geq$ 2$\times$ surge current for safety
    \item Minimum relay rating: $2 \times 8 = 16$~A
    \item Choose: 20~A relay (next standard size, provides margin)
\end{itemize}

\textbf{Why oversizing needed:}
\begin{itemize}
    \item Motor startup: High inrush current (low back-EMF initially)
    \item Contact welding risk: High current creates arc when switching
    \item Contact wear: Arcing erodes contact material over time
    \item Derating for reliability: 2$\times$ factor ensures long relay life
    \item Alternative: Use soft-start circuit to limit inrush (allows smaller relay)
\end{itemize}

\vspace{0.15cm}

\textbf{Example 3: RC Differentiator Design}

\textbf{Requirement:} Detect edges of 1~kHz square wave (period $T = 1$~ms)

\textbf{Design for sharp spikes:}
\begin{itemize}
    \item Rule: $\tau = RC < 0.1T$ for good differentiation
    \item Target: $\tau = 0.05T = 0.05 \times 1 \times 10^{-3} = 50$~µs
    \item Choose C = 1~nF (small, typical)
    \item Calculate R: $R = \frac{\tau}{C} = \frac{50 \times 10^{-6}}{1 \times 10^{-9}} = 50$~k$\Omega$
    \item Use standard value: R = 47~k$\Omega$ (gives $\tau = 47$~µs, close enough)
\end{itemize}

\textbf{Expected output:}
\begin{itemize}
    \item Input: Square wave, 0--5~V, 1~kHz
    \item Rising edge (0$\rightarrow$5~V): Positive spike, amplitude ~5~V, width ~5$\tau$ = 235~µs
    \item Falling edge (5$\rightarrow$0~V): Negative spike, amplitude ~$-5$~V, width ~235~µs
    \item Flat portions: Output decays to zero (capacitor charges/discharges)
    \item Spike much narrower than pulse (235~µs vs 500~µs = 47\% of half-period)
\end{itemize}

\textbf{Verification:}
\begin{itemize}
    \item Ratio: $\frac{\tau}{T} = \frac{47}{1000} = 0.047$ $\checkmark$ (less than 0.1 threshold)
    \item Good differentiation achieved
    \item For sharper spikes: Reduce $\tau$ further (use R = 10~k$\Omega$ $\rightarrow$ $\tau = 10$~µs)
\end{itemize}

\vspace{0.15cm}

\textbf{Example 4: Differentiator Output Voltage}

\textbf{Given:} Triangle wave, 0--10~V peak-to-peak, 100~Hz, RC = 1~ms

\textbf{Calculate output:}
\begin{itemize}
    \item Period: $T = \frac{1}{100} = 10$~ms
    \item Rising slope duration: $\frac{T}{2} = 5$~ms (0$\rightarrow$10~V in 5~ms)
    \item Slope: $\frac{dV}{dt} = \frac{10 - 0}{5 \times 10^{-3}} = 2000$~V/s = 2~V/ms
    \item Output voltage: $V_{out} = RC \frac{dV_{in}}{dt} = 1 \times 10^{-3} \times 2000 = 2$~V
\end{itemize}

\textbf{Output waveform:}
\begin{itemize}
    \item Rising slope (0$\rightarrow$10~V): Output = +2~V (constant)
    \item Falling slope (10$\rightarrow$0~V): Slope = $-2000$~V/s $\rightarrow$ Output = $-2$~V (constant)
    \item Result: Square wave, $\pm$2~V peak, 100~Hz
    \item Derivative of triangle wave is square wave $\checkmark$
\end{itemize}
\end{examplebox}

\vspace{0.2cm}

\noindent\textbf{\color{accentcolor} Key Points (Interview Focus)}
\begin{keypointsbox}
\begin{itemize}
    \item \textbf{Relay Purpose:} Small current controls large current via electromagnetic coupling. Coil (mA--100~mA) creates magnetic field, attracts armature, closes contacts switching load (A--100~A). Complete electrical isolation between control and load circuits. Example: 5~V 50~mA coil switches 120~V 10~A lamp (2400~W load controlled by 0.25~W input)
    
    \item \textbf{Relay Types:} SPST (1 circuit ON/OFF, 4 terminals). SPDT (1 circuit, 2 positions + common, 5 terminals). DPST (2 circuits ON/OFF together, 6 terminals). DPDT (2 circuits, 2 positions each, 8 terminals). Pole = number of independent circuits. Throw = number of positions per pole
    
    \item \textbf{Relay Specifications:} Rated coil voltage (nominal operation voltage). Set voltage (minimum to close, ~75\% rated). Switching capacity (max contact current, derate for inductive loads). Coil power (mW consumption). For motors: Relay rating $\geq$ 2$\times$ surge current (startup inrush 4--5$\times$ running)
    
    \item \textbf{Mechanical vs MOSFET Relay:} Mechanical: Moving contacts, audible click, 100k--1M cycles, true isolation when OFF, cheap. MOSFET (SSR): Optical coupling + semiconductor switch, silent, unlimited cycles, fast (µs), compact, expensive, small voltage drop and leakage. Use mechanical for cost/isolation, SSR for life/speed
    
    \item \textbf{RC Differentiator Principle:} Output = $RC \frac{dV_{in}}{dt}$ (derivative of input). Rapid change $\rightarrow$ large output, slow change $\rightarrow$ small output, DC $\rightarrow$ zero. Square wave input produces spikes at edges (positive on rising, negative on falling). Triangle wave input produces square wave (constant slope $\rightarrow$ constant output). Mathematically correct differentiation
    
    \item \textbf{Time Constant Effect:} $\tau = RC$ determines differentiation quality. Small $\tau$ ($<0.1T$) $\rightarrow$ sharp spikes, good differentiation. Large $\tau$ ($>10T$) $\rightarrow$ output resembles input, poor differentiation. Intermediate $\tau$ $\rightarrow$ partial differentiation. Design: Choose $\tau \ll T_{pulse}$ for edge detection, spike generation
    
    \item \textbf{Q: Why relay needs flyback diode?} A: Coil is inductor, stores energy in magnetic field. When turned OFF, current cannot stop instantly. Collapsing magnetic field induces large voltage spike ($V = -L \frac{dI}{dt}$, can reach 100s of volts). Damages transistor driver. Flyback diode (reverse across coil) provides path for induced current, clamps voltage to ~0.7~V, protects circuit. Always required for inductive relay coils
    
    \item \textbf{Q: How does RC differentiator work with square wave?} A: Rising edge: Input jumps 0$\rightarrow$5~V instantly, $\frac{dV}{dt}$ very high, capacitor cannot charge instantly, voltage difference appears across R, creates positive spike. Flat top: Input constant, $\frac{dV}{dt}=0$, capacitor charges to input voltage through R, current stops, output decays to zero. Falling edge: Input drops 5$\rightarrow$0~V, capacitor voltage higher than input, discharges through R in reverse, creates negative spike. Capacitor blocks DC, only passes changes
    
    \item \textbf{Q: Why MOSFET relay better for high-frequency switching?} A: Mechanical relay has mass (armature, contacts), inertia limits speed to ~1--10~ms switching time. Contacts bounce (multiple make/break cycles) during closure, ~1--5~ms settle. Max switching rate ~10--100~Hz. MOSFET relay: No moving parts, purely electronic, switching time 10--100~µs, no bounce, can switch at kHz--MHz rates. Used in PWM, high-speed control, data transmission
\end{itemize}
\end{keypointsbox}

\newpage

% Topic 2: Wheatstone Bridge and Series RLC Circuits
\subsection{Wheatstone Bridge and Series RLC Circuits}

\vspace{0.2cm}

\noindent\textbf{\color{accentcolor} TL;DR}
\begin{tldrbox}
\textbf{Wheatstone Bridge:} Circuit for measuring unknown resistance using three known resistors. \textbf{Structure:} Two voltage dividers in parallel, voltmeter between midpoints. \textbf{Balance condition:} $\frac{R_1}{R_2} = \frac{R_3}{R_4}$ when voltmeter reads 0~V (no current through meter). \textbf{To measure $R_2$:} Set $R_1$, $R_3$, adjust $R_4$ until balanced, then $R_2 = R_1 \frac{R_4}{R_3}$. \textbf{Applications:} Precision resistance measurement (down to m$\Omega$ range), sensor interfacing (strain gauges, RTDs, photocells with op-amps), detector circuits.

\textbf{Series RLC circuit:} R, L, C in series across AC supply. \textbf{Impedance:} $Z = \sqrt{R^2 + (X_L - X_C)^2}$ where $X_L = 2\pi fL$ and $X_C = \frac{1}{2\pi fC}$. \textbf{Current:} $I = \frac{V_s}{Z}$. \textbf{Resonance:} When $X_L = X_C$ (reactances cancel), $f_0 = \frac{1}{2\pi\sqrt{LC}}$, impedance minimum = R, current maximum. \textbf{Phasor diagram:} $V_L$ leads current by 90°, $V_C$ lags by 90° (opposite directions), $V_R$ in phase. Supply voltage: $V_s = \sqrt{V_R^2 + (V_L - V_C)^2}$ (vector sum).

\textbf{Oscillation vs damping:} When DC power applied then removed, LC oscillates at $f_0$ (energy alternates between magnetic and electric fields). \textbf{Damping:} Resistance dissipates energy, oscillations decay. Small R $\rightarrow$ underdamped (many oscillations before stopping). Large R $\rightarrow$ critically damped (fastest return to zero without overshoot). Very large R $\rightarrow$ overdamped (slow return).

\textbf{Key equations:} Bridge balance: $\frac{R_1}{R_2} = \frac{R_3}{R_4}$; RLC impedance: $Z = \sqrt{R^2 + (X_L - X_C)^2}$; Resonance: $f_0 = \frac{1}{2\pi\sqrt{LC}}$
\end{tldrbox}

\vspace{0.2cm}

\noindent\textbf{\color{accentcolor} Detailed Explanation}
\begin{detailbox}
\textbf{1. Wheatstone Bridge Fundamentals:}

Precision circuit for measuring resistance by comparing with known values.

\textbf{Historical Context:}
\begin{itemize}
    \item Developed by Charles Wheatstone for unknown resistance measurement
    \item Calibration tool for ohmmeters, voltmeters, ammeters
    \item Digital multimeters simpler today, but bridge still used for:
    \item Very low resistance (m$\Omega$ range, contact resistance)
    \item Sensor interfacing (strain gauges, temperature sensors)
    \item High-precision measurements (4-wire sensing)
\end{itemize}

\textbf{Circuit Structure:}
\begin{itemize}
    \item \textbf{Configuration:} Two voltage dividers in parallel
    \item First divider: $R_1$ series with $R_2$ (top to bottom)
    \item Second divider: $R_3$ series with $R_4$ (top to bottom)
    \item Voltage supply across both dividers (common top and bottom)
    \item Voltmeter connected between divider midpoints (junction $R_1$-$R_2$ and $R_3$-$R_4$)
    \item \textbf{Bridge:} Voltmeter connection is the "bridge" between two dividers
\end{itemize}

\textbf{Balance Condition:}

Bridge is balanced when voltmeter reads exactly 0~V (no potential difference).

\begin{itemize}
    \item \textbf{Left divider voltage:} $V_1 = V_s \frac{R_2}{R_1 + R_2}$ (voltage at $R_1$-$R_2$ junction)
    \item \textbf{Right divider voltage:} $V_2 = V_s \frac{R_4}{R_3 + R_4}$ (voltage at $R_3$-$R_4$ junction)
    \item \textbf{Balance:} $V_1 = V_2$ (no voltage difference, meter reads 0)
    \item $V_s \frac{R_2}{R_1 + R_2} = V_s \frac{R_4}{R_3 + R_4}$
    \item Cancel $V_s$: $\frac{R_2}{R_1 + R_2} = \frac{R_4}{R_3 + R_4}$
    \item Cross-multiply: $R_2(R_3 + R_4) = R_4(R_1 + R_2)$
    \item Expand: $R_2 R_3 + R_2 R_4 = R_4 R_1 + R_4 R_2$
    \item Cancel $R_2 R_4$: $R_2 R_3 = R_4 R_1$
    \item \textbf{Balance equation:} $\frac{R_1}{R_2} = \frac{R_3}{R_4}$ (ratio equality)
\end{itemize}

\textbf{Measuring Unknown Resistance:}

Assume $R_2$ is unknown, $R_1$, $R_3$ fixed, $R_4$ variable.

\begin{itemize}
    \item Set $R_1$ and $R_3$ to known values (e.g., 220~$\Omega$, 400~$\Omega$)
    \item Adjust $R_4$ (variable resistor, potentiometer) until voltmeter reads 0~V
    \item At balance: $\frac{R_1}{R_2} = \frac{R_3}{R_4}$
    \item Solve for unknown: $R_2 = R_1 \frac{R_4}{R_3}$
    \item Plug in known values, calculate $R_2$
    \item Example: $R_1 = 220$~$\Omega$, $R_3 = 400$~$\Omega$, $R_4 = 1000$~$\Omega$ (adjusted to balance)
    \item $R_2 = 220 \times \frac{1000}{400} = 220 \times 2.5 = 550$~$\Omega$ $\checkmark$
\end{itemize}

\textbf{Supply Voltage Independence:}

Balance condition independent of supply voltage.

\begin{itemize}
    \item Balance depends only on resistance ratios, not absolute voltages
    \item If $V_s$ doubled: Both $V_1$ and $V_2$ double equally
    \item Difference $(V_1 - V_2)$ remains zero if balanced
    \item Can use any stable supply (battery, regulated supply)
    \item Advantage: Voltage drift doesn't affect measurement accuracy
\end{itemize}

\textbf{2. Wheatstone Bridge Applications:}

Beyond simple resistance measurement.

\textbf{Sensor Interfacing (with Op-Amp):}
\begin{itemize}
    \item Replace one resistor with sensor (strain gauge, RTD, LDR photocell)
    \item Sensor resistance changes with physical quantity (force, temperature, light)
    \item Bridge output voltage proportional to resistance change
    \item Op-amp amplifies small voltage difference (high gain, high input impedance)
    \item \textbf{Example:} Light detector using LDR (Light-Dependent Resistor)
    \item Dark: LDR high resistance $\rightarrow$ bridge unbalanced one way $\rightarrow$ positive output
    \item Bright: LDR low resistance $\rightarrow$ bridge unbalanced opposite $\rightarrow$ negative output
    \item Op-amp output drives indicator, relay, or ADC
\end{itemize}

\textbf{Strain Gauge Measurement:}
\begin{itemize}
    \item Strain gauge: Resistance changes with mechanical strain ($\Delta R / R \approx 0.1$--2\%)
    \item Four-arm bridge: All four resistors are strain gauges (temperature compensation)
    \item Balanced at zero strain, unbalanced proportional to applied force
    \item Used in: Load cells, pressure sensors, torque sensors
\end{itemize}

\textbf{3. Series RLC Circuit Basics:}

R, L, C connected in series across AC supply, impedance depends on frequency.

\textbf{Component Reactances:}
\begin{itemize}
    \item \textbf{Resistance R:} Constant with frequency, dissipates power
    \item \textbf{Inductive reactance:} $X_L = 2\pi fL = \omega L$ (increases with frequency)
    \item At DC ($f=0$): $X_L = 0$ (inductor acts as short circuit)
    \item At high f: $X_L$ large (inductor blocks AC)
    \item \textbf{Capacitive reactance:} $X_C = \frac{1}{2\pi fC} = \frac{1}{\omega C}$ (decreases with frequency)
    \item At DC ($f=0$): $X_C = \infty$ (capacitor acts as open circuit)
    \item At high f: $X_C$ small (capacitor passes AC easily)
\end{itemize}

\textbf{Angular Frequency:}
\begin{itemize}
    \item $\omega = 2\pi f$ (radians per second)
    \item One cycle = $2\pi$ radians
    \item Frequency $f$ = cycles per second (Hz)
    \item Angular frequency = radians per second (rad/s)
    \item Simplifies equations: $X_L = \omega L$, $X_C = \frac{1}{\omega C}$
\end{itemize}

\textbf{4. Series RLC Impedance and Phasor Diagram:}

Total impedance combines resistance and reactance (vector addition).

\textbf{Voltage-Current Phase Relationships:}
\begin{itemize}
    \item \textbf{Resistor:} Voltage and current in phase (0° phase shift)
    \item \textbf{Inductor:} Voltage leads current by 90° (current lags voltage)
    \item Memory: "ELI" $\rightarrow$ E (voltage) Leads I (current) in Inductor
    \item \textbf{Capacitor:} Voltage lags current by 90° (current leads voltage)
    \item Memory: "ICE" $\rightarrow$ I (current) Comes before E (voltage) in Capacitor
\end{itemize}

\textbf{Phasor Diagram (Current as Reference):}
\begin{itemize}
    \item Current $I$ same through all components (series circuit) $\rightarrow$ use as reference (horizontal)
    \item $V_R = IR$ in phase with current (horizontal, same direction as $I$)
    \item $V_L = IX_L$ leads current by 90° (vertical, upward from $I$)
    \item $V_C = IX_C$ lags current by 90° (vertical, downward from $I$)
    \item $V_L$ and $V_C$ point opposite directions (cancel partially or fully)
\end{itemize}

\textbf{Supply Voltage (Vector Sum):}
\begin{itemize}
    \item Cannot simply add $V_R + V_L + V_C$ (different phases)
    \item Must use vector (phasor) addition
    \item $V_L$ and $V_C$ opposite: Net reactive voltage = $V_L - V_C$
    \item Supply voltage: $V_s = \sqrt{V_R^2 + (V_L - V_C)^2}$ (Pythagorean theorem)
    \item Substitute $V = IZ$: $V_s = I\sqrt{R^2 + (X_L - X_C)^2}$
    \item \textbf{Impedance:} $Z = \frac{V_s}{I} = \sqrt{R^2 + (X_L - X_C)^2}$ (total opposition to AC)
\end{itemize}

\textbf{Three Impedance Cases:}
\begin{itemize}
    \item \textbf{$X_L > X_C$:} Inductive dominant, $Z = \sqrt{R^2 + (X_L - X_C)^2}$, current lags voltage
    \item \textbf{$X_C > X_L$:} Capacitive dominant, $Z = \sqrt{R^2 + (X_C - X_L)^2}$, current leads voltage
    \item \textbf{$X_L = X_C$:} Resonance, reactances cancel, $Z = R$ (minimum), current maximum
\end{itemize}

\textbf{5. Resonance in Series RLC:}

Special frequency where reactances cancel, impedance minimized.

\textbf{Resonance Condition:}
\begin{itemize}
    \item $X_L = X_C$ (inductive and capacitive reactances equal)
    \item $2\pi f_0 L = \frac{1}{2\pi f_0 C}$
    \item Rearrange: $(2\pi f_0)^2 LC = 1$
    \item \textbf{Resonant frequency:} $f_0 = \frac{1}{2\pi\sqrt{LC}}$
    \item Depends only on L and C values, independent of R
\end{itemize}

\textbf{At Resonance:}
\begin{itemize}
    \item $X_L - X_C = 0$ (reactances cancel completely)
    \item Impedance: $Z = \sqrt{R^2 + 0^2} = R$ (minimum possible, purely resistive)
    \item Current: $I = \frac{V_s}{Z} = \frac{V_s}{R}$ (maximum current for given $V_s$)
    \item Voltage across R: $V_R = IR = V_s$ (all supply voltage across R)
    \item Voltage across L and C: Can be very large (Q times $V_s$), but cancel in phasor sum
    \item Power factor: 1 (purely resistive, voltage and current in phase)
\end{itemize}

\textbf{Applications:}
\begin{itemize}
    \item Radio tuning: Select frequency $f_0$ by adjusting L or C (variable capacitor)
    \item Impedance minimum at $f_0$ $\rightarrow$ maximum signal at resonance
    \item Filter: Passes $f_0$, attenuates other frequencies
    \item FM tuner: Vary C to change $f_0$ across 88--108~MHz band
\end{itemize}

\textbf{6. RLC Oscillation and Damping:}

Energy exchange between L and C when DC power applied then removed.

\textbf{Oscillation Mechanism:}
\begin{itemize}
    \item Initially: Capacitor charged, inductor de-energized
    \item Capacitor discharges through inductor
    \item Current builds magnetic field in inductor (stores energy)
    \item Capacitor fully discharged, inductor fully energized
    \item Magnetic field collapses, induces current (Lenz's law)
    \item Current charges capacitor in opposite polarity
    \item Process reverses: Energy oscillates L $\leftrightarrow$ C
    \item Frequency: $f_0 = \frac{1}{2\pi\sqrt{LC}}$ (resonant frequency)
\end{itemize}

\textbf{Ideal LC (No Resistance):}
\begin{itemize}
    \item Energy conserved: $E_{total} = E_L + E_C$ constant
    \item Magnetic energy: $E_L = \frac{1}{2}LI^2$
    \item Electric energy: $E_C = \frac{1}{2}CV^2$
    \item Oscillation continues forever (no damping)
    \item Harmonic oscillator (like frictionless pendulum)
\end{itemize}

\textbf{Real RLC (With Resistance):}

Resistance dissipates energy, oscillations decay.

\begin{itemize}
    \item Power loss: $P = I^2 R$ (resistance converts energy to heat)
    \item Each cycle: Some energy lost, amplitude decreases
    \item \textbf{Damping:} Gradual reduction in oscillation amplitude
    \item Eventually: All energy dissipated, oscillation stops
\end{itemize}

\textbf{Damping Regimes:}

Behavior depends on resistance value relative to critical damping.

\begin{itemize}
    \item \textbf{Underdamped ($R < R_{critical}$):} Small resistance
    \item Many oscillations before stopping (10--100+ cycles)
    \item Exponential envelope: Amplitude decays as $e^{-t/\tau}$ where $\tau = \frac{2L}{R}$
    \item Typical: Small R (10--100~$\Omega$)
    \item \textbf{Critically damped ($R = R_{critical}$):} Specific resistance value
    \item $R_{critical} = 2\sqrt{\frac{L}{C}}$ (critical damping resistance)
    \item No oscillation, fastest return to zero without overshoot
    \item Optimal for control systems (door closer, suspension)
    \item \textbf{Overdamped ($R > R_{critical}$):} Large resistance
    \item No oscillation, slow exponential decay
    \item Takes longer to settle than critically damped
    \item Example: Very large R (k$\Omega$--M$\Omega$)
\end{itemize}
\end{detailbox}

\vspace{0.2cm}

\noindent\textbf{\color{accentcolor} Practical Examples \& Numerical}
\begin{examplebox}
\textbf{Example 1: Wheatstone Bridge Measurement}

\textbf{Given:} Bridge balanced with $R_1 = 220$~$\Omega$, $R_3 = 400$~$\Omega$, $R_4 = 1000$~$\Omega$

\textbf{Find unknown $R_2$:}
\begin{itemize}
    \item Balance condition: $\frac{R_1}{R_2} = \frac{R_3}{R_4}$
    \item Rearrange: $R_2 = R_1 \frac{R_4}{R_3}$
    \item Substitute: $R_2 = 220 \times \frac{1000}{400} = 220 \times 2.5$
    \item $R_2 = 550$~$\Omega$ $\checkmark$
\end{itemize}

\textbf{Verification:}
\begin{itemize}
    \item Check ratio: $\frac{R_1}{R_2} = \frac{220}{550} = 0.4$
    \item $\frac{R_3}{R_4} = \frac{400}{1000} = 0.4$ $\checkmark$ (ratios equal)
    \item Bridge balanced $\checkmark$
\end{itemize}

\vspace{0.15cm}

\textbf{Example 2: Series RLC Impedance Calculation}

\textbf{Given:} R = 2~$\Omega$, L = 0.15~H, C = 100~µF, supply 100~V at 50~Hz

\textbf{Calculate impedance and current:}
\begin{itemize}
    \item Inductive reactance: $X_L = 2\pi fL = 2\pi \times 50 \times 0.15$
    \item $X_L = 47.12$~$\Omega$
    \item Capacitive reactance: $X_C = \frac{1}{2\pi fC} = \frac{1}{2\pi \times 50 \times 100 \times 10^{-6}}$
    \item $X_C = \frac{1}{0.0314} = 31.83$~$\Omega$
    \item Net reactance: $X_L - X_C = 47.12 - 31.83 = 15.29$~$\Omega$ (inductive)
    \item Impedance: $Z = \sqrt{R^2 + (X_L - X_C)^2} = \sqrt{4 + 233.78}$
    \item $Z = \sqrt{237.78} = 15.42$~$\Omega$
    \item Current: $I = \frac{V_s}{Z} = \frac{100}{15.42} = 6.49$~A
\end{itemize}

\textbf{Voltage drops:}
\begin{itemize}
    \item $V_R = IR = 6.49 \times 2 = 12.98$~V
    \item $V_L = IX_L = 6.49 \times 47.12 = 305.8$~V
    \item $V_C = IX_C = 6.49 \times 31.83 = 206.6$~V
    \item Verify: $V_s = \sqrt{V_R^2 + (V_L - V_C)^2} = \sqrt{168.5 + (99.2)^2} = \sqrt{10007.1} = 100$~V $\checkmark$
    \item Note: $V_L$ and $V_C$ individually exceed $V_s$ (resonance effect), but cancel in phasor sum
\end{itemize}

\vspace{0.15cm}

\textbf{Example 3: Resonant Frequency Calculation}

\textbf{Given:} L = 0.15~H, C = 100~µF (same circuit as Example 2)

\textbf{Find resonant frequency:}
\begin{itemize}
    \item $f_0 = \frac{1}{2\pi\sqrt{LC}} = \frac{1}{2\pi\sqrt{0.15 \times 100 \times 10^{-6}}}$
    \item $f_0 = \frac{1}{2\pi\sqrt{1.5 \times 10^{-5}}} = \frac{1}{2\pi \times 3.873 \times 10^{-3}}$
    \item $f_0 = \frac{1}{0.02433} = 41.1$~Hz
\end{itemize}

\textbf{At resonance (41.1~Hz):}
\begin{itemize}
    \item $X_L = 2\pi \times 41.1 \times 0.15 = 38.7$~$\Omega$
    \item $X_C = \frac{1}{2\pi \times 41.1 \times 100 \times 10^{-6}} = 38.7$~$\Omega$ $\checkmark$ (equal)
    \item $Z = R = 2$~$\Omega$ (minimum impedance)
    \item $I = \frac{100}{2} = 50$~A (maximum current, 7.7$\times$ higher than at 50~Hz!)
    \item Power: $P = I^2 R = 50^2 \times 2 = 5000$~W (all power dissipated in R)
\end{itemize}

\vspace{0.15cm}

\textbf{Example 4: Critical Damping Resistance}

\textbf{Given:} L = 0.15~H, C = 100~µF, want critically damped response

\textbf{Calculate $R_{critical}$:}
\begin{itemize}
    \item $R_{critical} = 2\sqrt{\frac{L}{C}} = 2\sqrt{\frac{0.15}{100 \times 10^{-6}}}$
    \item $R_{critical} = 2\sqrt{1500} = 2 \times 38.73 = 77.46$~$\Omega$
\end{itemize}

\textbf{Damping behavior:}
\begin{itemize}
    \item $R = 2$~$\Omega$ (Example 2): Underdamped (many oscillations, $R \ll R_{critical}$)
    \item $R = 77.46$~$\Omega$: Critically damped (fastest settle, no overshoot)
    \item $R = 500$~$\Omega$: Overdamped (slow settle, no oscillation, $R > R_{critical}$)
\end{itemize}
\end{examplebox}

\vspace{0.2cm}

\noindent\textbf{\color{accentcolor} Key Points (Interview Focus)}
\begin{keypointsbox}
\begin{itemize}
    \item \textbf{Wheatstone Bridge Balance:} $\frac{R_1}{R_2} = \frac{R_3}{R_4}$ when voltmeter reads 0~V. Two voltage dividers in parallel, bridge voltage = difference between divider midpoints. Balanced when ratios equal, independent of supply voltage. To measure unknown: Set 3 resistors, adjust 4th until balance, calculate from ratio. Precision method for low resistance (m$\Omega$), sensor interfacing
    
    \item \textbf{Series RLC Impedance:} $Z = \sqrt{R^2 + (X_L - X_C)^2}$ where $X_L = 2\pi fL$ (inductive reactance increases with f), $X_C = \frac{1}{2\pi fC}$ (capacitive reactance decreases with f). Phasor addition: $V_L$ leads current 90°, $V_C$ lags 90° (opposite), $V_R$ in phase. Supply voltage vector sum: $V_s = \sqrt{V_R^2 + (V_L - V_C)^2}$
    
    \item \textbf{Resonance Condition:} $X_L = X_C$ at $f_0 = \frac{1}{2\pi\sqrt{LC}}$. Reactances cancel, impedance minimum = R, current maximum $I = V_s/R$. All supply voltage across R, but $V_L$ and $V_C$ can exceed $V_s$ (high Q). Used in radio tuning: Adjust L or C to change $f_0$, select desired frequency. Power factor = 1 (resistive, voltage and current in phase)
    
    \item \textbf{LC Oscillation:} Energy oscillates between magnetic field (L) and electric field (C) at $f_0 = \frac{1}{2\pi\sqrt{LC}}$. Capacitor charges $\rightarrow$ discharges through L $\rightarrow$ builds magnetic field $\rightarrow$ field collapses $\rightarrow$ charges C opposite polarity $\rightarrow$ repeat. Ideal LC: Oscillates forever (energy conserved). Real RLC: Resistance dissipates energy ($P = I^2R$), oscillations decay
    
    \item \textbf{Damping Types:} Underdamped ($R < R_{crit}$): Many oscillations, exponential decay envelope. Critically damped ($R = 2\sqrt{L/C}$): No oscillation, fastest settle without overshoot, optimal for control. Overdamped ($R > R_{crit}$): No oscillation, slow exponential return. Small R $\rightarrow$ oscillates, large R $\rightarrow$ smooth decay
    
    \item \textbf{Frequency Dependence:} At low f: $X_C$ high (capacitor blocks), $X_L$ low (inductor passes), circuit capacitive. At high f: $X_L$ high (inductor blocks), $X_C$ low (capacitor passes), circuit inductive. At $f_0$: $X_L = X_C$ (resonance), circuit resistive. Impedance varies with frequency, minimum at resonance
    
    \item \textbf{Q: Why can $V_L$ and $V_C$ exceed supply voltage in RLC?} A: At resonance or near resonance, reactive voltages can be Q times supply voltage where $Q = \frac{X_L}{R}$ (quality factor). Example: If Q=20, $V_L = V_C = 20V_s$. They are 180° out of phase, cancel in phasor sum: $V_L - V_C = 0$ at exact resonance. Energy circulates between L and C, creating large voltages but no net reactive voltage. Supply only overcomes resistive drop
    
    \item \textbf{Q: How does Wheatstone bridge sensor work?} A: Replace one resistor with sensor (LDR, strain gauge, RTD). Sensor resistance changes with physical quantity (light, force, temperature). Bridge balanced at reference condition. Change in sensor $\rightarrow$ unbalances bridge $\rightarrow$ voltage appears between midpoints. Op-amp amplifies this voltage (µV--mV range) to usable level. Output proportional to sensor change. Temperature compensation: Use matched sensors in opposite bridge arms (cancel common effects)
    
    \item \textbf{Q: Why is critical damping important?} A: Critical damping ($R = 2\sqrt{L/C}$) gives fastest possible return to equilibrium without overshoot. Underdamped: Overshoots and oscillates (slow settling, ringing). Overdamped: No overshoot but slow (takes longer to reach final value). Critical: Perfect balance, reaches zero in minimum time with no overshoot. Used in: Door closers, shock absorbers, galvanometer damping, control systems requiring fast accurate response
\end{itemize}
\end{keypointsbox}
