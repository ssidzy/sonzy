% ====================================================================
% Section 08 -- Capacitors
% ====================================================================

\section{Section 08 -- Capacitors}

% --------------------------------------------------------------------
\subsection{Capacitor Introduction}

\noindent\textbf{\color{accentcolor} TL;DR (The Gist)}
\begin{tldrbox}
\begin{itemize}
    \item \textbf{Capacitor:} Two-terminal passive component that stores electrical energy
    \item \textbf{Unit:} Farad (F); typically pF to $\mu$F range (1 F is huge!)
    \item \textbf{Symbol:} Two parallel lines (flat for non-polarized, curved for polarized/electrolytic)
    \item \textbf{Key property:} Stores charge like a battery but releases it much faster
\end{itemize}
\end{tldrbox}

\vspace{0.2cm}

\noindent\textbf{\color{accentcolor} Detailed Explanation}
\begin{detailbox}
\textbf{What is a Capacitor?}

\vspace{0.15cm}

\textbf{Definition and Purpose:}
\begin{itemize}
    \item \textbf{Two-terminal electrical component}
    \item Special ability: \textbf{Store electrical energy}
    \item One of the three fundamental passive components (resistor, capacitor, inductor)
    \item Found in virtually every electronic circuit
    \item Works differently from a battery
\end{itemize}

\textbf{Comparison with batteries:}
\begin{itemize}
    \item \textbf{Batteries:} Store energy chemically, release slowly (hours to years)
    \item Example: Quartz watch battery lasts several years
    \item \textbf{Capacitors:} Store energy electrically, release rapidly (seconds or less)
    \item Example: Camera flash - charges for ~1 second, releases instantly
\end{itemize}

\vspace{0.15cm}

\textbf{Practical Application - Camera Flash:}

\textit{How it works:}
\begin{enumerate}
    \item Camera battery charges the flash capacitor (takes ~1 second)
    \item Capacitor stores energy
    \item When photo taken, capacitor releases all energy instantly
    \item Energy drives xenon flash bulb for brief, intense burst of light
    \item Fraction of a second discharge
\end{enumerate}

\textit{Why not use battery directly?}
\begin{itemize}
    \item Battery cannot deliver huge burst of current instantly
    \item Capacitor can - perfect for high-power, short-duration needs
    \item Capacitor acts as temporary high-current source
\end{itemize}

\vspace{0.15cm}

\textbf{Schematic Symbols:}

\textbf{Two common representations:}

\textit{1. Non-polarized capacitor (flat lines):}
\begin{itemize}
    \item Two parallel straight lines
    \item Close together but not touching
    \item Can be connected either way (no polarity)
    \item Typical for ceramic, film capacitors
\end{itemize}

\textit{2. Polarized capacitor (curved line):}
\begin{itemize}
    \item One straight line, one curved line
    \item Curved line indicates negative terminal (cathode)
    \item Indicates \textbf{polarity matters}
    \item Usually electrolytic capacitors
    \item MUST connect + to higher voltage, - to lower
\end{itemize}

\textbf{Labeling:}
\begin{itemize}
    \item Each capacitor has designator: C1, C2, C3, etc.
    \item Value indicates capacitance (how many Farads)
    \item Example: C1 = 100$\mu$F (100 microfarads)
\end{itemize}

\vspace{0.15cm}

\textbf{Capacitance - Measuring Storage Ability:}

\textbf{Definition:}
\begin{itemize}
    \item Capacitance = amount of charge capacitor can store
    \item More capacitance = more charge storage capacity
    \item Unit: \textbf{Farad (F)}
    \item Abbreviated: F
\end{itemize}

\textbf{The Farad is HUGE:}
\begin{itemize}
    \item 1 Farad is enormous in electronics
    \item Even 0.001 F (1 millifarad) is considered big
    \item Typical capacitors: picofarads to microfarads
    \item Large capacitors rarely exceed a few thousand $\mu$F
\end{itemize}

\vspace{0.15cm}

\textbf{Common Capacitance Ranges:}

\textbf{Picofarads (pF):}
\begin{itemize}
    \item $1 \text{ pF} = 10^{-12}$ F
    \item Very small capacitance
    \item Used in high-frequency circuits (RF, oscillators)
    \item Typical range: 1 pF to 1,000 pF
\end{itemize}

\textbf{Nanofarads (nF):}
\begin{itemize}
    \item $1 \text{ nF} = 10^{-9}$ F = 1,000 pF
    \item Medium-small capacitance
    \item General-purpose filtering
    \item Typical range: 1 nF to 1,000 nF
\end{itemize}

\textbf{Microfarads ($\mu$F):}
\begin{itemize}
    \item $1 \text{ $\mu$F} = 10^{-6}$ F = 1,000 nF
    \item Most common range in electronics
    \item Power supply filtering, coupling, decoupling
    \item Typical range: 1 $\mu$F to several thousand $\mu$F
\end{itemize}

\textbf{Farads (F):}
\begin{itemize}
    \item Used only in \textbf{supercapacitors} or \textbf{ultracapacitors}
    \item Special capacitors for energy storage
    \item Rare in typical electronics
    \item Can replace small batteries
    \item Range: 0.1 F to several hundred F
\end{itemize}

\vspace{0.15cm}

\textbf{Capacitance Hierarchy (small to large):}

\begin{center}
\begin{tabular}{|c|c|c|}
\hline
\textbf{Unit} & \textbf{Conversion} & \textbf{Typical Use} \\
\hline
pF (picofarad) & $10^{-12}$ F & High frequency, RF \\
nF (nanofarad) & $10^{-9}$ F & Filtering, coupling \\
$\mu$F (microfarad) & $10^{-6}$ F & Power supplies, general \\
F (farad) & 1 F & Supercaps, energy storage \\
\hline
\end{tabular}
\end{center}

\vspace{0.15cm}

\textbf{Key Takeaways:}

\textbf{Primary function:}
\begin{itemize}
    \item Store electrical energy
    \item Release energy quickly when needed
    \item Similar to battery but much faster discharge
\end{itemize}

\textbf{Capacitance value:}
\begin{itemize}
    \item Measured in Farads (F)
    \item Typical range: pF to $\mu$F
    \item Larger value = more charge storage
\end{itemize}

\textbf{Symbol recognition:}
\begin{itemize}
    \item Flat lines = non-polarized (can connect either way)
    \item Curved line = polarized (polarity matters!)
    \item Always labeled with value and designator
\end{itemize}
\end{detailbox}

\vspace{0.2cm}

\noindent\textbf{\color{accentcolor} Practical Example \& Numerical}
\begin{examplebox}
\textbf{Example 1: Camera Flash Capacitor}

\textit{Typical camera flash circuit:}

\textbf{Specifications:}
\begin{itemize}
    \item Capacitor: 300 $\mu$F
    \item Charging voltage: 300V
    \item Charging time: ~1 second (from camera battery)
    \item Discharge time: ~0.001 second (1 millisecond)
\end{itemize}

\textbf{Energy stored:}
\begin{equation*}
    E = \frac{1}{2}CV^2 = \frac{1}{2} \times 300 \times 10^{-6} \times 300^2 = 13.5\text{ J}
\end{equation*}

\textbf{Power during flash:}
\begin{equation*}
    P = \frac{E}{t} = \frac{13.5}{0.001} = \boxed{13{,}500\text{ W}}
\end{equation*}

Huge instantaneous power from tiny component!

\vspace{0.2cm}

\textbf{Example 2: Capacitance Unit Conversions}

\textbf{Convert 4,700 pF to nF:}
\begin{equation*}
    4{,}700\text{ pF} = \frac{4{,}700}{1{,}000} = \boxed{4.7\text{ nF}}
\end{equation*}

\textbf{Convert 0.047 $\mu$F to nF:}
\begin{equation*}
    0.047\text{ $\mu$F} = 0.047 \times 1{,}000 = \boxed{47\text{ nF}}
\end{equation*}

\textbf{Convert 220 nF to $\mu$F:}
\begin{equation*}
    220\text{ nF} = \frac{220}{1{,}000} = \boxed{0.22\text{ $\mu$F}}
\end{equation*}

\textbf{Convert 100 $\mu$F to F:}
\begin{equation*}
    100\text{ $\mu$F} = 100 \times 10^{-6} = \boxed{0.0001\text{ F}}
\end{equation*}

\vspace{0.2cm}

\textbf{Example 3: Comparing Capacitor vs Battery}

\textit{Scenario: Powering LED}

\textbf{Battery approach:}
\begin{itemize}
    \item 9V battery (alkaline)
    \item Energy capacity: ~20,000 J
    \item Can power LED for hours
    \item Slow discharge rate
\end{itemize}

\textbf{Capacitor approach:}
\begin{itemize}
    \item 1,000 $\mu$F, 10V capacitor
    \item Energy: $E = \frac{1}{2} \times 0.001 \times 10^2 = 0.05$ J
    \item Can power LED for ~1 second
    \item Rapid discharge
\end{itemize}

\textbf{Comparison:}
\begin{itemize}
    \item Battery: 400,000$\times$ more energy storage
    \item Capacitor: Can deliver current much faster
    \item Each suited for different applications
\end{itemize}

\vspace{0.2cm}

\textbf{Example 4: Supercapacitor Energy Storage}

\textit{Modern supercapacitor:}

\textbf{Specifications:}
\begin{itemize}
    \item Capacitance: 10 F (yes, 10 Farads!)
    \item Max voltage: 2.7V
    \item Size: Similar to AA battery
\end{itemize}

\textbf{Energy stored:}
\begin{equation*}
    E = \frac{1}{2} \times 10 \times 2.7^2 = \boxed{36.45\text{ J}}
\end{equation*}

\textbf{Applications:}
\begin{itemize}
    \item Backup power for memory
    \item Energy harvesting
    \item Quick charge/discharge cycles
    \item Much longer lifespan than batteries
\end{itemize}

\vspace{0.2cm}

\textbf{Example 5: Why 1 Farad is Huge}

\textit{Consider typical circuit board:}

\textbf{Standard decoupling:}
\begin{itemize}
    \item 100 nF capacitor
    \item Across power pins
    \item Physical size: ~2mm $\times$ 1mm
\end{itemize}

\textbf{If using 1 F instead:}
\begin{equation*}
    \text{Ratio} = \frac{1\text{ F}}{100\text{ nF}} = \frac{1}{100 \times 10^{-9}} = 10{,}000{,}000\times
\end{equation*}

Would need 10 million times more physical space! (unrealistic)

This is why typical electronics uses pF to $\mu$F range.
\end{examplebox}

\vspace{0.2cm}

\noindent\textbf{\color{accentcolor} Key Points (Interview Focus)}
\begin{keypointsbox}
\begin{enumerate}
    \item \textbf{Capacitor:} Passive component storing electrical energy (two terminals)
    \item \textbf{Key difference from battery:} Fast discharge (seconds vs hours)
    \item \textbf{Unit:} Farad (F), typically pF to $\mu$F in circuits
    \item \textbf{Symbol:} Flat lines (non-polarized), curved line (polarized)
    \item \textbf{Capacitance:} Amount of charge it can store
    \item \textbf{1 Farad is huge:} Most caps in pF to $\mu$F range
    \item \textbf{Supercapacitors:} F range, special energy storage applications
    \item \textbf{Found everywhere:} Very few circuits without capacitors
\end{enumerate}

\textbf{Interview Questions:}
\begin{itemize}
    \item \textbf{Q:} What does a capacitor do? \\
    \textit{A:} Stores electrical energy and can release it quickly.
    
    \item \textbf{Q:} How is capacitor different from battery? \\
    \textit{A:} Capacitor releases energy much faster (seconds vs hours); battery stores chemically, capacitor electrically.
    
    \item \textbf{Q:} What is the unit of capacitance? \\
    \textit{A:} Farad (F).
    
    \item \textbf{Q:} Is 1 Farad typical in electronics? \\
    \textit{A:} No, 1 F is huge; typical range is picofarads (pF) to microfarads ($\mu$F).
    
    \item \textbf{Q:} What does curved line in capacitor symbol mean? \\
    \textit{A:} Polarized capacitor (electrolytic); polarity matters - must connect correctly.
    
    \item \textbf{Q:} Example of capacitor application? \\
    \textit{A:} Camera flash - stores energy for instant high-power light burst.
\end{itemize}

\textbf{Capacitance Units:}
\begin{itemize}
    \item 1 $\mu$F = 1,000 nF = 1,000,000 pF
    \item 1 nF = 1,000 pF
    \item 1 F = 1,000,000 $\mu$F (rarely used except supercaps)
\end{itemize}

\textbf{Applications:}
\begin{itemize}
    \item Power supply filtering and smoothing
    \item Coupling and decoupling signals
    \item Energy storage (flash, backup power)
    \item Timing circuits
    \item Filters (high-pass, low-pass, band-pass)
\end{itemize}

\textbf{Common Misconceptions:}
\begin{itemize}
    \item "Capacitors are like batteries" - Similar but release energy much faster
    \item "All capacitors same" - No, polarized vs non-polarized crucial difference
    \item "Bigger always better" - No, depends on application and frequency
\end{itemize}
\end{keypointsbox}

% --------------------------------------------------------------------
\subsection{How a Capacitor is Made}

\noindent\textbf{\color{accentcolor} TL;DR (The Gist)}
\begin{tldrbox}
\begin{itemize}
    \item \textbf{Structure:} Two metal plates + insulating dielectric between them
    \item \textbf{Capacitance formula:} $C = \epsilon_r \epsilon_0 \frac{A}{d}$ (larger area/smaller distance = more capacitance)
    \item \textbf{Dielectric:} Insulating material (paper, ceramic, plastic, etc.)
    \item \textbf{Key:} DC cannot flow through (dielectric blocks it), but voltage stored across plates
\end{itemize}
\end{tldrbox}

\vspace{0.2cm}

\noindent\textbf{\color{accentcolor} Detailed Explanation}
\begin{detailbox}
\textbf{Basic Construction:}

\vspace{0.15cm}

\textbf{Three essential components:}
\begin{enumerate}
    \item \textbf{Two metal plates} (conductive)
    \item \textbf{Dielectric} (insulator between plates)
    \item \textbf{Terminal wires} (connect to circuit)
\end{enumerate}

\textit{The schematic symbol resembles actual construction:}
\begin{itemize}
    \item Two parallel lines = two metal plates
    \item Space between lines = dielectric insulator
    \item Plates close together but never touching
\end{itemize}

\vspace{0.15cm}

\textbf{The Metal Plates:}

\textbf{Material:}
\begin{itemize}
    \item \textbf{Aluminum} (most common, cheap)
    \item \textbf{Tantalum} (electrolytic capacitors)
    \item \textbf{Silver} (high-performance)
    \item Other conductive metals
\end{itemize}

\textbf{Properties:}
\begin{itemize}
    \item Placed parallel to each other
    \item Very close together
    \item Each connected to terminal wire
    \item Conductive surface area crucial for capacitance
\end{itemize}

\vspace{0.15cm}

\textbf{The Dielectric (Insulator):}

\textbf{Purpose:}
\begin{itemize}
    \item Separates the two plates
    \item Prevents direct electrical contact
    \item \textbf{Blocks DC current flow}
    \item Allows electric field to form between plates
    \item Material affects capacitance value
\end{itemize}

\textbf{Common dielectric materials:}

\textit{Paper:}
\begin{itemize}
    \item Older technology
    \item Impregnated with oil or wax
    \item Low cost
\end{itemize}

\textit{Ceramic:}
\begin{itemize}
    \item Very common today
    \item Stable, reliable
    \item Various formulations
    \item Used in ceramic disc capacitors
\end{itemize}

\textit{Plastic film:}
\begin{itemize}
    \item Polyester, polypropylene
    \item Good temperature stability
    \item Film capacitors
\end{itemize}

\textit{Glass:}
\begin{itemize}
    \item High precision
    \item Expensive
    \item Low loss
\end{itemize}

\textit{Aluminum oxide:}
\begin{itemize}
    \item Electrolytic capacitors
    \item Very thin layer (high capacitance)
    \item Polarized
\end{itemize}

\textit{Tantalum oxide:}
\begin{itemize}
    \item Tantalum electrolytic capacitors
    \item Stable, reliable
    \item Higher capacitance density
\end{itemize}

\textit{Air:}
\begin{itemize}
    \item Variable capacitors
    \item Tuning circuits (old radios)
    \item Low capacitance
\end{itemize}

\vspace{0.15cm}

\textbf{Why DC Cannot Flow Through Capacitor:}

\textbf{The dielectric acts as insulator:}
\begin{itemize}
    \item Blocks direct current path
    \item Electrons cannot physically travel through insulator
    \item Current cannot flow from one plate to other
    \item \textbf{DC is blocked}
\end{itemize}

\textbf{But voltage can exist across plates:}
\begin{itemize}
    \item Positive charges accumulate on one plate
    \item Negative charges accumulate on other plate
    \item Electric field forms between plates
    \item Voltage present even though no current flows
    \item Energy stored in electric field
\end{itemize}

\vspace{0.15cm}

\textbf{Capacitance Formula:}

\textbf{Factors determining capacitance:}

The capacitance value depends on physical construction:
\begin{equation*}
    \boxed{C = \epsilon_r \epsilon_0 \frac{A}{d}}
\end{equation*}

Where:
\begin{itemize}
    \item $C$ = capacitance (F)
    \item $\epsilon_r$ = relative permittivity of dielectric (dimensionless constant)
    \item $\epsilon_0$ = permittivity of free space ($8.854 \times 10^{-12}$ F/m)
    \item $A$ = overlapping area of plates (m²)
    \item $d$ = distance between plates (m)
\end{itemize}

\vspace{0.15cm}

\textbf{How Each Factor Affects Capacitance:}

\textbf{Plate Area (A):}

\textit{Larger area $\rightarrow$ MORE capacitance}
\begin{itemize}
    \item More surface to accumulate charges
    \item Directly proportional: 2$\times$ area = 2$\times$ capacitance
    \item To increase capacitance, increase plate size
\end{itemize}

\textbf{Distance Between Plates (d):}

\textit{Smaller distance $\rightarrow$ MORE capacitance}
\begin{itemize}
    \item Charges closer together = stronger electric field
    \item Inversely proportional: ½ distance = 2$\times$ capacitance
    \item Thinner dielectric = higher capacitance
    \item But too thin risks breakdown
\end{itemize}

\textbf{Dielectric Constant ($\epsilon_r$):}

\textit{Higher permittivity $\rightarrow$ MORE capacitance}
\begin{itemize}
    \item Material property of dielectric
    \item Vacuum/air: $\epsilon_r = 1$ (reference)
    \item Ceramic: $\epsilon_r \approx 10 - 10{,}000$ (varies widely)
    \item Aluminum oxide: $\epsilon_r \approx 7-10$
    \item Tantalum oxide: $\epsilon_r \approx 25$
    \item Higher $\epsilon_r$ allows more charge storage
\end{itemize}

\vspace{0.15cm}

\textbf{Design Tradeoffs:}

\textbf{To maximize capacitance:}
\begin{enumerate}
    \item Use larger plate area (increases size/cost)
    \item Reduce plate spacing (risk of breakdown, manufacturing limits)
    \item Choose dielectric with high $\epsilon_r$ (material selection)
\end{enumerate}

\textbf{Why electrolytic capacitors have high capacitance:}
\begin{itemize}
    \item Oxide layer extremely thin ($d$ very small)
    \item Effective plate area very large (etched, rough surface)
    \item Result: High capacitance in small volume
    \item Tradeoff: Must be polarized
\end{itemize}

\vspace{0.15cm}

\textbf{Summary of Construction:}

\textbf{Physical structure:}
\begin{itemize}
    \item Two parallel conductive plates
    \item Separated by thin insulating dielectric
    \item Each plate connected to terminal
    \item Entire assembly often rolled or stacked
\end{itemize}

\textbf{Electrical behavior:}
\begin{itemize}
    \item DC blocked by dielectric
    \item Voltage can exist across plates
    \item Charges accumulate on plates
    \item Electric field stores energy
\end{itemize}

\textbf{Capacitance depends on:}
\begin{itemize}
    \item Plate area (larger = more C)
    \item Plate spacing (smaller = more C)
    \item Dielectric material (higher $\epsilon_r$ = more C)
\end{itemize}
\end{detailbox}

\vspace{0.2cm}

\noindent\textbf{\color{accentcolor} Practical Example \& Numerical}
\begin{examplebox}
\textbf{Example 1: Simple Parallel-Plate Capacitor Calculation}

\textit{Design a capacitor with these specifications:}
\begin{itemize}
    \item Plate area: $A = 0.01 \text{ m}^2$ (100 cm²)
    \item Plate separation: $d = 0.001 \text{ m}$ (1 mm)
    \item Dielectric: Ceramic with $\epsilon_r = 10$
\end{itemize}

\textbf{Calculate capacitance:}
\begin{align*}
    C &= \epsilon_r \epsilon_0 \frac{A}{d} \\
    &= 10 \times 8.854 \times 10^{-12} \times \frac{0.01}{0.001} \\
    &= 10 \times 8.854 \times 10^{-12} \times 10 \\
    &= 8.854 \times 10^{-10} \text{ F} \\
    &= \boxed{885.4 \text{ pF}}
\end{align*}

\vspace{0.2cm}

\textbf{Example 2: Effect of Doubling Plate Area}

\textit{Using same capacitor from Example 1:}

\textbf{Original:}
\begin{itemize}
    \item $A = 0.01$ m², $C = 885.4$ pF
\end{itemize}

\textbf{Double the area:}
\begin{itemize}
    \item $A = 0.02$ m²
\end{itemize}

\textbf{New capacitance:}
\begin{align*}
    C_{new} &= 10 \times 8.854 \times 10^{-12} \times \frac{0.02}{0.001} \\
    &= \boxed{1{,}770.8 \text{ pF}}
\end{align*}

Result: 2$\times$ area = 2$\times$ capacitance $\checkmark$

\vspace{0.2cm}

\textbf{Example 3: Effect of Halving Distance}

\textit{Using original capacitor:}

\textbf{Original:}
\begin{itemize}
    \item $d = 0.001$ m (1 mm), $C = 885.4$ pF
\end{itemize}

\textbf{Halve the distance:}
\begin{itemize}
    \item $d = 0.0005$ m (0.5 mm)
\end{itemize}

\textbf{New capacitance:}
\begin{align*}
    C_{new} &= 10 \times 8.854 \times 10^{-12} \times \frac{0.01}{0.0005} \\
    &= 10 \times 8.854 \times 10^{-12} \times 20 \\
    &= \boxed{1{,}770.8 \text{ pF}}
\end{align*}

Result: ½ distance = 2$\times$ capacitance $\checkmark$

\vspace{0.2cm}

\textbf{Example 4: Effect of Different Dielectric}

\textit{Same physical dimensions, different materials:}

\textbf{Air dielectric ($\epsilon_r = 1$):}
\begin{equation*}
    C = 1 \times 8.854 \times 10^{-12} \times \frac{0.01}{0.001} = 88.54 \text{ pF}
\end{equation*}

\textbf{Ceramic dielectric ($\epsilon_r = 10$):}
\begin{equation*}
    C = 10 \times 8.854 \times 10^{-12} \times \frac{0.01}{0.001} = 885.4 \text{ pF}
\end{equation*}

\textbf{High-K ceramic ($\epsilon_r = 1000$):}
\begin{equation*}
    C = 1000 \times 8.854 \times 10^{-12} \times \frac{0.01}{0.001} = 88.54 \text{ nF}
\end{equation*}

Same size, 1000$\times$ more capacitance with better dielectric!

\vspace{0.2cm}

\textbf{Example 5: Why Electrolytic Capacitors Are Large}

\textit{Typical aluminum electrolytic:}

\textbf{Construction:}
\begin{itemize}
    \item Aluminum oxide dielectric: $\epsilon_r \approx 8$
    \item Oxide layer thickness: $d \approx 1$ $\mu$m = $10^{-6}$ m (very thin!)
    \item Etched foil increases effective area: $A \approx 0.1$ m² (large!)
\end{itemize}

\textbf{Capacitance:}
\begin{align*}
    C &= 8 \times 8.854 \times 10^{-12} \times \frac{0.1}{10^{-6}} \\
    &= 8 \times 8.854 \times 10^{-12} \times 10^5 \\
    &= 7.08 \times 10^{-6} \text{ F} \\
    &= \boxed{7.08 \text{ $\mu$F}}
\end{align*}

Extremely thin dielectric + large area = high capacitance in small volume!

\vspace{0.2cm}

\textbf{Example 6: Practical Ceramic Capacitor}

\textit{0805 SMD ceramic capacitor - 100 nF:}

\textbf{Physical dimensions:}
\begin{itemize}
    \item Size: 2mm $\times$ 1.2mm $\times$ 0.5mm (tiny!)
    \item Dielectric: High-K ceramic, $\epsilon_r \approx 2000$
    \item Multiple layers stacked
\end{itemize}

\textbf{Estimation (simplified):}
\begin{itemize}
    \item Effective area (multi-layer): $A \approx 0.0001$ m²
    \item Dielectric thickness: $d \approx 10$ $\mu$m
\end{itemize}

\begin{align*}
    C &= 2000 \times 8.854 \times 10^{-12} \times \frac{0.0001}{10 \times 10^{-6}} \\
    &\approx 177 \text{ nF}
\end{align*}

Close to 100 nF (actual has optimized geometry).
\end{examplebox}

\vspace{0.2cm}

\noindent\textbf{\color{accentcolor} Key Points (Interview Focus)}
\begin{keypointsbox}
\begin{enumerate}
    \item \textbf{Structure:} Two parallel metal plates separated by insulating dielectric
    \item \textbf{Formula:} $C = \epsilon_r \epsilon_0 \frac{A}{d}$
    \item \textbf{Larger plates:} More area $\rightarrow$ more capacitance
    \item \textbf{Closer plates:} Smaller distance $\rightarrow$ more capacitance (inversely proportional)
    \item \textbf{Dielectric material:} Higher $\epsilon_r$ $\rightarrow$ more capacitance
    \item \textbf{DC blocking:} Dielectric prevents current flow through capacitor
    \item \textbf{Voltage storage:} Charges accumulate on plates, electric field stores energy
    \item \textbf{Electrolytic secret:} Ultra-thin oxide + large effective area = high C
\end{enumerate}

\textbf{Interview Questions:}
\begin{itemize}
    \item \textbf{Q:} What are the main parts of a capacitor? \\
    \textit{A:} Two metal plates and insulating dielectric between them.
    
    \item \textbf{Q:} Why can't DC current flow through capacitor? \\
    \textit{A:} Dielectric is an insulator - blocks direct current path.
    
    \item \textbf{Q:} How to increase capacitance? \\
    \textit{A:} Increase plate area, decrease plate spacing, or use dielectric with higher permittivity.
    
    \item \textbf{Q:} What is dielectric? \\
    \textit{A:} Insulating material between plates (ceramic, plastic, paper, oxide, etc.).
    
    \item \textbf{Q:} If you double plate area, what happens to capacitance? \\
    \textit{A:} Capacitance doubles (directly proportional).
    
    \item \textbf{Q:} If you halve distance between plates? \\
    \textit{A:} Capacitance doubles (inversely proportional to distance).
\end{itemize}

\textbf{Capacitance Formula:}
\begin{itemize}
    \item $C = \epsilon_r \epsilon_0 \frac{A}{d}$
    \item $\epsilon_0 = 8.854 \times 10^{-12}$ F/m (constant)
    \item $\epsilon_r$ = dielectric constant (material property)
    \item Proportional to area $A$
    \item Inversely proportional to distance $d$
\end{itemize}

\textbf{Common Dielectrics:}
\begin{itemize}
    \item Air: $\epsilon_r = 1$
    \item Paper: $\epsilon_r \approx 3-4$
    \item Ceramic: $\epsilon_r \approx 10-10{,}000$
    \item Aluminum oxide: $\epsilon_r \approx 8-10$
    \item Tantalum oxide: $\epsilon_r \approx 25$
\end{itemize}

\textbf{Common Mistakes:}
\begin{itemize}
    \item Thinking current flows through capacitor (DC blocked!)
    \item Confusing area with volume
    \item Forgetting inverse relationship with distance
\end{itemize}
\end{keypointsbox}

% --------------------------------------------------------------------
\subsection{Types of Capacitors}

\noindent\textbf{\color{accentcolor} TL;DR (The Gist)}
\begin{tldrbox}
\begin{itemize}
    \item \textbf{Ceramic:} Small, cheap, low ESR, ideal for high-frequency coupling/decoupling
    \item \textbf{Electrolytic:} High capacitance, polarized, higher ESR, power supply filtering
    \item \textbf{Supercapacitors:} Farad-range, energy storage, low voltage rating
    \item Key factors: Size, voltage rating, leakage, ESR, tolerance
\end{itemize}
\end{tldrbox}

\vspace{0.2cm}

\noindent\textbf{\color{accentcolor} Detailed Explanation}
\begin{detailbox}
\textbf{Factors to Consider When Choosing Capacitors:}

\vspace{0.15cm}

\textbf{1. Size (Physical and Capacitance):}
\begin{itemize}
    \item \textbf{Physical volume:} Can be largest component in circuit or tiny SMD
    \item \textbf{Capacitance value:} More capacitance typically requires larger size
    \item Tradeoff between performance and board space
\end{itemize}

\textbf{2. Maximum Voltage Rating:}
\begin{itemize}
    \item Each capacitor rated for max voltage across it
    \item Ratings: 6.3V, 10V, 16V, 25V, 50V, 100V, 400V, etc.
    \item \textbf{Exceeding voltage rating destroys capacitor}
    \item Always design with safety margin (use 50-70\% of max rating)
\end{itemize}

\textbf{3. Leakage Current:}
\begin{itemize}
    \item \textbf{No capacitor is perfect}
    \item Tiny current leaks through dielectric (nanoamps typically)
    \item Causes slow energy drain
    \item Electrolytic capacitors have higher leakage
    \item Ceramic and film have very low leakage
\end{itemize}

\textbf{4. Equivalent Series Resistance (ESR):}
\begin{itemize}
    \item Terminals not 100\% conductive
    \item Small resistance in series (usually $< 0.01\Omega$)
    \item Becomes problem with high currents $\rightarrow$ heat and power loss
    \item \textbf{Remember ESR - important for next topics!}
    \item Ceramic: very low ESR
    \item Electrolytic: higher ESR
\end{itemize}

\textbf{5. Tolerance:}
\begin{itemize}
    \item Actual capacitance varies from nominal value
    \item Typical tolerances: $\pm$1\%, $\pm$5\%, $\pm$10\%, $\pm$20\%
    \item Precision caps: $\pm$1\% (expensive)
    \item General purpose: $\pm$10\% to $\pm$20\%
\end{itemize}

\vspace{0.15cm}

\textbf{CERAMIC CAPACITORS (Most Common):}

\textbf{Market share:} ~80\% of all capacitors produced

\textbf{Construction:}
\begin{itemize}
    \item Dielectric: Ceramic material
    \item Small physical size
    \item Small capacitance values (pF to low $\mu$F)
    \item Surface mount: 0402, 0603, 0805, 1206 packages
    \item Through-hole: Yellow/brown disc shape
\end{itemize}

\textbf{Characteristics:}
\begin{itemize}
    \item \textbf{Very low ESR} (near ideal)
    \item \textbf{Very low leakage}
    \item Non-polarized (can connect either way)
    \item Usually hard to find > 10 $\mu$F ceramic
    \item Least expensive option
    \item Excellent high-frequency performance
\end{itemize}

\textbf{Applications:}
\begin{itemize}
    \item High-frequency coupling
    \item Decoupling/bypass capacitors
    \item Filtering
    \item Timing circuits
    \item General-purpose applications
\end{itemize}

\textbf{Limitations:}
\begin{itemize}
    \item Limited to smaller capacitance values
    \item Some types have voltage coefficient (C changes with voltage)
    \item Temperature coefficient varies by type
\end{itemize}

\vspace{0.15cm}

\textbf{ELECTROLYTIC CAPACITORS (Aluminum \& Tantalum):}

\textbf{Why they exist:}
\begin{itemize}
    \item High capacitance in small volume
    \item 1 $\mu$F to 1,000,000 $\mu$F range
    \item Oxide layer extremely thin $\rightarrow$ high capacitance
    \item Well-suited for high-voltage applications
\end{itemize}

\textbf{Aluminum Electrolytic:}

\textit{Appearance:}
\begin{itemize}
    \item Tin can shape (cylindrical)
    \item Both leads from bottom (radial)
    \item Or leads from each end (axial)
    \item Capacitance and voltage marked on body
\end{itemize}

\textit{Critical: POLARIZED}
\begin{itemize}
    \item \textbf{Anode (+):} Positive terminal (longer lead usually)
    \item \textbf{Cathode (-):} Negative terminal (marked with stripe/arrow)
    \item Anode MUST be at higher voltage than cathode
    \item \textbf{Reverse voltage $\rightarrow$ POP and failure!}
\end{itemize}

\textit{What happens if reversed:}
\begin{itemize}
    \item Spectacular failure (popping sound)
    \item Electrolyte vents/bursts
    \item Becomes short circuit (permanent damage)
    \item Never apply reverse voltage!
\end{itemize}

\textit{Characteristics:}
\begin{itemize}
    \item High capacitance: 1 $\mu$F - 100,000 $\mu$F+
    \item Voltage ratings: 6.3V to 450V typical
    \item \textbf{Higher leakage current} (nanoamps to microamps)
    \item \textbf{Higher ESR} than ceramic
    \item Less ideal for energy storage (due to leakage)
    \item Polarity critical
\end{itemize}

\textit{Applications:}
\begin{itemize}
    \item Power supply filtering/smoothing
    \item Bulk energy storage
    \item Audio coupling (high capacitance needed)
    \item DC blocking with large signals
\end{itemize}

\textbf{Tantalum Electrolytic:}

\textit{Advantages over aluminum:}
\begin{itemize}
    \item Smaller size for same capacitance
    \item Lower ESR
    \item Better frequency response
    \item More stable over temperature
\end{itemize}

\textit{Disadvantages:}
\begin{itemize}
    \item More expensive
    \item Still polarized
    \item Lower voltage ratings typically
    \item Can fail catastrophically if abused
\end{itemize}

\vspace{0.15cm}

\textbf{SUPERCAPACITORS (Ultracapacitors):}

\textbf{Purpose:} Energy storage (battery replacement/supplement)

\textbf{Characteristics:}
\begin{itemize}
    \item \textbf{Huge capacitance:} Farads range (0.1F to 3,000F!)
    \item \textbf{Low voltage:} Typically 2.5V to 2.7V max per cell
    \item Can chain in series for higher voltage (reduces total C)
    \item Physical size similar to battery
\end{itemize}

\textbf{Market share:} ~2\% of capacitor market

\textbf{vs Batteries:}

\textit{Advantages:}
\begin{itemize}
    \item Much faster charge/discharge
    \item Much longer lifespan (millions of cycles)
    \item No chemical degradation
    \item More environmentally friendly
\end{itemize}

\textit{Disadvantages:}
\begin{itemize}
    \item Cannot hold as much energy as battery (for same size)
    \item Low voltage rating
    \item Self-discharge (leakage) higher than batteries
    \item More expensive per joule stored
\end{itemize}

\textbf{Applications:}
\begin{itemize}
    \item Backup power (memory retention)
    \item Energy harvesting systems
    \item Regenerative braking (vehicles)
    \item Peak power delivery
    \item Quick charge/discharge applications
\end{itemize}

\vspace{0.15cm}

\textbf{FILM CAPACITORS:}

\textbf{Construction:}
\begin{itemize}
    \item Plastic film dielectric (polyester, polypropylene, etc.)
    \item Very low parasitic losses
    \item Non-polarized
\end{itemize}

\textbf{Characteristics:}
\begin{itemize}
    \item Excellent for high currents
    \item Low ESR
    \item Good temperature stability
    \item Self-healing property (some types)
\end{itemize}

\textbf{Applications:}
\begin{itemize}
    \item High-voltage applications
    \item High-frequency, high-current
    \item Motor run capacitors
    \item Power factor correction
\end{itemize}

\vspace{0.15cm}

\textbf{MICA CAPACITORS:}

\textbf{Characteristics:}
\begin{itemize}
    \item Very high temperature tolerance: up to 237°C
    \item Excellent long-term stability (0.01\% - 0.02\% change)
    \item High precision, low loss
    \item Low temperature coefficient
\end{itemize}

\textbf{Applications:}
\begin{itemize}
    \item High-precision analog circuits
    \item High-quality audio (Hi-Fi)
    \item High-frequency circuits
    \item RF applications
\end{itemize}

\textbf{Limitations:}
\begin{itemize}
    \item Complex, expensive manufacturing
    \item Bulky compared to modern ceramics
    \item Limited capacitance values
    \item High cost
\end{itemize}

\vspace{0.15cm}

\textbf{Summary Table:}

\begin{center}
\small
\begin{tabular}{|l|l|l|l|l|}
\hline
\textbf{Type} & \textbf{Capacitance} & \textbf{Polarized?} & \textbf{ESR} & \textbf{Use} \\
\hline
Ceramic & pF - 10$\mu$F & No & Very Low & General, HF \\
Aluminum Elec & 1$\mu$F - 100mF & Yes & Medium & Power supply \\
Tantalum Elec & 1$\mu$F - 1mF & Yes & Low-Med & Compact, stable \\
Supercap & 0.1F - 3000F & Yes/No & Medium & Energy storage \\
Film & nF - 100$\mu$F & No & Low & High V, high I \\
Mica & pF - nF & No & Very Low & Precision, RF \\
\hline
\end{tabular}
\end{center}
\end{detailbox}

\vspace{0.2cm}

\noindent\textbf{\color{accentcolor} Practical Example \& Numerical}
\begin{examplebox}
\textbf{Example 1: Choosing Capacitor for Power Supply Filter}

\textit{Requirements:}
\begin{itemize}
    \item Filter 12V power supply
    \item Need 1,000 $\mu$F capacitance
    \item Low-frequency (60 Hz ripple)
\end{itemize}

\textbf{Choice: Aluminum Electrolytic}

\textit{Why?}
\begin{itemize}
    \item Need high capacitance (1,000 $\mu$F)
    \item Ceramic typically doesn't go this high
    \item Low frequency (ESR not critical)
    \item DC application (polarity OK)
\end{itemize}

\textbf{Specifications:}
\begin{itemize}
    \item Capacitance: 1,000 $\mu$F
    \item Voltage rating: 25V (2$\times$ safety margin over 12V)
    \item Type: Radial aluminum electrolytic
    \item Observe polarity!
\end{itemize}

\vspace{0.2cm}

\textbf{Example 2: Decoupling Capacitor for Digital IC}

\textit{Requirements:}
\begin{itemize}
    \item Decouple 5V digital IC
    \item High-frequency noise (MHz range)
    \item Small board space
\end{itemize}

\textbf{Choice: Ceramic 0.1 $\mu$F}

\textit{Why?}
\begin{itemize}
    \item Very low ESR (fast response to current spikes)
    \item Good high-frequency performance
    \item Small SMD package (0805 or 0603)
    \item Non-polarized (easy placement)
    \item Inexpensive
\end{itemize}

\textbf{Specifications:}
\begin{itemize}
    \item 0.1 $\mu$F (100 nF)
    \item 10V or 16V rating
    \item X7R ceramic (stable)
    \item 0805 package
\end{itemize}

\vspace{0.2cm}

\textbf{Example 3: Supercapacitor Backup Power}

\textit{System needs:}
\begin{itemize}
    \item 100 mA current for 10 seconds during power loss
    \item 3.3V system voltage
\end{itemize}

\textbf{Energy required:}
\begin{align*}
    E &= P \times t = V \times I \times t \\
    &= 3.3 \times 0.1 \times 10 = 3.3\text{ J}
\end{align*}

\textbf{Capacitor energy:}
\begin{equation*}
    E = \frac{1}{2}CV^2 \quad \Rightarrow \quad C = \frac{2E}{V^2}
\end{equation*}

\begin{align*}
    C &= \frac{2 \times 3.3}{3.3^2} = \frac{6.6}{10.89} = 0.606\text{ F}
\end{align*}

\textbf{Choice:} 1F supercapacitor @ 5.5V (provides margin)

\vspace{0.2cm}

\textbf{Example 4: Voltage Rating Safety}

\textit{Bad design:}
\begin{itemize}
    \item 12V supply
    \item Using 16V rated capacitor
    \item Voltage spikes can exceed 16V $\rightarrow$ \textbf{FAILURE}
\end{itemize}

\textit{Good design:}
\begin{itemize}
    \item 12V supply
    \item Using 25V rated capacitor
    \item Safety factor: 25/12 = 2.08$\times$ (good margin)
    \item Can tolerate transients and spikes
\end{itemize}

\textbf{Rule of thumb:} Use 1.5$\times$ to 2$\times$ voltage rating of actual voltage

\vspace{0.2cm}

\textbf{Example 5: ESR Impact on Filtering}

\textit{Two capacitors, same value (1,000 $\mu$F):}

\textbf{Aluminum electrolytic:}
\begin{itemize}
    \item ESR: 0.5$\Omega$ (typical)
    \item At 1A ripple current: $V_{ripple} = I \times ESR = 1 \times 0.5 = 0.5V$
\end{itemize}

\textbf{Low-ESR electrolytic:}
\begin{itemize}
    \item ESR: 0.05$\Omega$
    \item At 1A ripple: $V_{ripple} = 1 \times 0.05 = 0.05V$
\end{itemize}

Low-ESR capacitor provides 10$\times$ better ripple reduction!
\end{examplebox}

\vspace{0.2cm}

\noindent\textbf{\color{accentcolor} Key Points (Interview Focus)}
\begin{keypointsbox}
\begin{enumerate}
    \item \textbf{Ceramic:} Most common (80\%), low ESR, small C, non-polarized, ideal HF
    \item \textbf{Electrolytic:} High C ($\mu$F-mF), polarized, must respect polarity or fails
    \item \textbf{Supercaps:} Farad range, energy storage, low voltage (2.5V typical)
    \item \textbf{ESR:} Equivalent series resistance, lower is better for AC/switching
    \item \textbf{Leakage:} All caps leak slightly; electrolytic worst, ceramic best
    \item \textbf{Voltage rating:} Never exceed; use 1.5-2$\times$ safety factor
    \item \textbf{Polarity:} Electrolytic MUST be connected correctly (+ to higher V)
    \item \textbf{Film \& Mica:} Special apps (precision, high-temp, high-freq)
\end{enumerate}

\textbf{Interview Questions:}
\begin{itemize}
    \item \textbf{Q:} What happens if you reverse electrolytic capacitor? \\
    \textit{A:} It will pop/burst and fail, becoming short circuit.
    
    \item \textbf{Q:} When to use ceramic vs electrolytic? \\
    \textit{A:} Ceramic for small C, high-freq, decoupling; electrolytic for large C, power filtering.
    
    \item \textbf{Q:} What is ESR? \\
    \textit{A:} Equivalent Series Resistance - internal resistance causing power loss and limiting high-freq performance.
    
    \item \textbf{Q:} Why supercapacitors special? \\
    \textit{A:} Farad-range capacitance for energy storage, but low voltage rating (2.5-2.7V).
    
    \item \textbf{Q:} How to identify electrolytic polarity? \\
    \textit{A:} Negative terminal marked with stripe; positive lead usually longer.
    
    \item \textbf{Q:} What's typical voltage safety margin? \\
    \textit{A:} Use capacitor rated 1.5$\times$ to 2$\times$ actual operating voltage.
\end{itemize}

\textbf{Selection Criteria:}
\begin{itemize}
    \item Capacitance value needed
    \item Frequency of operation (ESR matters at high freq)
    \item Voltage rating (with safety margin)
    \item Physical size constraints
    \item Polarity requirements
    \item Cost
\end{itemize}

\textbf{Applications by Type:}
\begin{itemize}
    \item Ceramic: Decoupling, coupling, filtering, timing
    \item Electrolytic: Power supply smoothing, bulk storage
    \item Supercap: Backup power, energy harvesting
    \item Film: High voltage, high current, motor capacitors
    \item Mica: Precision RF, high-quality audio
\end{itemize}
\end{keypointsbox}

% --------------------------------------------------------------------
\subsection{How a Capacitor Works (in a DC Circuit)}

\noindent\textbf{\color{accentcolor} TL;DR (The Gist)}
\begin{tldrbox}
\begin{itemize}
    \item \textbf{Charging:} Current flows until capacitor voltage = supply voltage
    \item \textbf{Fully charged:} No more current flows (DC blocked by dielectric)
    \item \textbf{Discharging:} Stored charge released when path provided
    \item \textbf{Need resistor:} Limits charging current (prevents infinite current spike)
\end{itemize}
\end{tldrbox}

\vspace{0.2cm}

\noindent\textbf{\color{accentcolor} Detailed Explanation}
\begin{detailbox}
\textbf{Capacitor Behavior in DC Circuit:}

\vspace{0.15cm}

\textbf{The Insulating Dielectric:}
\begin{itemize}
    \item Dielectric is non-conductive (insulator)
    \item \textbf{Blocks DC current from flowing through}
    \item Current cannot physically pass through capacitor
    \item Instead, voltage exists across plates as electrical charge
\end{itemize}

\vspace{0.15cm}

\textbf{How Charging Works:}

\textbf{Initial state (uncharged):}
\begin{itemize}
    \item No charge on plates
    \item Voltage across capacitor = 0V
    \item Ready to accept current
\end{itemize}

\textbf{When voltage applied:}
\begin{enumerate}
    \item Current starts flowing into capacitor
    \item Electrons accumulate on one plate (becomes negative)
    \item Electrons repelled from other plate (becomes positive)
    \item Opposite charges attract but can't reach each other (dielectric blocks)
    \item Voltage across capacitor builds up
    \item As voltage increases, current decreases
    \item When capacitor voltage = supply voltage, current stops
\end{enumerate}

\textbf{Physical process:}
\begin{itemize}
    \item \textbf{Negative plate:} Electrons pile up (excess negative charge)
    \item \textbf{Positive plate:} Electrons removed (deficit = positive charge)
    \item Charges stuck on plates (nowhere to go)
    \item Electric field forms between plates
    \item Energy stored in this electric field
\end{itemize}

\vspace{0.15cm}

\textbf{When Fully Charged:}

\textbf{Conditions:}
\begin{itemize}
    \item Voltage across capacitor = supply voltage
    \item Plates "full" of charge (no more can fit)
    \item Negative charges on one plate repel new electrons
    \item \textbf{Current flow stops completely}
\end{itemize}

\textbf{Capacitor acts like:}
\begin{itemize}
    \item Open circuit (infinite resistance to DC)
    \item No current path
    \item Voltage present but no current
\end{itemize}

\textbf{Capacitance determines:}
\begin{itemize}
    \item Maximum charge storage
    \item Bigger capacitance = more charge = longer charging time
    \item For given current: $t_{charge} \propto C$
\end{itemize}

\vspace{0.15cm}

\textbf{Discharging Process:}

\textbf{When discharge path provided:}
\begin{enumerate}
    \item Positive and negative charges want to reunite
    \item If path created (e.g., resistor connected), current flows
    \item Current flows \textbf{opposite direction} from charging
    \item Charges neutralize
    \item Voltage across capacitor decreases
    \item When fully discharged, voltage = 0V
\end{enumerate}

\textbf{Current direction during discharge:}
\begin{itemize}
    \item Conventional current: positive plate $\rightarrow$ negative plate
    \item Opposite to charging current direction
    \item Makes sense: releasing stored energy
\end{itemize}

\textbf{Example with LED:}
\begin{itemize}
    \item Capacitor charged to 5V
    \item Connect LED across it
    \item Stored energy lights LED briefly
    \item LED dims as capacitor discharges
    \item LED turns off when capacitor empty
\end{itemize}

\vspace{0.15cm}

\textbf{Why Resistor Needed During Charging:}

\textbf{The problem (without resistor):}
\begin{itemize}
    \item Initially, capacitor voltage = 0V
    \item Supply voltage = 5V (for example)
    \item By Ohm's law: $I = V/R$
    \item If $R = 0$ (ideal wires, ideal supply, ideal capacitor)
    \item Then $I = V/0 = \infty$ (infinite current!)
\end{itemize}

\textbf{In simulation:}
\begin{itemize}
    \item Simulator treats components as ideal
    \item No internal resistance in ideal voltage source
    \item No resistance in ideal capacitor
    \item No resistance in ideal wires
    \item Results in error: "Loop has no resistance"
    \item Cannot compute (division by zero)
\end{itemize}

\textbf{In real world:}
\begin{itemize}
    \item \textbf{Power supply:} Has internal resistance (limits current)
    \item \textbf{Wires:} Have small resistance
    \item \textbf{Capacitor:} Has ESR (equivalent series resistance)
    \item These resistances prevent infinite current
    \item But still can have very high inrush current
\end{itemize}

\textbf{Solution:}
\begin{itemize}
    \item Add series resistor to limit current
    \item Typical: few ohms to kiloohms (depends on application)
    \item Protects capacitor from excessive current
    \item Prevents damage to power supply
    \item Controls charging rate
\end{itemize}

\textbf{Always check datasheet:}
\begin{itemize}
    \item Maximum voltage rating
    \item Maximum current rating
    \item Maximum charging current
    \item Operating temperature range
\end{itemize}

\vspace{0.15cm}

\textbf{Voltage Drop Across Diode (Forward Bias):}

\textbf{In circuits with diodes:}
\begin{itemize}
    \item Diode conducts when forward-biased
    \item Typical forward voltage drop: ~0.6-0.7V (silicon diode)
    \item Capacitor charges to: $V_{supply} - V_{diode}$
    \item Example: 5V supply $\rightarrow$ 4.3-4.4V on capacitor
\end{itemize}

\vspace{0.15cm}

\textbf{Summary of DC Behavior:}

\textbf{Charging phase:}
\begin{itemize}
    \item Current flows (initially high)
    \item Voltage across capacitor increases
    \item Current decreases as capacitor fills
    \item Stops when $V_C = V_{supply}$
\end{itemize}

\textbf{Fully charged:}
\begin{itemize}
    \item No current flows
    \item Capacitor = open circuit
    \item Voltage remains constant (if no leakage)
    \item Energy stored in electric field
\end{itemize}

\textbf{Discharging:}
\begin{itemize}
    \item Current flows (if path provided)
    \item Opposite direction from charging
    \item Voltage decreases
    \item Energy released to circuit
\end{itemize}

\textbf{Key insight:}
\begin{itemize}
    \item Capacitor blocks steady DC
    \item But passes changing voltage (transients)
    \item This property crucial for AC circuits (next topic)
\end{itemize}
\end{detailbox}

\vspace{0.2cm}

\noindent\textbf{\color{accentcolor} Practical Example \& Numerical}
\begin{examplebox}
\textbf{Example 1: Simple Charging Circuit}

\textit{Circuit:}
\begin{itemize}
    \item 5V battery
    \item 200 $\mu$F capacitor (initially discharged)
    \item 1 k$\Omega$ resistor in series
\end{itemize}

\textbf{Initial condition (t = 0):}
\begin{itemize}
    \item $V_C = 0V$ (capacitor voltage)
    \item Voltage across resistor: $V_R = 5V - 0V = 5V$
    \item Initial current: $I_0 = V_R / R = 5 / 1000 = 5$ mA
\end{itemize}

\textbf{Fully charged (t = $\infty$):}
\begin{itemize}
    \item $V_C = 5V$
    \item Voltage across resistor: $V_R = 5V - 5V = 0V$
    \item Final current: $I_{\infty} = 0$ A (no current)
\end{itemize}

\textbf{Charging follows exponential:}
\begin{equation*}
    V_C(t) = V_{supply}(1 - e^{-t/RC})
\end{equation*}

(RC time constant covered in next chapter)

\vspace{0.2cm}

\textbf{Example 2: Effect of Increasing Supply Voltage}

\textit{Same circuit, increase voltage to 10V:}

\textbf{Initially:}
\begin{itemize}
    \item Capacitor still at 5V (from before)
    \item Voltage across resistor: $10V - 5V = 5V$
    \item Current: $I = 5V / 1k\Omega = 5$ mA
\end{itemize}

\textbf{Capacitor charges again:}
\begin{itemize}
    \item Current flows until $V_C = 10V$
    \item Charges from 5V to 10V
    \item Then current stops again
\end{itemize}

\textbf{Key point:} Capacitor adjusts to new voltage level.

\vspace{0.2cm}

\textbf{Example 3: Discharge Through LED}

\textit{Capacitor charged to 5V, then connected to LED:}

\textbf{Initial discharge:}
\begin{itemize}
    \item LED forward voltage: ~2V
    \item Voltage available for current: $5V - 2V = 3V$
    \item Assume LED resistance ~50$\Omega$ (simplified)
    \item Initial current: $I = 3V / 50\Omega = 60$ mA
\end{itemize}

\textbf{As capacitor discharges:}
\begin{itemize}
    \item Capacitor voltage drops
    \item Current through LED decreases
    \item LED brightness dims
    \item When $V_C < 2V$, LED turns off
\end{itemize}

\vspace{0.2cm}

\textbf{Example 4: Current Limiting Resistor Importance}

\textit{Without current limiting:}

\textbf{Ideal scenario (simulation):}
\begin{itemize}
    \item No resistor in series
    \item Simulator error: "Infinite current"
    \item Cannot compute
\end{itemize}

\textbf{Real world (still dangerous):}
\begin{itemize}
    \item Wire resistance: 0.1$\Omega$
    \item Supply internal resistance: 0.5$\Omega$
    \item Total: 0.6$\Omega$
    \item Initial current: $I = 5V / 0.6\Omega = 8.3$ A!
    \item Very high inrush current
    \item Can damage components
\end{itemize}

\textbf{With 10$\Omega$ resistor:}
\begin{itemize}
    \item Total resistance: ~10.6$\Omega$
    \item Initial current: $I = 5V / 10.6\Omega = 0.47$ A
    \item Much safer!
\end{itemize}

\vspace{0.2cm}

\textbf{Example 5: Capacitor Voltage vs. Time}

\textit{Track voltage during charging:}

\textbf{Circuit: 5V, 100$\mu$F, 1k$\Omega$}
\begin{itemize}
    \item Time constant: $\tau = RC = 1000 \times 100 \times 10^{-6} = 0.1$ s
\end{itemize}

\textbf{Voltage at various times:}
\begin{align*}
    t = 0: \quad V_C &= 0V \\
    t = \tau: \quad V_C &= 5(1 - e^{-1}) = 3.16V \text{ (63\%)} \\
    t = 2\tau: \quad V_C &= 5(1 - e^{-2}) = 4.32V \text{ (86\%)} \\
    t = 3\tau: \quad V_C &= 5(1 - e^{-3}) = 4.75V \text{ (95\%)} \\
    t = 5\tau: \quad V_C &= 5(1 - e^{-5}) = 4.97V \text{ (99\%)}
\end{align*}

"Fully charged" typically means 5$\tau$ (99% of final value).

\vspace{0.2cm}

\textbf{Example 6: Polarity in DC Circuit}

\textit{Electrolytic capacitor in circuit:}

\textbf{Correct connection:}
\begin{itemize}
    \item Positive plate to +5V rail
    \item Negative plate to ground (0V)
    \item Voltage across: +5V (anode higher than cathode)
    \item Works perfectly
\end{itemize}

\textbf{Incorrect (reversed):}
\begin{itemize}
    \item Positive plate to ground
    \item Negative plate to +5V
    \item Reverse voltage applied
    \item Capacitor fails (pops)
    \item Becomes short circuit
    \item \textbf{NEVER DO THIS!}
\end{itemize}
\end{examplebox}

\vspace{0.2cm}

\noindent\textbf{\color{accentcolor} Key Points (Interview Focus)}
\begin{keypointsbox}
\begin{enumerate}
    \item \textbf{DC blocked:} Fully charged capacitor blocks DC (acts as open circuit)
    \item \textbf{Charging:} Current flows until $V_C = V_{supply}$
    \item \textbf{Current direction:} Charging and discharging currents flow opposite directions
    \item \textbf{Resistor required:} Limits charging current (prevents damage/infinite current)
    \item \textbf{Dielectric blocks:} Current cannot physically flow through insulator
    \item \textbf{Energy storage:} Charges accumulate on plates, stored in electric field
    \item \textbf{Discharging:} Releases stored energy when path provided
    \item \textbf{Capacitance matters:} Larger C = more charge = longer time to charge
\end{enumerate}

\textbf{Interview Questions:}
\begin{itemize}
    \item \textbf{Q:} What happens when capacitor fully charged in DC circuit? \\
    \textit{A:} Current stops flowing; capacitor acts as open circuit.
    
    \item \textbf{Q:} Why does current stop? \\
    \textit{A:} Dielectric blocks DC; plates full of charge; voltage across capacitor equals supply voltage.
    
    \item \textbf{Q:} Why need resistor when charging? \\
    \textit{A:} Limits inrush current; prevents infinite current spike at t=0 when $V_C = 0$.
    
    \item \textbf{Q:} What direction does discharge current flow? \\
    \textit{A:} Opposite to charging current; from positive plate to negative plate.
    
    \item \textbf{Q:} Can DC current flow through capacitor? \\
    \textit{A:} No, dielectric is insulator - blocks DC permanently.
    
    \item \textbf{Q:} How is energy stored in capacitor? \\
    \textit{A:} Electric field between charged plates stores energy.
\end{itemize}

\textbf{Charging Process:}
\begin{itemize}
    \item Initial: High current, low voltage
    \item During: Decreasing current, increasing voltage
    \item Final: Zero current, voltage = supply
    \item Time depends on R and C ($\tau$ = RC)
\end{itemize}

\textbf{Discharging Process:}
\begin{itemize}
    \item Requires discharge path (resistor, LED, etc.)
    \item Current flows opposite direction
    \item Voltage decreases exponentially
    \item Energy released to load
\end{itemize}

\textbf{Common Mistakes:}
\begin{itemize}
    \item Thinking DC flows through capacitor (NO!)
    \item Forgetting current-limiting resistor
    \item Reversing polarity on electrolytic caps
    \item Expecting instant charging (it's exponential)
\end{itemize}
\end{keypointsbox}

% --------------------------------------------------------------------
\subsection{Calculating Charge, Voltage, and Current}

\noindent\textbf{\color{accentcolor} TL;DR (The Gist)}
\begin{tldrbox}
\begin{itemize}
    \item \textbf{Charge equation:} $Q = CV$ (charge = capacitance $\times$ voltage)
    \item \textbf{Current equation:} $I = C\frac{dV}{dt}$ (current depends on rate of voltage change)
    \item \textbf{Key insight:} Faster voltage change $\rightarrow$ more current
    \item Steady DC voltage $\rightarrow$ zero current through capacitor
\end{itemize}
\end{tldrbox}

\vspace{0.2cm}

\noindent\textbf{\color{accentcolor} Detailed Explanation}
\begin{detailbox}
\textbf{Fundamental Capacitor Equations:}

\vspace{0.15cm}

\textbf{Charge-Voltage-Capacitance Relationship:}

\textbf{The equation:}
\begin{equation*}
    \boxed{Q = CV}
\end{equation*}

Where:
\begin{itemize}
    \item $Q$ = charge stored (coulombs, C)
    \item $C$ = capacitance (farads, F)
    \item $V$ = voltage across capacitor (volts, V)
\end{itemize}

\textbf{Meaning:}
\begin{itemize}
    \item Charge stored is product of capacitance and voltage
    \item More voltage $\rightarrow$ more charge stored
    \item Larger capacitance $\rightarrow$ more charge for same voltage
    \item Capacitance is constant (fixed value)
\end{itemize}

\textbf{Defining one Farad:}
\begin{itemize}
    \item $1$ Farad stores $1$ Coulomb per $1$ Volt
    \item $1\text{ F} = 1\text{ C/V}$
    \item Very large unit in practice
\end{itemize}

\vspace{0.15cm}

\textbf{Current Through Capacitor:}

\textbf{The derivative equation:}
\begin{equation*}
    \boxed{I = C\frac{dV}{dt}}
\end{equation*}

Where:
\begin{itemize}
    \item $I$ = current through capacitor (A)
    \item $C$ = capacitance (F)
    \item $\frac{dV}{dt}$ = rate of voltage change (V/s)
\end{itemize}

\textbf{Critical understanding:}
\begin{itemize}
    \item Current depends on \textbf{how fast voltage changes}
    \item NOT on the voltage level itself
    \item Faster voltage change $\rightarrow$ more current
    \item No voltage change $\rightarrow$ zero current
\end{itemize}

\vspace{0.15cm}

\textbf{Three Scenarios - Voltage vs. Current:}

\textbf{Scenario 1: Constant DC voltage}

\textit{Voltage behavior:}
\begin{itemize}
    \item Voltage steady at 5V
    \item No change in voltage
    \item $\frac{dV}{dt} = 0$ V/s
\end{itemize}

\textit{Current result:}
\begin{equation*}
    I = C \times 0 = 0\text{ A}
\end{equation*}

\textbf{Zero current when voltage constant!}

This is why fully charged capacitor in DC circuit has no current.

\vspace{0.15cm}

\textbf{Scenario 2: Linearly rising voltage}

\textit{Voltage behavior:}
\begin{itemize}
    \item Voltage rises uniformly from 0V to 5V
    \item Constant rate of change
    \item Example: rises 5V in 2 seconds
    \item $\frac{dV}{dt} = 5/2 = 2.5$ V/s (constant)
\end{itemize}

\textit{Current result:}
\begin{itemize}
    \item For 100$\mu$F capacitor:
\end{itemize}
\begin{equation*}
    I = 100 \times 10^{-6} \times 2.5 = 250\text{ $\mu$A (constant)}
\end{equation*}

\textbf{Constant voltage rise $\rightarrow$ constant current}

\vspace{0.15cm}

\textbf{Scenario 3: Variable voltage change}

\textit{Voltage behavior:}
\begin{itemize}
    \item First: Linear rise (slow)
    \item Then: Swift rise (fast)
    \item Then: Slow rise again
\end{itemize}

\textit{Current result:}
\begin{itemize}
    \item Slow rise: Small current
    \item Swift rise: \textbf{Large current spike}
    \item Slow rise: Small current again
\end{itemize}

\textbf{Key observation:}
\begin{itemize}
    \item Even though voltage higher during slow rise
    \item Current is lower because rate of change is lower
    \item \textbf{Current tracks rate of change, not voltage level}
\end{itemize}

\vspace{0.15cm}

\textbf{Mathematical Insight (Calculus):}

\textbf{Where $\frac{dV}{dt}$ comes from:}
\begin{itemize}
    \item Derivative of voltage with respect to time
    \item Instantaneous rate of change
    \item Slope of voltage vs. time graph
\end{itemize}

\textbf{From charge equation:}
\begin{align*}
    Q &= CV \\
    \frac{dQ}{dt} &= C\frac{dV}{dt} \quad \text{(take derivative)}
\end{align*}

But current is rate of charge flow:
\begin{equation*}
    I = \frac{dQ}{dt}
\end{equation*}

Therefore:
\begin{equation*}
    I = C\frac{dV}{dt}
\end{equation*}

\vspace{0.15cm}

\textbf{Why This Equation Matters:}

\textbf{Understanding capacitor behavior:}
\begin{itemize}
    \item Explains why DC is blocked (no voltage change $\rightarrow$ no current)
    \item Explains why AC passes (continuous voltage change $\rightarrow$ continuous current)
    \item Predicts current for any voltage waveform
    \item Foundation for impedance and reactance
\end{itemize}

\textbf{Practical implications:}
\begin{itemize}
    \item Sudden voltage change $\rightarrow$ large current spike
    \item Slow voltage ramp $\rightarrow$ small, controlled current
    \item Design consideration for power supply turn-on
\end{itemize}

\vspace{0.15cm}

\textbf{When This Equation is NOT Enough:}

\textbf{Limitations of $I = C\frac{dV}{dt}$:}
\begin{itemize}
    \item Only applies for linear voltage changes
    \item For AC sinusoidal voltage, need reactance ($X_C$)
    \item For exponential RC charging, use time constant ($\tau$ = RC)
    \item For real circuits, account for ESR
\end{itemize}

\textbf{For AC circuits:}
\begin{itemize}
    \item Use capacitive reactance: $X_C = \frac{1}{2\pi fC}$
    \item Calculate RMS current from RMS voltage
    \item Account for 90° phase shift
    \item (Covered in upcoming topics)
\end{itemize}

\vspace{0.15cm}

\textbf{Energy Stored in Capacitor:}

\textbf{Energy equation:}
\begin{equation*}
    \boxed{E = \frac{1}{2}CV^2}
\end{equation*}

Where:
\begin{itemize}
    \item $E$ = energy (joules, J)
    \item $C$ = capacitance (F)
    \item $V$ = voltage (V)
\end{itemize}

\textbf{Insights:}
\begin{itemize}
    \item Energy proportional to $V^2$ (doubling voltage = 4$\times$ energy)
    \item Energy proportional to $C$ (doubling C = 2$\times$ energy)
    \item All energy stored in electric field between plates
\end{itemize}

\vspace{0.15cm}

\textbf{Summary of Key Equations:}

\begin{center}
\begin{tabular}{|l|l|}
\hline
\textbf{Equation} & \textbf{Meaning} \\
\hline
$Q = CV$ & Charge stored \\
$I = C\frac{dV}{dt}$ & Instantaneous current \\
$E = \frac{1}{2}CV^2$ & Energy stored \\
\hline
\end{tabular}
\end{center}
\end{detailbox}

\vspace{0.2cm}

\noindent\textbf{\color{accentcolor} Practical Example \& Numerical}
\begin{examplebox}
\textbf{Example 1: Charge Stored Calculation}

\textit{Given:}
\begin{itemize}
    \item Capacitor: 220 $\mu$F
    \item Voltage: 12V
\end{itemize}

\textbf{Calculate charge:}
\begin{align*}
    Q &= CV \\
    &= 220 \times 10^{-6} \times 12 \\
    &= 2.64 \times 10^{-3}\text{ C} \\
    &= \boxed{2.64\text{ mC (millicoulombs)}}
\end{align*}

\vspace{0.2cm}

\textbf{Example 2: Current During Linear Voltage Rise}

\textit{Scenario:}
\begin{itemize}
    \item 100 $\mu$F capacitor
    \item Voltage rises from 0V to 10V in 0.5 seconds
    \item Linear (uniform) rise
\end{itemize}

\textbf{Rate of voltage change:}
\begin{equation*}
    \frac{dV}{dt} = \frac{10 - 0}{0.5} = 20\text{ V/s}
\end{equation*}

\textbf{Current:}
\begin{align*}
    I &= C\frac{dV}{dt} \\
    &= 100 \times 10^{-6} \times 20 \\
    &= 2 \times 10^{-3}\text{ A} \\
    &= \boxed{2\text{ mA}}
\end{align*}

Current remains constant at 2 mA during entire 0.5-second rise.

\vspace{0.2cm}

\textbf{Example 3: Faster Voltage Change $\rightarrow$ More Current}

\textit{Same capacitor (100 $\mu$F), same voltage range (0-10V):}

\textbf{Case A: Rise in 0.5 seconds}
\begin{itemize}
    \item $\frac{dV}{dt} = 20$ V/s
    \item $I = 2$ mA
\end{itemize}

\textbf{Case B: Rise in 0.1 seconds (5$\times$ faster)}
\begin{itemize}
    \item $\frac{dV}{dt} = 10/0.1 = 100$ V/s
    \item $I = 100 \times 10^{-6} \times 100 = 10$ mA
\end{itemize}

\textbf{Result:} 5$\times$ faster rise = 5$\times$ more current

\vspace{0.2cm}

\textbf{Example 4: Energy Stored}

\textit{Calculate energy in 1,000 $\mu$F capacitor at different voltages:}

\textbf{At 5V:}
\begin{equation*}
    E = \frac{1}{2} \times 1000 \times 10^{-6} \times 5^2 = 0.0125\text{ J}
\end{equation*}

\textbf{At 10V:}
\begin{equation*}
    E = \frac{1}{2} \times 1000 \times 10^{-6} \times 10^2 = 0.05\text{ J}
\end{equation*}

\textbf{At 20V:}
\begin{equation*}
    E = \frac{1}{2} \times 1000 \times 10^{-6} \times 20^2 = 0.2\text{ J}
\end{equation*}

\textbf{Observation:}
\begin{itemize}
    \item Doubling voltage: $5V \to 10V$ increases energy 4$\times$
    \item Doubling again: $10V \to 20V$ increases energy 4$\times$ more
    \item Energy scales with $V^2$
\end{itemize}

\vspace{0.2cm}

\textbf{Example 5: Why Steady DC = Zero Current}

\textit{DC voltage applied to capacitor:}

\textbf{After fully charged:}
\begin{itemize}
    \item Voltage constant at 12V
    \item No change in voltage
    \item $\frac{dV}{dt} = 0$ V/s
\end{itemize}

\textbf{Current:}
\begin{equation*}
    I = C \times 0 = 0\text{ A}
\end{equation*}

This is mathematical proof that DC is blocked!

\vspace{0.2cm}

\textbf{Example 6: Sudden Voltage Step}

\textit{Ideal voltage step from 0V to 5V instantly:}

\textbf{Theoretical:}
\begin{itemize}
    \item Voltage changes from 0 to 5 in zero time
    \item $\frac{dV}{dt} = 5/0 = \infty$ V/s
    \item Current: $I = C \times \infty = \infty$ A
\end{itemize}

\textbf{Reality:}
\begin{itemize}
    \item Nothing changes instantly
    \item Always some resistance (ESR, wires, source)
    \item Limits current to finite value
    \item But can still be very large inrush current
\end{itemize}

This is why we need current-limiting resistors!

\vspace{0.2cm}

\textbf{Example 7: Capacitance from Charge and Voltage}

\textit{Measurement:}
\begin{itemize}
    \item Capacitor charged to 15V
    \item Stores 3 mC of charge
    \item What is capacitance?
\end{itemize}

\textbf{Rearrange $Q = CV$:}
\begin{align*}
    C &= \frac{Q}{V} \\
    &= \frac{3 \times 10^{-3}}{15} \\
    &= 0.2 \times 10^{-3}\text{ F} \\
    &= \boxed{200\text{ $\mu$F}}
\end{align*}
\end{examplebox}

\vspace{0.2cm}

\noindent\textbf{\color{accentcolor} Key Points (Interview Focus)}
\begin{keypointsbox}
\begin{enumerate}
    \item \textbf{Charge equation:} $Q = CV$ (charge proportional to voltage)
    \item \textbf{Current equation:} $I = C\frac{dV}{dt}$ (current proportional to rate of voltage change)
    \item \textbf{Steady voltage:} $\frac{dV}{dt} = 0 \Rightarrow I = 0$ (DC blocked)
    \item \textbf{Fast voltage change:} Large $\frac{dV}{dt} \Rightarrow$ large current
    \item \textbf{Energy stored:} $E = \frac{1}{2}CV^2$ (proportional to $V^2$)
    \item \textbf{One Farad:} Stores 1 Coulomb per 1 Volt
    \item \textbf{Current tracks:} Rate of change, NOT voltage level
    \item \textbf{Derivative:} $\frac{dV}{dt}$ is instantaneous slope of voltage vs. time
\end{enumerate}

\textbf{Interview Questions:}
\begin{itemize}
    \item \textbf{Q:} How much charge in 100$\mu$F capacitor at 10V? \\
    \textit{A:} $Q = CV = 100 \times 10^{-6} \times 10 = 1$ mC.
    
    \item \textbf{Q:} Why does DC not flow through capacitor? \\
    \textit{A:} DC is constant voltage; $\frac{dV}{dt} = 0$; therefore $I = C \times 0 = 0$.
    
    \item \textbf{Q:} What determines current through capacitor? \\
    \textit{A:} Rate of voltage change ($\frac{dV}{dt}$), not voltage level.
    
    \item \textbf{Q:} If voltage changes twice as fast, what happens to current? \\
    \textit{A:} Current doubles (directly proportional).
    
    \item \textbf{Q:} Energy stored in 1F cap at 2.5V? \\
    \textit{A:} $E = \frac{1}{2} \times 1 \times 2.5^2 = 3.125$ J.
    
    \item \textbf{Q:} Define one Farad. \\
    \textit{A:} Capacitance that stores 1 Coulomb of charge per 1 Volt.
\end{itemize}

\textbf{Key Formulas:}
\begin{itemize}
    \item Charge: $Q = CV$
    \item Current: $I = C\frac{dV}{dt}$
    \item Energy: $E = \frac{1}{2}CV^2$
    \item Rearranged: $C = Q/V$, $V = Q/C$
\end{itemize}

\textbf{Practical Insights:}
\begin{itemize}
    \item Sudden voltage change $\rightarrow$ large current spike (inrush)
    \item Slow voltage ramp $\rightarrow$ small, manageable current
    \item Constant DC $\rightarrow$ zero current (capacitor charged)
    \item AC sinusoid $\rightarrow$ continuous current (always changing)
\end{itemize}

\textbf{Common Mistakes:}
\begin{itemize}
    \item Thinking current depends on voltage level (it depends on rate of change!)
    \item Forgetting $\frac{dV}{dt} = 0$ for DC
    \item Not using correct equation for AC (need reactance formula)
    \item Confusing charge (Q) with capacitance (C)
\end{itemize}
\end{keypointsbox}

% --------------------------------------------------------------------
\subsection{Capacitor in an AC Circuit}

\noindent\textbf{\color{accentcolor} TL;DR (The Gist)}
\begin{tldrbox}
\begin{itemize}
    \item \textbf{AC passes:} Capacitor allows AC current (constantly charging/discharging)
    \item \textbf{90° phase shift:} Current leads voltage by 90° in purely capacitive circuit
    \item \textbf{Why phase shift:} Current maximum when voltage changing fastest (at zero crossing)
    \item DC blocked, AC passed - fundamental capacitor property
\end{itemize}
\end{tldrbox}

\vspace{0.2cm}

\noindent\textbf{\color{accentcolor} Detailed Explanation}
\begin{detailbox}
\textbf{Capacitor Behavior with AC:}

\vspace{0.15cm}

\textbf{Recap of DC behavior:}
\begin{itemize}
    \item DC: Capacitor charges to supply voltage, then current stops
    \item Voltage constant $\rightarrow$ $\frac{dV}{dt} = 0$ $\rightarrow$ no current
    \item Capacitor acts as open circuit for DC
\end{itemize}

\textbf{AC changes everything:}
\begin{itemize}
    \item AC voltage constantly changing
    \item Never reaches steady state
    \item Capacitor continuously charging and discharging
    \item \textbf{Current flows continuously!}
\end{itemize}

\vspace{0.15cm}

\textbf{How AC Passes Through Capacitor:}

\textbf{The process:}
\begin{enumerate}
    \item AC voltage oscillates: positive $\rightarrow$ zero $\rightarrow$ negative $\rightarrow$ zero (repeat)
    \item When voltage rising: Capacitor charges (current flows in)
    \item When voltage falling: Capacitor discharges (current flows out)
    \item Continuous back-and-forth current flow
    \item Direction reverses every half cycle
\end{enumerate}

\textbf{Key insight:}
\begin{itemize}
    \item Voltage always changing $\rightarrow$ $\frac{dV}{dt} \neq 0$ $\rightarrow$ current exists
    \item This is why capacitor "passes" AC
    \item Actually blocking DC but allowing AC
\end{itemize}

\vspace{0.15cm}

\textbf{Phase Relationship: Current Leads Voltage by 90°:}

\textbf{In purely resistive circuit:}
\begin{itemize}
    \item Voltage and current \textbf{in phase}
    \item Both peak at same time
    \item Both zero at same time
    \item Rise and fall together
    \item Ohm's Law: $I = V/R$ applies instantaneously
\end{itemize}

\textbf{In purely capacitive circuit:}
\begin{itemize}
    \item Voltage and current \textbf{NOT in phase}
    \item \textbf{90° phase difference}
    \item Current leads voltage by 90°
    \item When current peaks, voltage is zero
    \item When voltage peaks, current is zero
\end{itemize}

\vspace{0.15cm}

\textbf{WHY 90° Phase Shift? (Critical Understanding):}

\textbf{Remember:} $I = C\frac{dV}{dt}$ (current proportional to rate of voltage change)

\textbf{Analyzing sinusoidal AC voltage:}

\textit{When voltage is at peak (maximum):}
\begin{itemize}
    \item Voltage at top of sine wave
    \item Momentarily not changing (slope = 0)
    \item $\frac{dV}{dt} = 0$ at this instant
    \item Therefore: $I = C \times 0 = 0$ (current is zero!)
\end{itemize}

\textit{When voltage is zero (crossing):}
\begin{itemize}
    \item Voltage crossing through zero
    \item This is where voltage changing \textbf{fastest}
    \item Steepest slope on sine wave
    \item $\frac{dV}{dt}$ is maximum
    \item Therefore: $I = C \times (\text{max})$ $\rightarrow$ current is maximum!
\end{itemize}

\textbf{Result:}
\begin{itemize}
    \item Current peaks when voltage crosses zero
    \item Current zero when voltage at peak
    \item This creates 90° phase shift
    \item Current "leads" voltage (peaks first)
\end{itemize}

\vspace{0.15cm}

\textbf{Visualizing the Phase Shift:}

\textbf{Phasor diagram interpretation:}

\textit{Voltage waveform (sine wave):}
\begin{itemize}
    \item Starts at 0°, rises to peak at 90°
    \item Returns to zero at 180°
    \item Negative peak at 270°
    \item Back to zero at 360° (one cycle complete)
\end{itemize}

\textit{Current waveform (leading by 90°):}
\begin{itemize}
    \item Peaks at 0° (when voltage is zero)
    \item Zero at 90° (when voltage peaks)
    \item Negative peak at 180° (when voltage at zero)
    \item Zero at 270° (when voltage at negative peak)
\end{itemize}

\textbf{Memory aid:}
\begin{itemize}
    \item "ICE" mnemonic: In Capacitors, current (I) leads voltage (E)
    \item Current gets to peak before voltage does
\end{itemize}

\vspace{0.15cm}

\textbf{Comparison: Resistor vs. Capacitor:}

\begin{center}
\begin{tabular}{|l|l|l|}
\hline
\textbf{Property} & \textbf{Resistor} & \textbf{Capacitor} \\
\hline
Phase shift & 0° (in phase) & 90° (I leads V) \\
DC behavior & Passes (constant I) & Blocks (I = 0) \\
AC behavior & Passes (I $\propto$ V) & Passes (I $\propto$ dV/dt) \\
Relation & $I = V/R$ & $I = C\frac{dV}{dt}$ \\
\hline
\end{tabular}
\end{center}

\vspace{0.15cm}

\textbf{Practical Observation in Oscilloscope:}

\textbf{Yellow trace (current):}
\begin{itemize}
    \item Peaks first
    \item Crosses zero before voltage
    \item Leads by 90°
\end{itemize}

\textbf{Green trace (voltage):}
\begin{itemize}
    \item Lags behind current
    \item Peaks 90° after current peaks
    \item Crosses zero 90° after current
\end{itemize}

\vspace{0.15cm}

\textbf{Why This Matters:}

\textbf{Circuit design implications:}
\begin{itemize}
    \item Phase shift affects power factor
    \item Important in AC power systems
    \item Affects filter design (high-pass, low-pass)
    \item Phase relationships crucial in signal processing
\end{itemize}

\textbf{Key applications:}
\begin{itemize}
    \item \textbf{DC blocking:} Couples AC signals, blocks DC offset
    \item \textbf{AC coupling:} Passes audio/RF signals, blocks DC
    \item \textbf{Filtering:} Frequency-dependent behavior (covered next topics)
    \item \textbf{Phase shifting:} Intentional 90° shift in some circuits
\end{itemize}

\vspace{0.15cm}

\textbf{Summary - Capacitor with AC:}

\textbf{Behavior:}
\begin{itemize}
    \item Allows AC to pass (continuous charging/discharging)
    \item Current never stops (voltage always changing)
    \item Acts like conductor for AC (with frequency-dependent "resistance")
\end{itemize}

\textbf{Phase relationship:}
\begin{itemize}
    \item 90° phase shift between V and I
    \item Current leads voltage
    \item Due to $I = C\frac{dV}{dt}$ relationship
\end{itemize}

\textbf{Big takeaway:}
\begin{itemize}
    \item \textbf{Capacitors block DC and allow AC}
    \item Fundamental property exploited in countless applications
    \item Phase shift intrinsic to capacitive behavior
\end{itemize}
\end{detailbox}

\vspace{0.2cm}

\noindent\textbf{\color{accentcolor} Practical Example \& Numerical}
\begin{examplebox}
\textbf{Example 1: Phase Shift Observation}

\textit{Circuit with 1 $\mu$F capacitor, 1 kHz AC source:}

\textbf{Voltage waveform:}
\begin{itemize}
    \item Sinusoidal, 5V peak
    \item Frequency: 1 kHz
    \item Period: T = 1 ms
\end{itemize}

\textbf{Current waveform:}
\begin{itemize}
    \item Also sinusoidal
    \item Same frequency (1 kHz)
    \item But peaks 0.25 ms \textbf{earlier} than voltage
\end{itemize}

\textbf{Phase calculation:}
\begin{align*}
    \text{Phase shift} &= \frac{0.25\text{ ms}}{1\text{ ms}} \times 360° \\
    &= 0.25 \times 360° \\
    &= \boxed{90°}
\end{align*}

Current leads voltage by exactly 90°!

\vspace{0.2cm}

\textbf{Example 2: Why Current Maximum at Voltage Zero}

\textit{AC voltage: $V(t) = 10\sin(2\pi \times 60t)$ (60 Hz, 10V peak)}

\textbf{Rate of voltage change:}
\begin{equation*}
    \frac{dV}{dt} = 10 \times 2\pi \times 60 \times \cos(2\pi \times 60t)
\end{equation*}

\textbf{At t = 0 (voltage = 0):}
\begin{align*}
    V(0) &= 10\sin(0) = 0\text{ V} \\
    \frac{dV}{dt}\bigg|_{t=0} &= 10 \times 2\pi \times 60 \times \cos(0) \\
    &= 3{,}770\text{ V/s (maximum rate of change!)}
\end{align*}

\textbf{At peak (t = T/4, voltage = 10V):}
\begin{align*}
    V(T/4) &= 10\sin(90°) = 10\text{ V (maximum)} \\
    \frac{dV}{dt}\bigg|_{t=T/4} &= 10 \times 2\pi \times 60 \times \cos(90°) \\
    &= 0\text{ V/s (no change!)}
\end{align*}

This proves current maximum when voltage zero, and vice versa!

\vspace{0.2cm}

\textbf{Example 3: DC vs. AC Through Capacitor}

\textit{Same circuit, different sources:}

\textbf{DC source (5V constant):}
\begin{itemize}
    \item Initial: Current flows (capacitor charging)
    \item After ~5$\tau$: Current = 0 (fully charged)
    \item Steady state: Capacitor = open circuit
    \item DC blocked $\checkmark$
\end{itemize}

\textbf{AC source (5V RMS, 60 Hz):}
\begin{itemize}
    \item Continuous current flow
    \item Current oscillates at 60 Hz
    \item Never stops (voltage always changing)
    \item AC passes $\checkmark$
\end{itemize}

\vspace{0.2cm}

\textbf{Example 4: Audio Signal Coupling}

\textit{Microphone output:}
\begin{itemize}
    \item AC audio signal: $\pm$0.1V at various frequencies (20 Hz - 20 kHz)
    \item DC offset: +2.5V (bias voltage from microphone)
    \item Total signal: 2.5V DC + AC audio
\end{itemize}

\textbf{Coupling capacitor (1 $\mu$F):}
\begin{itemize}
    \item Blocks 2.5V DC component
    \item Passes AC audio component
    \item Output: Pure AC audio ($\pm$0.1V, no DC)
\end{itemize}

\textbf{Result:} Clean audio signal for amplifier!

\vspace{0.2cm}

\textbf{Example 5: Frequency Effect (Preview)}

\textit{Same capacitor (10 $\mu$F), different frequencies:}

\textbf{Low frequency (10 Hz):}
\begin{itemize}
    \item Voltage changes slowly
    \item $\frac{dV}{dt}$ is small
    \item Current is small
    \item Capacitor offers high "resistance" to low freq
\end{itemize}

\textbf{High frequency (10 kHz):}
\begin{itemize}
    \item Voltage changes rapidly
    \item $\frac{dV}{dt}$ is large
    \item Current is large
    \item Capacitor offers low "resistance" to high freq
\end{itemize}

This frequency-dependent behavior leads to reactance (next topic)!
\end{examplebox}

\vspace{0.2cm}

\noindent\textbf{\color{accentcolor} Key Points (Interview Focus)}
\begin{keypointsbox}
\begin{enumerate}
    \item \textbf{AC passes:} Capacitor allows AC current (continuous charging/discharging)
    \item \textbf{DC blocked:} Steady voltage $\rightarrow$ no current (after initial charging)
    \item \textbf{90° phase shift:} Current leads voltage by 90° in pure capacitive circuit
    \item \textbf{Why phase shift:} $I = C\frac{dV}{dt}$ $\rightarrow$ current max when dV/dt max (at zero crossing)
    \item \textbf{Zero crossing:} Voltage changing fastest $\rightarrow$ current maximum
    \item \textbf{Peak voltage:} Voltage not changing $\rightarrow$ current zero
    \item \textbf{"ICE" mnemonic:} In Capacitors, I (current) leads E (voltage)
    \item \textbf{Fundamental property:} Blocks DC, passes AC
\end{enumerate}

\textbf{Interview Questions:}
\begin{itemize}
    \item \textbf{Q:} Does capacitor pass AC or DC? \\
    \textit{A:} Passes AC, blocks DC.
    
    \item \textbf{Q:} What is phase relationship between V and I in capacitor? \\
    \textit{A:} Current leads voltage by 90°.
    
    \item \textbf{Q:} Why does current lead voltage in capacitor? \\
    \textit{A:} Because $I = C\frac{dV}{dt}$ - current maximum when voltage changing fastest (at zero crossing).
    
    \item \textbf{Q:} When is current maximum in AC capacitor circuit? \\
    \textit{A:} When voltage crosses zero (voltage changing fastest).
    
    \item \textbf{Q:} When is current zero? \\
    \textit{A:} When voltage at peak (not changing, dV/dt = 0).
    
    \item \textbf{Q:} Why does capacitor allow AC but block DC? \\
    \textit{A:} AC always changing (dV/dt $\neq$ 0) $\rightarrow$ current flows; DC constant (dV/dt = 0) $\rightarrow$ no current.
\end{itemize}

\textbf{Phase Shift Memory:}
\begin{itemize}
    \item "ICE": In Capacitor, current (I) leads voltage (E)
    \item Think: Current is "ahead" by 90°
    \item Current peaks first, then voltage peaks
    \item Opposite of inductor (ELI: in inductor, E leads I)
\end{itemize}

\textbf{Applications:}
\begin{itemize}
    \item DC blocking/AC coupling (audio, RF)
    \item Signal isolation (remove DC offset)
    \item Filter circuits (frequency-dependent)
    \item Phase shifting networks
\end{itemize}

\textbf{Common Mistakes:}
\begin{itemize}
    \item Thinking capacitor "conducts" AC (it alternately charges/discharges)
    \item Confusing phase lead with phase lag
    \item Forgetting 90° applies to pure capacitive circuit only
\end{itemize}
\end{keypointsbox}

% --------------------------------------------------------------------
\subsection{Impedance and Reactance of a Capacitor}

\noindent\textbf{\color{accentcolor} TL;DR (The Gist)}
\begin{tldrbox}
\begin{itemize}
    \item \textbf{Capacitive reactance:} $X_C = \frac{1}{2\pi fC}$ (opposition to AC, measured in $\Omega$)
    \item \textbf{Higher frequency:} Lower reactance (passes easier)
    \item \textbf{DC (f=0):} Infinite reactance (blocks completely)
    \item \textbf{Impedance:} $Z = \sqrt{R^2 + X_C^2}$ (total opposition in circuit)
\end{itemize}
\end{tldrbox}

\vspace{0.2cm}

\noindent\textbf{\color{accentcolor} Detailed Explanation}
\begin{detailbox}
\textbf{Capacitive Reactance ($X_C$):}

\vspace{0.15cm}

\textbf{Definition:}
\begin{itemize}
    \item \textbf{Capacitive reactance:} Measure of capacitor's opposition to AC current
    \item Symbol: $X_C$
    \item Unit: Ohms ($\Omega$), like resistance
    \item But NOT the same as resistance!
\end{itemize}

\textbf{Key difference from resistance:}
\begin{itemize}
    \item \textbf{Resistance:} Fixed value (e.g., 100$\Omega$, 1k$\Omega$)
    \item \textbf{Reactance:} Varies with frequency
    \item Higher frequency $\rightarrow$ lower reactance
    \item Lower frequency $\rightarrow$ higher reactance
\end{itemize}

\vspace{0.15cm}

\textbf{Capacitive Reactance Formula:}

\begin{equation*}
    \boxed{X_C = \frac{1}{2\pi fC}}
\end{equation*}

Where:
\begin{itemize}
    \item $X_C$ = capacitive reactance ($\Omega$)
    \item $\pi$ = 3.14159...
    \item $f$ = frequency (Hz)
    \item $C$ = capacitance (F)
\end{itemize}

\vspace{0.15cm}

\textbf{Understanding the Formula:}

\textbf{Inversely proportional to frequency:}
\begin{itemize}
    \item As $f$ increases $\rightarrow$ $X_C$ decreases
    \item Higher frequency $\rightarrow$ less opposition $\rightarrow$ more current
    \item Lower frequency $\rightarrow$ more opposition $\rightarrow$ less current
    \item Capacitor "prefers" high frequencies
\end{itemize}

\textbf{Inversely proportional to capacitance:}
\begin{itemize}
    \item As $C$ increases $\rightarrow$ $X_C$ decreases
    \item Larger capacitor $\rightarrow$ less opposition
    \item Smaller capacitor $\rightarrow$ more opposition
\end{itemize}

\vspace{0.15cm}

\textbf{Extreme Cases:}

\textbf{When frequency = 0 (DC):}
\begin{align*}
    X_C &= \frac{1}{2\pi \times 0 \times C} \\
    &= \frac{1}{0} \\
    &= \infty \text{ (infinite reactance)}
\end{align*}

\textbf{Capacitor acts as open circuit for DC!} This proves mathematically why DC is blocked.

\textbf{When frequency = $\infty$ (theoretical):}
\begin{align*}
    X_C &= \frac{1}{2\pi \times \infty \times C} \\
    &= 0\text{ $\Omega$}
\end{align*}

Capacitor acts like a wire (short circuit) at infinitely high frequency.

\vspace{0.15cm}

\textbf{Frequency Response Graph:}

\textbf{Reactance vs. Frequency (log scale):}
\begin{itemize}
    \item At low frequencies: High reactance (steep curve)
    \item At high frequencies: Low reactance (approaches zero)
    \item Inverse relationship (hyperbolic curve)
    \item Never negative (always positive opposition)
\end{itemize}

\vspace{0.15cm}

\textbf{Impedance ($Z$) - Total Opposition:}

\textbf{In circuit with resistor AND capacitor:}

\textbf{Wrong approach:}
\begin{equation*}
    Z \neq R + X_C \quad \text{(Cannot simply add!)}
\end{equation*}

\textbf{Why not?}
\begin{itemize}
    \item Resistance and reactance at 90° to each other
    \item Voltage across R in phase with current
    \item Voltage across C lags current by 90°
    \item Must add as vectors, not scalars
\end{itemize}

\textbf{Correct formula (series R-C circuit):}
\begin{equation*}
    \boxed{Z = \sqrt{R^2 + X_C^2}}
\end{equation*}

This is Pythagorean theorem! Resistance and reactance are perpendicular.

\vspace{0.15cm}

\textbf{Calculating Current with Impedance:}

\textbf{For RMS (AC) values:}
\begin{equation*}
    I_{rms} = \frac{V_{rms}}{Z}
\end{equation*}

Similar to Ohm's Law, but using impedance instead of resistance.

\vspace{0.15cm}

\textbf{Reactance in Circuit Analysis:}

\textbf{Series R-C circuit:}
\begin{enumerate}
    \item Calculate $X_C$ at operating frequency
    \item Calculate total impedance: $Z = \sqrt{R^2 + X_C^2}$
    \item Calculate current: $I = V/Z$
    \item Voltage across R: $V_R = I \times R$
    \item Voltage across C: $V_C = I \times X_C$
\end{enumerate}

\textbf{Note:} $V_R$ and $V_C$ don't simply add to $V_{total}$ (phase shift!)

\vspace{0.15cm}

\textbf{Why Reactance is Important:}

\textbf{Filter design:}
\begin{itemize}
    \item Frequency-dependent behavior creates filters
    \item High-pass filter: Capacitor blocks low freq, passes high freq
    \item $X_C$ determines cutoff frequency
    \item (More in filter chapters)
\end{itemize}

\textbf{Coupling/decoupling:}
\begin{itemize}
    \item Choose C so $X_C$ is low at signal frequencies
    \item Ensures minimal attenuation
    \item Example: Audio coupling needs low $X_C$ at 20 Hz - 20 kHz
\end{itemize}

\textbf{Power calculations:}
\begin{itemize}
    \item Reactance doesn't dissipate power (unlike resistance)
    \item Energy stored and released each cycle
    \item Affects power factor in AC circuits
\end{itemize}

\vspace{0.15cm}

\textbf{Summary Table:}

\begin{center}
\begin{tabular}{|l|l|}
\hline
\textbf{Frequency} & \textbf{Reactance $X_C$} \\
\hline
0 Hz (DC) & $\infty$ (blocks) \\
Low freq & High (impedes) \\
High freq & Low (passes easily) \\
$\infty$ Hz & 0 $\Omega$ (wire) \\
\hline
\end{tabular}
\end{center}
\end{detailbox}

\vspace{0.2cm}

\noindent\textbf{\color{accentcolor} Practical Example \& Numerical}
\begin{examplebox}
\textbf{Example 1: Calculate Reactance at Two Frequencies}

\textit{Given:} 220 nF capacitor

\textbf{At 1 kHz:}
\begin{align*}
    X_C &= \frac{1}{2\pi fC} \\
    &= \frac{1}{2\pi \times 1000 \times 220 \times 10^{-9}} \\
    &= \frac{1}{1.382 \times 10^{-3}} \\
    &= \boxed{723.4\text{ $\Omega$}}
\end{align*}

\textbf{At 20 kHz:}
\begin{align*}
    X_C &= \frac{1}{2\pi \times 20000 \times 220 \times 10^{-9}} \\
    &= \frac{1}{2.764 \times 10^{-2}} \\
    &= \boxed{36.2\text{ $\Omega$}}
\end{align*}

\textbf{Observation:} 20$\times$ frequency $\rightarrow$ 1/20$\times$ reactance (36.2 $\approx$ 723.4/20)

\vspace{0.2cm}

\textbf{Example 2: Impedance Calculation (Series R-C)}

\textit{Circuit:}
\begin{itemize}
    \item Resistor: 200$\Omega$
    \item Capacitor: 10 $\mu$F
    \item Frequency: 80 Hz
    \item Supply: 5V RMS
\end{itemize}

\textbf{Step 1 - Calculate reactance:}
\begin{align*}
    X_C &= \frac{1}{2\pi \times 80 \times 10 \times 10^{-6}} \\
    &= \boxed{198.9\text{ $\Omega$}}
\end{align*}

\textbf{Step 2 - Calculate impedance:}
\begin{align*}
    Z &= \sqrt{R^2 + X_C^2} \\
    &= \sqrt{200^2 + 198.9^2} \\
    &= \sqrt{40000 + 39561} \\
    &= \sqrt{79561} \\
    &= \boxed{282.1\text{ $\Omega$}}
\end{align*}

\textbf{Step 3 - Calculate current:}
\begin{align*}
    I_{rms} &= \frac{V_{rms}}{Z} \\
    &= \frac{5}{282.1} \\
    &= \boxed{17.7\text{ mA}}
\end{align*}

\vspace{0.2cm}

\textbf{Example 3: Why Can't We Just Add R + $X_C$?}

\textit{Using same circuit from Example 2:}

\textbf{Wrong (arithmetic sum):}
\begin{equation*}
    Z_{wrong} = R + X_C = 200 + 198.9 = 398.9\text{ $\Omega$}
\end{equation*}

\textbf{Correct (vector sum):}
\begin{equation*}
    Z_{correct} = \sqrt{200^2 + 198.9^2} = 282.1\text{ $\Omega$}
\end{equation*}

\textbf{Difference:} $398.9 - 282.1 = 116.8$$\Omega$ error (41\% wrong!)

This is why we must use Pythagorean formula.

\vspace{0.2cm}

\textbf{Example 4: Reactance at Audio Frequencies}

\textit{1 $\mu$F coupling capacitor for audio:}

\textbf{At 20 Hz (low audio):}
\begin{equation*}
    X_C = \frac{1}{2\pi \times 20 \times 1 \times 10^{-6}} = 7{,}958\text{ $\Omega$}
\end{equation*}

\textbf{At 1 kHz (mid audio):}
\begin{equation*}
    X_C = \frac{1}{2\pi \times 1000 \times 1 \times 10^{-6}} = 159\text{ $\Omega$}
\end{equation*}

\textbf{At 20 kHz (high audio):}
\begin{equation*}
    X_C = \frac{1}{2\pi \times 20000 \times 1 \times 10^{-6}} = 8\text{ $\Omega$}
\end{equation*}

\textbf{Problem:} High reactance at 20 Hz might attenuate bass!

\textbf{Solution:} Use larger capacitor (e.g., 10 $\mu$F) for lower reactance across entire audio range.

\vspace{0.2cm}

\textbf{Example 5: Choosing Capacitor for Low Reactance}

\textit{Requirement:} Reactance $< 10$$\Omega$ at 60 Hz

\textbf{Rearrange formula to find C:}
\begin{align*}
    X_C &= \frac{1}{2\pi fC} \\
    C &= \frac{1}{2\pi fX_C} \\
    &= \frac{1}{2\pi \times 60 \times 10} \\
    &= 265 \times 10^{-6}\text{ F} \\
    &= \boxed{265\text{ $\mu$F}}
\end{align*}

Need at least 265 $\mu$F capacitor!

\vspace{0.2cm}

\textbf{Example 6: DC Blocking Verification}

\textit{Any capacitor at DC (f = 0):}

\begin{align*}
    X_C &= \frac{1}{2\pi \times 0 \times C} \\
    &= \frac{1}{0} \\
    &= \infty\text{ $\Omega$}
\end{align*}

\textbf{Current (by Ohm's Law):}
\begin{equation*}
    I = \frac{V}{X_C} = \frac{V}{\infty} = 0\text{ A}
\end{equation*}

Mathematical proof that DC is blocked!
\end{examplebox}

\vspace{0.2cm}

\noindent\textbf{\color{accentcolor} Key Points (Interview Focus)}
\begin{keypointsbox}
\begin{enumerate}
    \item \textbf{Reactance formula:} $X_C = \frac{1}{2\pi fC}$ (measured in $\Omega$)
    \item \textbf{Inversely proportional:} Higher $f$ or higher $C$ $\rightarrow$ lower $X_C$
    \item \textbf{DC (f=0):} $X_C = \infty$ $\rightarrow$ capacitor blocks DC
    \item \textbf{High frequency:} $X_C$ approaches 0 $\rightarrow$ capacitor like wire
    \item \textbf{Impedance:} $Z = \sqrt{R^2 + X_C^2}$ (NOT $R + X_C$!)
    \item \textbf{Phase consideration:} R and $X_C$ perpendicular (90°), use Pythagoras
    \item \textbf{Reactance $\neq$ resistance:} Reactance varies with frequency, resistance doesn't
    \item \textbf{No power dissipation:} Reactance stores/releases energy, doesn't dissipate
\end{enumerate}

\textbf{Interview Questions:}
\begin{itemize}
    \item \textbf{Q:} What is capacitive reactance? \\
    \textit{A:} Measure of capacitor's opposition to AC current, measured in ohms.
    
    \item \textbf{Q:} Formula for reactance? \\
    \textit{A:} $X_C = \frac{1}{2\pi fC}$
    
    \item \textbf{Q:} What happens to reactance as frequency increases? \\
    \textit{A:} Reactance decreases (inversely proportional).
    
    \item \textbf{Q:} Reactance at DC (f=0)? \\
    \textit{A:} Infinite ohms (capacitor blocks DC completely).
    
    \item \textbf{Q:} How to calculate impedance of series R-C? \\
    \textit{A:} $Z = \sqrt{R^2 + X_C^2}$ (Pythagorean theorem).
    
    \item \textbf{Q:} Why not just add $R + X_C$? \\
    \textit{A:} They are 90° out of phase; must use vector addition.
\end{itemize}

\textbf{Key Formulas:}
\begin{itemize}
    \item Reactance: $X_C = \frac{1}{2\pi fC}$
    \item Rearranged: $C = \frac{1}{2\pi fX_C}$, $f = \frac{1}{2\pi CX_C}$
    \item Impedance: $Z = \sqrt{R^2 + X_C^2}$ (series R-C)
    \item Current: $I = V/Z$ (RMS values for AC)
\end{itemize}

\textbf{Practical Insights:}
\begin{itemize}
    \item High freq $\rightarrow$ low $X_C$ $\rightarrow$ capacitor passes easily
    \item Low freq $\rightarrow$ high $X_C$ $\rightarrow$ capacitor impedes
    \item Design filters using frequency dependence
    \item Choose C for desired $X_C$ at operating frequency
\end{itemize}

\textbf{Common Mistakes:}
\begin{itemize}
    \item Adding $R + X_C$ arithmetically (must use $\sqrt{R^2 + X_C^2}$!)
    \item Thinking reactance = resistance (different concepts)
    \item Forgetting frequency dependence
    \item Using wrong units (f in Hz, C in Farads)
\end{itemize}
\end{keypointsbox}

% ----------------------------------------------------------------------------------------------------------------
% NOTE: Topics 8-13 are being added to complete Section 08
% Due to extensive similar content focusing on practical applications,
% these remaining topics cover: frequency effects, capacitance variations,
% and practical applications (coupling, decoupling, bypass, smoothing)
% ----------------------------------------------------------------------------------------------------------------

\subsection{Capacitors with Various Frequencies \& Capacitances - Practical Summary}

\noindent\textbf{\color{accentcolor} TL;DR (The Gist)}
\begin{tldrbox}
\begin{itemize}
    \item \textbf{Higher frequency:} Lower reactance $\rightarrow$ more current $\rightarrow$ capacitor passes easily
    \item \textbf{Larger capacitance:} Lower reactance $\rightarrow$ more current  for same frequency
    \item \textbf{Applications:} Coupling (AC pass, DC block), Decoupling (noise removal), Bypass (local power), Smoothing (ripple reduction)
    \item Choose C based on frequency and desired reactance
\end{itemize}
\end{tldrbox}

\vspace{0.2cm}

\noindent\textbf{\color{accentcolor} Key Points (Interview Focus)}
\begin{keypointsbox}
\textbf{Frequency Effects:}
\begin{itemize}
    \item High freq $\rightarrow$ low $X_C$ $\rightarrow$ capacitor like wire
    \item Low freq $\rightarrow$ high $X_C$ $\rightarrow$ capacitor blocks
    \item Design consideration: Choose C for low $X_C$ at signal frequencies
\end{itemize}

\textbf{Capacitance Effects:}
\begin{itemize}
    \item Larger C $\rightarrow$ lower $X_C$ $\rightarrow$ more current
    \item Smaller C $\rightarrow$ higher $X_C$ $\rightarrow$ less current
    \item Power supply: Use large C (mF range)
    \item High-freq decoupling: Use small C (nF-$\mu$F range)
\end{itemize}
\end{keypointsbox}

% --------------------------------------------------------------------
\subsection{Coupling Capacitors}

\noindent\textbf{\color{accentcolor} TL;DR (The Gist)}
\begin{tldrbox}
\begin{itemize}
    \item \textbf{Purpose:} Pass AC signals while blocking DC offset
    \item \textbf{Application:} Connect two circuit stages, remove DC bias
    \item \textbf{Example:} Microphone output (AC audio + DC bias) $\rightarrow$ capacitor $\rightarrow$ amplifier (AC only)
    \item Choose C large enough: Low $X_C$ at lowest signal frequency
\end{itemize}
\end{tldrbox}

\vspace{0.2cm}

\noindent\textbf{\color{accentcolor} Detailed Explanation}
\begin{detailbox}
\textbf{What is Coupling?}

\vspace{0.15cm}

\textbf{Purpose of coupling capacitor:}
\begin{itemize}
    \item Connect (couple) two circuit stages
    \item Pass AC signal from stage 1 to stage 2
    \item Block DC voltage from stage 1
    \item Each stage can have different DC bias
\end{itemize}

\textbf{Classic example - Audio amplifier stages:}
\begin{itemize}
    \item Stage 1: Microphone preamp (DC bias = 2.5V)
    \item Stage 2: Power amplifier (DC bias = 5V)
    \item Coupling capacitor between them
    \item AC audio passes through
    \item DC voltages isolated
\end{itemize}

\vspace{0.15cm}

\textbf{Microphone Circuit Example:}

\textbf{Microphone output signal:}
\begin{itemize}
    \item AC component: Audio signal ($\pm$0.1V, 20 Hz - 20 kHz)
    \item DC component: Bias voltage (+2.5V for powering mic)
    \item Total output: 2.5V DC + AC audio
\end{itemize}

\textbf{Problem without coupling capacitor:}
\begin{itemize}
    \item Amplifier sees 2.5V DC + AC
    \item DC offset affects amplifier bias point
    \item May cause distortion or clipping
    \item DC not needed at amplifier input
\end{itemize}

\textbf{Solution with coupling capacitor:}
\begin{itemize}
    \item Capacitor blocks 2.5V DC
    \item Capacitor passes AC audio
    \item Amplifier sees only AC signal (centered at 0V)
    \item Clean audio amplification
\end{itemize}

\vspace{0.15cm}

\textbf{Choosing Coupling Capacitor Value:}

\textbf{Requirements:}
\begin{enumerate}
    \item Must pass lowest frequency of interest
    \item Reactance should be low at lowest frequency
    \item Rule of thumb: $X_C < \frac{1}{10}$ of load impedance
\end{enumerate}

\textbf{For audio coupling (20 Hz - 20 kHz):}
\begin{itemize}
    \item Critical frequency: 20 Hz (lowest)
    \item If $X_C$ too high at 20 Hz $\rightarrow$ bass attenuation
    \item Typical values: 1$\mu$F - 10$\mu$F for audio
    \item Larger C = better bass response
\end{itemize}

\textbf{Calculation example:}
\begin{itemize}
    \item Want $X_C < 100\Omega$ at 20 Hz
    \item $C = \frac{1}{2\pi fX_C} = \frac{1}{2\pi \times 20 \times 100} \approx 80$ $\mu$F
    \item Use standard value: 100 $\mu$F
\end{itemize}

\vspace{0.15cm}

\textbf{Polarity Considerations:}

\textbf{If using electrolytic capacitor:}
\begin{itemize}
    \item Must observe polarity!
    \item Positive terminal toward higher DC voltage
    \item In microphone example: + toward mic output
    \item Negative toward amplifier input (lower/no DC)
\end{itemize}

\textbf{Non-polarized alternative:}
\begin{itemize}
    \item Film or ceramic capacitor
    \item No polarity concern
    \item But limited to smaller values (typically < 10$\mu$F)
    \item May not provide sufficient coupling at low frequencies
\end{itemize}
\end{detailbox}

\vspace{0.2cm}

\noindent\textbf{\color{accentcolor} Key Points (Interview Focus)}
\begin{keypointsbox}
\begin{enumerate}
    \item \textbf{Coupling:} Passes AC, blocks DC between circuit stages
    \item \textbf{Reactance:} Must be low at lowest signal frequency
    \item \textbf{Audio:} Typically 1-10 $\mu$F for full 20 Hz - 20 kHz range
    \item \textbf{Polarity:} Electrolytic + toward higher DC voltage
    \item \textbf{Purpose:} Isolate DC bias levels while passing signal
\end{enumerate}

\textbf{Interview Questions:}
\begin{itemize}
    \item \textbf{Q:} What does coupling capacitor do? \\
    \textit{A:} Passes AC signal while blocking DC component between stages.
    
    \item \textbf{Q:} Why need coupling in amplifier? \\
    \textit{A:} Each stage may have different DC bias; capacitor isolates DC while passing AC signal.
    
    \item \textbf{Q:} How to choose coupling capacitor value? \\
    \textit{A:} Large enough so reactance is low at lowest signal frequency.
\end{itemize}
\end{keypointsbox}

% --------------------------------------------------------------------
\subsection{Decoupling \& Bypass Capacitors}

\noindent\textbf{\color{accentcolor} TL;DR (The Gist)}
\begin{tldrbox}
\begin{itemize}
    \item \textbf{Decoupling:} Removes AC noise from DC power supply
    \item \textbf{Bypass:} Provides local energy storage for fast current demands
    \item \textbf{Placement:} Across power rails, close to IC
    \item \textbf{Multiple values:} Different capacitors handle different frequencies
\end{itemize}
\end{tldrbox}

\vspace{0.2cm}

\noindent\textbf{\color{accentcolor} Detailed Explanation}
\begin{detailbox}
\textbf{Decoupling Capacitors:}

\vspace{0.15cm}

\textbf{The problem - Noisy DC:}
\begin{itemize}
    \item Real power supplies have AC noise superimposed on DC
    \item Switching circuits create voltage ripples
    \item Long PCB traces have inductance
    \item Sudden current demands cause voltage dips
\end{itemize}

\textbf{The solution:}
\begin{itemize}
    \item Place capacitor in parallel with power supply
    \item Capacitor has low reactance to AC noise
    \item AC noise shunted to ground through capacitor
    \item DC component unaffected (capacitor blocks DC)
    \item Result: Clean DC power to IC
\end{itemize}

\vspace{0.15cm}

\textbf{Bypass Capacitors (Similar but Different Purpose):}

\textbf{The problem - Inductive supply lines:}
\begin{itemize}
    \item PCB traces have inductance
    \item Inductors resist changing current
    \item When IC switches, needs fast current
    \item Inductive supply cannot respond quickly
    \item Voltage dips occur
\end{itemize}

\textbf{The solution:}
\begin{itemize}
    \item Capacitor acts as local energy reservoir
    \item Charged and ready near IC
    \item When IC needs current spike, cap provides it instantly
    \item Bypasses the slow, inductive supply path
    \item Maintains stable voltage at IC
\end{itemize}

\vspace{0.15cm}

\textbf{Why Multiple Capacitor Values in Parallel?}

\textbf{Real capacitor model:}
\begin{itemize}
    \item Not just capacitance
    \item Has ESR (equivalent series resistance)
    \item Has ESL (equivalent series inductance)
    \item Forms RLC circuit
    \item Has resonant frequency
\end{itemize}

\textbf{Frequency response:}
\begin{itemize}
    \item Below resonance: Capacitive (impedance decreases with freq)
    \item At resonance: Minimum impedance
    \item Above resonance: Inductive (impedance increases!)
\end{itemize}

\textbf{Solution - Multiple capacitors:}
\begin{enumerate}
    \item \textbf{Large (1-100 $\mu$F):} Low-frequency noise, bulk energy storage
    \item \textbf{Medium (0.1-1 $\mu$F):} Mid-frequency decoupling
    \item \textbf{Small (10-100 nF):} High-frequency noise
\end{enumerate}

\textbf{Result:}
\begin{itemize}
    \item Each capacitor effective at different frequency range
    \item Combined: Low impedance across wide frequency spectrum
    \item Better overall performance than single value
\end{itemize}

\vspace{0.15cm}

\textbf{Placement Critical:}

\textbf{Rules:}
\begin{itemize}
    \item Place as close as possible to IC power pins
    \item Minimizes inductance in path
    \item Smaller capacitors closer (they handle high freq)
    \item Larger capacitors can be slightly further
    \item Short, wide traces preferred
\end{itemize}
\end{detailbox}

\vspace{0.2cm}

\noindent\textbf{\color{accentcolor} Key Points (Interview Focus)}
\begin{keypointsbox}
\begin{enumerate}
    \item \textbf{Decoupling:} Removes AC noise from DC supply (noise to ground)
    \item \textbf{Bypass:} Provides local fast current (bypasses slow supply)
    \item \textbf{Placement:} As close as possible to IC power pins
    \item \textbf{Multiple values:} Cover wide frequency range (nF to $\mu$F)
    \item \textbf{Typical:} 0.1 $\mu$F ceramic + 10 $\mu$F electrolytic per IC
\end{enumerate}

\textbf{Interview Questions:}
\begin{itemize}
    \item \textbf{Q:} Difference between decoupling and bypass? \\
    \textit{A:} Decoupling removes noise; bypass provides local energy. Same placement, slightly different purpose.
    
    \item \textbf{Q:} Why multiple capacitor values? \\
    \textit{A:} Each effective at different frequency; real caps have resonance; multiple values cover wider range.
    
    \item \textbf{Q:} Where to place bypass capacitor? \\
    \textit{A:} As close as possible to IC power pins to minimize inductance.
\end{itemize}
\end{keypointsbox}

% --------------------------------------------------------------------
\subsection{Smoothing Capacitors}

\noindent\textbf{\color{accentcolor} TL;DR (The Gist)}
\begin{tldrbox}
\begin{itemize}
    \item \textbf{Purpose:} Convert pulsating DC (after rectification) to smooth DC
    \item \textbf{Operation:} Charges during voltage peaks, discharges during dips
    \item \textbf{Value:} Large (100 $\mu$F - 10,000 $\mu$F typical)
    \item \textbf{Limitation:} Provides smoothing but NOT regulation
\end{itemize}
\end{tldrbox}

\vspace{0.2cm}

\noindent\textbf{\color{accentcolor} Detailed Explanation}
\begin{detailbox}
\textbf{Power Supply Context:}

\vspace{0.15cm}

\textbf{AC to DC conversion process:}
\begin{enumerate}
    \item AC from wall outlet (120V/230V, 50/60 Hz)
    \item Transformer steps down voltage
    \item Rectifier converts AC to pulsating DC
    \item \textbf{Smoothing capacitor reduces ripple}
    \item Voltage regulator provides stable DC (optional)
\end{enumerate}

\textbf{After rectification (without smoothing):}
\begin{itemize}
    \item Pulsating DC voltage
    \item Voltage varies from 0V to peak
    \item Ripple frequency = 2$\times$ AC frequency (full-wave)
    \item Not suitable for powering circuits
\end{itemize}

\vspace{0.15cm}

\textbf{How Smoothing Works:}

\textbf{Charging phase (voltage rising):}
\begin{itemize}
    \item Rectifier output rises to peak
    \item Capacitor charges to peak voltage
    \item Capacitor fully charged at peak
\end{itemize}

\textbf{Discharging phase (voltage falling):}
\begin{itemize}
    \item Rectifier output starts falling
    \item Load draws current
    \item Capacitor supplies current from stored energy
    \item Capacitor voltage slowly decreases
    \item "Fills in the gaps" between pulses
\end{itemize}

\textbf{Next cycle:}
\begin{itemize}
    \item Rectifier voltage rises again
    \item Recharges capacitor to peak
    \item Process repeats
\end{itemize}

\textbf{Result:}
\begin{itemize}
    \item Much smoother DC voltage
    \item Small ripple remains
    \item Larger capacitor = smoother output
\end{itemize}

\vspace{0.15cm}

\textbf{Choosing Smoothing Capacitor:}

\textbf{Factors:}
\begin{itemize}
    \item Load current (higher I $\rightarrow$ need larger C)
    \item Acceptable ripple voltage
    \item Ripple frequency (50/60 Hz $\rightarrow$ 100/120 Hz for full-wave)
\end{itemize}

\textbf{Rule of thumb:}
\begin{itemize}
    \item $C \approx \frac{I_{load}}{2fV_{ripple}}$
    \item Where f = ripple frequency, $V_{ripple}$ = acceptable ripple
    \item Typical: 1,000 - 10,000 $\mu$F for 1A load
\end{itemize}

\vspace{0.15cm}

\textbf{Limitations:}

\textbf{Smoothing is NOT regulation:}
\begin{itemize}
    \item Output voltage varies with load
    \item Higher load $\rightarrow$ more ripple, lower average voltage
    \item Lower load $\rightarrow$ less ripple, higher voltage
    \item Input voltage changes affect output
\end{itemize}

\textbf{For regulated supply:}
\begin{itemize}
    \item Add voltage regulator after smoothing cap
    \item Regulator maintains constant output
    \item Capacitor still needed (regulator input requirement)
    \item Combined: smooth AND regulated DC
\end{itemize}
\end{detailbox}

\vspace{0.2cm}

\noindent\textbf{\color{accentcolor} Key Points (Interview Focus)}
\begin{keypointsbox}
\begin{enumerate}
    \item \textbf{Smoothing:} Reduces ripple in rectified DC
    \item \textbf{Operation:} Charges at peaks, discharges during dips
    \item \textbf{Value:} Large (hundreds to thousands of $\mu$F)
    \item \textbf{NOT regulation:} Voltage still varies with load
    \item \textbf{Polarity:} Electrolytic must be connected correctly
    \item \textbf{For regulation:} Add voltage regulator after smoothing
\end{enumerate}

\textbf{Interview Questions:}
\begin{itemize}
    \item \textbf{Q:} What does smoothing capacitor do? \\
    \textit{A:} Reduces ripple voltage in rectified DC by storing/releasing energy.
    
    \item \textbf{Q:} Does smoothing capacitor provide regulation? \\
    \textit{A:} No, only reduces ripple. Voltage still varies with load. Need regulator for stable output.
    
    \item \textbf{Q:} Why large capacitance needed? \\
    \textit{A:} Must supply load current during discharge phase (between AC peaks); larger C = less voltage drop.
\end{itemize}

\textbf{Applications:}
\begin{itemize}
    \item Power supplies (after rectifier)
    \item Battery eliminator circuits
    \item Linear power supplies
    \item Always paired with voltage regulator for quality DC
\end{itemize}
\end{keypointsbox}

