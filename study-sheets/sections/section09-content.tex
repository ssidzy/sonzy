% ====================================================================
% SECTION 09: RC TIME CONSTANT
% ====================================================================

\section*{\LARGE\color{headercolor} Section 09 -- RC Time Constant}
\addcontentsline{toc}{section}{Section 09: RC Time Constant}

% --------------------------------------------------------------------
\subsection{RC Time Constant ($\tau = RC$)}

\noindent\textbf{\color{accentcolor} TL;DR (The Gist)}
\begin{tldrbox}
\begin{itemize}
    \item \textbf{Time Constant:} $\tau = R \times C$ (in seconds)
    \item \textbf{Charging:} After 1$\tau$: 63\% charged | After 5$\tau$: ~99\% charged (full)
    \item \textbf{Discharging:} After 1$\tau$: 37\% remains | After 5$\tau$: ~1\% remains (empty)
    \item \textbf{Units:} R in ohms ($\Omega$), C in farads (F) $\rightarrow$ $\tau$ in seconds (s)
\end{itemize}
\end{tldrbox}

\vspace{0.2cm}

\noindent\textbf{\color{accentcolor} Detailed Explanation}
\begin{detailbox}
\textbf{What is RC Time Constant?}

\vspace{0.15cm}

\textbf{Definition:}
\begin{itemize}
    \item Time constant ($\tau$) = Resistance $\times$ Capacitance
    \item $\tau = R \times C$ where R in $\Omega$, C in F
    \item Measured in seconds (s)
    \item Determines how fast capacitor charges/discharges
    \item Larger $\tau$ = slower charging/discharging
\end{itemize}

\textbf{Why does it matter?}
\begin{itemize}
    \item Capacitor doesn't charge/discharge instantly
    \item Resistor limits current $\rightarrow$ slows charging
    \item Larger R or C $\rightarrow$ longer time to charge
    \item Time constant quantifies this delay
    \item Critical for timing circuits and filters
\end{itemize}

\vspace{0.15cm}

\textbf{The Charging Process:}

\textbf{RC charging circuit:}
\begin{itemize}
    \item Resistor in series with capacitor
    \item DC voltage source applied
    \item Capacitor starts at 0V
    \item Current flows: $I = \frac{V_{supply} - V_C}{R}$
    \item As $V_C$ increases, current decreases
    \item Eventually $V_C = V_{supply}$, current = 0
\end{itemize}

\textbf{Exponential voltage rise:}
\begin{itemize}
    \item $V_C(t) = V_{supply} \times (1 - e^{-t/\tau})$
    \item NOT linear! Exponential curve
    \item Fast rise initially, slows down later
    \item Asymptotically approaches $V_{supply}$
    \item Never truly reaches 100\% (but very close)
\end{itemize}

\textbf{Key time points during charging:}
\begin{enumerate}
    \item \textbf{At $t = 0$:} $V_C = 0$V (0\% charged)
    \item \textbf{At $t = 1\tau$:} $V_C = 0.632 \times V_{supply}$ (63.2\% charged)
    \item \textbf{At $t = 2\tau$:} $V_C = 0.865 \times V_{supply}$ (86.5\% charged)
    \item \textbf{At $t = 3\tau$:} $V_C = 0.950 \times V_{supply}$ (95.0\% charged)
    \item \textbf{At $t = 4\tau$:} $V_C = 0.982 \times V_{supply}$ (98.2\% charged)
    \item \textbf{At $t = 5\tau$:} $V_C = 0.993 \times V_{supply}$ (99.3\% charged)
\end{enumerate}

\textbf{The "5 tau rule":}
\begin{itemize}
    \item After 5$\tau$, capacitor considered "fully charged"
    \item Actually 99.3\%, but close enough for practical purposes
    \item Total charging time $\approx 5 \times R \times C$
\end{itemize}

\vspace{0.15cm}

\textbf{The Discharging Process:}

\textbf{RC discharging circuit:}
\begin{itemize}
    \item Capacitor initially charged to $V_{initial}$
    \item Voltage source removed or switched to ground
    \item Capacitor discharges through resistor
    \item Current flows opposite direction
    \item Current: $I = \frac{V_C}{R}$ (decreases as $V_C$ decreases)
\end{itemize}

\textbf{Exponential voltage decay:}
\begin{itemize}
    \item $V_C(t) = V_{initial} \times e^{-t/\tau}$
    \item Exponential decay curve
    \item Fast drop initially, slows down later
    \item Asymptotically approaches 0V
\end{itemize}

\textbf{Key time points during discharging:}
\begin{enumerate}
    \item \textbf{At $t = 0$:} $V_C = V_{initial}$ (100\% charged)
    \item \textbf{At $t = 1\tau$:} $V_C = 0.368 \times V_{initial}$ (36.8\% remains)
    \item \textbf{At $t = 2\tau$:} $V_C = 0.135 \times V_{initial}$ (13.5\% remains)
    \item \textbf{At $t = 3\tau$:} $V_C = 0.050 \times V_{initial}$ (5.0\% remains)
    \item \textbf{At $t = 4\tau$:} $V_C = 0.018 \times V_{initial}$ (1.8\% remains)
    \item \textbf{At $t = 5\tau$:} $V_C = 0.007 \times V_{initial}$ (0.7\% remains)
\end{enumerate}

\textbf{The "5 tau rule" for discharging:}
\begin{itemize}
    \item After 5$\tau$, capacitor considered "fully discharged"
    \item Actually 0.7\%, essentially zero for practical purposes
    \item Total discharging time $\approx 5 \times R \times C$
\end{itemize}

\vspace{0.15cm}

\textbf{Current During Charging/Discharging:}

\textbf{Charging current:}
\begin{itemize}
    \item $I(t) = \frac{V_{supply}}{R} \times e^{-t/\tau}$
    \item Maximum at $t = 0$: $I_{max} = \frac{V_{supply}}{R}$
    \item Exponentially decreases
    \item At $t = 1\tau$: Current drops to 37\% of initial
    \item At $t = 5\tau$: Current essentially zero
\end{itemize}

\textbf{Discharging current:}
\begin{itemize}
    \item $I(t) = -\frac{V_{initial}}{R} \times e^{-t/\tau}$
    \item Negative sign: Opposite direction to charging
    \item Maximum magnitude at $t = 0$
    \item Exponentially decreases in magnitude
\end{itemize}

\vspace{0.15cm}

\textbf{Unit Conversions (Critical!):}

\textbf{Remember powers of ten:}
\begin{itemize}
    \item Resistors often in k$\Omega$ or M$\Omega$
    \item Capacitors often in $\mu$F, nF, or pF
    \item Must convert to base units for $\tau$ calculation!
\end{itemize}

\textbf{Common conversions:}
\begin{itemize}
    \item 1 k$\Omega$ = 1,000 $\Omega$ = $10^3$ $\Omega$
    \item 1 M$\Omega$ = 1,000,000 $\Omega$ = $10^6$ $\Omega$
    \item 1 $\mu$F = 0.000001 F = $10^{-6}$ F
    \item 1 nF = 0.000000001 F = $10^{-9}$ F
    \item 1 pF = 0.000000000001 F = $10^{-12}$ F
\end{itemize}

\textbf{Shortcut for common combinations:}
\begin{itemize}
    \item k$\Omega$ $\times$ $\mu$F = ms (milliseconds)
    \item M$\Omega$ $\times$ $\mu$F = s (seconds)
    \item k$\Omega$ $\times$ nF = $\mu$s (microseconds)
\end{itemize}

\vspace{0.15cm}

\textbf{Factors Affecting Time Constant:}

\textbf{Resistance effect:}
\begin{itemize}
    \item Larger R $\rightarrow$ limits current more $\rightarrow$ slower charging
    \item Double R $\rightarrow$ double $\tau$ $\rightarrow$ 2$\times$ longer time
    \item Smaller R $\rightarrow$ faster charging (but higher current!)
\end{itemize}

\textbf{Capacitance effect:}
\begin{itemize}
    \item Larger C $\rightarrow$ more charge needed $\rightarrow$ slower charging
    \item Double C $\rightarrow$ double $\tau$ $\rightarrow$ 2$\times$ longer time
    \item Smaller C $\rightarrow$ faster charging
\end{itemize}

\textbf{Practical implications:}
\begin{itemize}
    \item Want fast charging? Use small R and small C
    \item Want slow charging (timing delay)? Use large R or large C
    \item Trade-offs: Small R = high current, large C = bigger/expensive
\end{itemize}
\end{detailbox}

\vspace{0.2cm}

\noindent\textbf{\color{accentcolor} Practical Examples \& Numerical Calculations}
\begin{examplebox}
\textbf{Example 1: Basic Time Constant Calculation}

Given: R = 100 k$\Omega$, C = 200 $\mu$F

Calculate time constant:
\begin{itemize}
    \item Convert to base units: R = 100,000 $\Omega$, C = 0.0002 F
    \item $\tau = R \times C = 100,000 \times 0.0002 = 20$ seconds
    \item Or use shortcut: 100 k$\Omega$ $\times$ 200 $\mu$F = 100 $\times$ 200 ms = 20,000 ms = 20 s
\end{itemize}

Full charging time:
\begin{itemize}
    \item $5\tau = 5 \times 20 = 100$ seconds
    \item Takes ~1 minute 40 seconds to fully charge
\end{itemize}

\vspace{0.15cm}

\textbf{Example 2: Charging to Specific Voltage}

Given: $V_{supply}$ = 12V, R = 10 k$\Omega$, C = 100 $\mu$F

Time constant: $\tau = 10 \times 10^3 \times 100 \times 10^{-6} = 1$ second

Voltage after 1 second (1$\tau$):
\begin{itemize}
    \item $V_C = 12 \times 0.632 = 7.58$V (63.2\% of 12V)
\end{itemize}

Voltage after 2 seconds (2$\tau$):
\begin{itemize}
    \item $V_C = 12 \times 0.865 = 10.38$V (86.5\% of 12V)
\end{itemize}

Voltage after 5 seconds (5$\tau$):
\begin{itemize}
    \item $V_C = 12 \times 0.993 = 11.92$V (99.3\% of 12V - essentially full)
\end{itemize}

\vspace{0.15cm}

\textbf{Example 3: Discharging from Initial Voltage}

Given: $V_{initial}$ = 9V, R = 47 k$\Omega$, C = 1000 $\mu$F

Time constant: $\tau = 47 \times 10^3 \times 1000 \times 10^{-6} = 47$ seconds

Voltage after 47 seconds (1$\tau$):
\begin{itemize}
    \item $V_C = 9 \times 0.368 = 3.31$V (36.8\% remains)
\end{itemize}

Voltage after 94 seconds (2$\tau$):
\begin{itemize}
    \item $V_C = 9 \times 0.135 = 1.22$V (13.5\% remains)
\end{itemize}

Time to fully discharge:
\begin{itemize}
    \item $5\tau = 5 \times 47 = 235$ seconds $\approx$ 3 minutes 55 seconds
\end{itemize}

\vspace{0.15cm}

\textbf{Example 4: Initial Charging Current}

Given: $V_{supply}$ = 5V, R = 1 k$\Omega$, C = 470 $\mu$F

Time constant: $\tau = 1000 \times 470 \times 10^{-6} = 0.47$ seconds

Initial charging current ($t = 0$):
\begin{itemize}
    \item $I_{max} = \frac{V_{supply}}{R} = \frac{5}{1000} = 0.005$A = 5 mA
    \item This is the current at the instant voltage is applied
\end{itemize}

Current after 0.47 seconds (1$\tau$):
\begin{itemize}
    \item $I = 5 \text{ mA} \times 0.368 = 1.84$ mA (dropped to 37\%)
\end{itemize}

Current after 2.35 seconds (5$\tau$):
\begin{itemize}
    \item $I \approx 0$ mA (essentially zero, capacitor fully charged)
\end{itemize}

\vspace{0.15cm}

\textbf{Example 5: Designing for Specific Delay}

Requirement: Need 10-second delay

Choose C = 100 $\mu$F (standard value)

Calculate required R:
\begin{itemize}
    \item $\tau = R \times C$
    \item $R = \frac{\tau}{C} = \frac{10}{100 \times 10^{-6}} = 100,000$ $\Omega$ = 100 k$\Omega$
\end{itemize}

Verification:
\begin{itemize}
    \item $\tau = 100 \text{ k}\Omega \times 100 \text{ $\mu$F} = 10$ s $\checkmark$
    \item Full charge time: $5\tau = 50$ seconds
\end{itemize}

\vspace{0.15cm}

\textbf{Example 6: Effect of Doubling Components}

Original: R = 10 k$\Omega$, C = 22 $\mu$F, $\tau$ = 0.22 s

Double resistance (R = 20 k$\Omega$):
\begin{itemize}
    \item New $\tau = 20 \times 10^3 \times 22 \times 10^{-6} = 0.44$ s
    \item Time constant doubles! (2$\times$ slower)
\end{itemize}

Double capacitance (C = 44 $\mu$F, R back to 10 k$\Omega$):
\begin{itemize}
    \item New $\tau = 10 \times 10^3 \times 44 \times 10^{-6} = 0.44$ s
    \item Time constant also doubles! (same effect as doubling R)
\end{itemize}

Double both (R = 20 k$\Omega$, C = 44 $\mu$F):
\begin{itemize}
    \item New $\tau = 20 \times 10^3 \times 44 \times 10^{-6} = 0.88$ s
    \item Time constant quadruples! (4$\times$ slower)
\end{itemize}

\vspace{0.15cm}

\textbf{Example 7: Unit Conversion Practice}

Calculate $\tau$ for: R = 2.2 M$\Omega$, C = 10 $\mu$F

Method 1 (convert to base units):
\begin{itemize}
    \item R = 2,200,000 $\Omega$, C = 0.00001 F
    \item $\tau = 2,200,000 \times 0.00001 = 22$ seconds
\end{itemize}

Method 2 (use shortcut: M$\Omega$ $\times$ $\mu$F = s):
\begin{itemize}
    \item $\tau = 2.2 \times 10 = 22$ seconds $\checkmark$
\end{itemize}

Full charge time: $5\tau = 110$ seconds = 1 minute 50 seconds
\end{examplebox}

\vspace{0.2cm}

\noindent\textbf{\color{accentcolor} Key Points (Interview Focus)}
\begin{keypointsbox}
\begin{enumerate}
    \item \textbf{Time Constant Formula:} $\tau = R \times C$ (R in $\Omega$, C in F, $\tau$ in s)
    
    \item \textbf{Charging Key Points:}
    \begin{itemize}
        \item 1$\tau$: 63.2\% charged
        \item 5$\tau$: 99.3\% charged (considered "full")
        \item $V_C(t) = V_{supply}(1 - e^{-t/\tau})$
    \end{itemize}
    
    \item \textbf{Discharging Key Points:}
    \begin{itemize}
        \item 1$\tau$: 36.8\% remains
        \item 5$\tau$: 0.7\% remains (considered "empty")
        \item $V_C(t) = V_{initial} \times e^{-t/\tau}$
    \end{itemize}
    
    \item \textbf{Exponential Process:} NOT linear! Fast initially, slows down later
    
    \item \textbf{5 Tau Rule:} Full charge/discharge takes approximately $5 \times R \times C$
    
    \item \textbf{Current Behavior:} Maximum at $t=0$, exponentially decreases
    
    \item \textbf{Unit Shortcuts:}
    \begin{itemize}
        \item k$\Omega$ $\times$ $\mu$F = ms
        \item M$\Omega$ $\times$ $\mu$F = s
        \item k$\Omega$ $\times$ nF = $\mu$s
    \end{itemize}
    
    \item \textbf{Component Effects:}
    \begin{itemize}
        \item Larger R or C $\rightarrow$ slower charging (longer $\tau$)
        \item Smaller R or C $\rightarrow$ faster charging (shorter $\tau$)
        \item Proportional relationship: 2$\times$R or 2$\times$C $\rightarrow$ 2$\times$$\tau$
    \end{itemize}
\end{enumerate}

\vspace{0.2cm}

\textbf{Interview Questions:}
\begin{itemize}
    \item \textbf{Q:} What is the RC time constant? \\
    \textit{A:} $\tau = R \times C$, measured in seconds. It determines how fast a capacitor charges or discharges through a resistor.
    
    \item \textbf{Q:} How much is a capacitor charged after 1 time constant? \\
    \textit{A:} 63.2\% of the supply voltage (charging) or 36.8\% remaining (discharging).
    
    \item \textbf{Q:} When is a capacitor considered fully charged? \\
    \textit{A:} After 5 time constants (5$\tau$), when it reaches 99.3\% of supply voltage.
    
    \item \textbf{Q:} Is charging/discharging linear or exponential? \\
    \textit{A:} Exponential. Fast change initially, then slows down asymptotically.
    
    \item \textbf{Q:} What happens to time constant if you double the resistance? \\
    \textit{A:} Time constant doubles ($\tau$ is proportional to R). Charging takes twice as long.
    
    \item \textbf{Q:} Why does resistor slow down charging? \\
    \textit{A:} Resistor limits current: $I = \frac{V_{supply} - V_C}{R}$. Larger R $\rightarrow$ smaller I $\rightarrow$ slower charging.
    
    \item \textbf{Q:} What is the initial current when charging starts? \\
    \textit{A:} $I_{max} = \frac{V_{supply}}{R}$ (capacitor voltage is 0, so full voltage across R).
\end{itemize}

\vspace{0.2cm}

\textbf{Formulas Summary:}
\begin{itemize}
    \item \textbf{Time Constant:} $\tau = R \times C$
    \item \textbf{Charging Voltage:} $V_C(t) = V_{supply}(1 - e^{-t/\tau})$
    \item \textbf{Discharging Voltage:} $V_C(t) = V_{initial} \times e^{-t/\tau}$
    \item \textbf{Charging Current:} $I(t) = \frac{V_{supply}}{R} e^{-t/\tau}$
    \item \textbf{Discharging Current:} $I(t) = -\frac{V_{initial}}{R} e^{-t/\tau}$
    \item \textbf{Full Charge/Discharge Time:} $t_{full} \approx 5\tau$
\end{itemize}

\vspace{0.2cm}

\textbf{Common Mistakes:}
\begin{itemize}
    \item Forgetting unit conversions (using k$\Omega$ or $\mu$F directly)
    \item Thinking charging is linear (it's exponential!)
    \item Confusing charging percentages (63\%) with discharging (37\%)
    \item Assuming capacitor fully charges at 1$\tau$ (actually only 63\%)
    \item Not accounting for 5$\tau$ total time in timing circuits
\end{itemize}
\end{keypointsbox}

% --------------------------------------------------------------------
\subsection{RC Circuits: Charging, Discharging, and Signal Filtering}

\noindent\textbf{\color{accentcolor} TL;DR (The Gist)}
\begin{tldrbox}
\begin{itemize}
    \item \textbf{Square Wave Input:} Capacitor charges/discharges repeatedly
    \item \textbf{If pulse width $\gg$ 5$\tau$:} Full charge/discharge, output looks like input
    \item \textbf{If pulse width $\ll$ 5$\tau$:} Partial charge/discharge, output smoothed/filtered
    \item \textbf{Low-Pass Filter:} Passes low frequencies, blocks high frequencies. $f_c = \frac{1}{2\pi RC}$
\end{itemize}
\end{tldrbox}

\vspace{0.2cm}

\noindent\textbf{\color{accentcolor} Detailed Explanation}
\begin{detailbox}
\textbf{RC Circuit Response to Square Waves:}

\vspace{0.15cm}

\textbf{The setup:}
\begin{itemize}
    \item RC circuit (resistor + capacitor in series)
    \item Input: Square wave (alternates between high and low)
    \item Output: Voltage across capacitor
    \item Behavior depends on relationship between pulse width and time constant
\end{itemize}

\vspace{0.15cm}

\textbf{Case 1: Pulse Width $\gg$ 5$\tau$ (Long Pulses)}

\textbf{What happens:}
\begin{itemize}
    \item Pulse duration much longer than 5 time constants
    \item Capacitor has enough time to fully charge (during high)
    \item Capacitor has enough time to fully discharge (during low)
    \item Output voltage closely follows input square wave
    \item Nearly perfect square wave at output
\end{itemize}

\textbf{Example:}
\begin{itemize}
    \item $\tau = 1$ ms, Pulse width = 50 ms (50$\tau$)
    \item Capacitor charges fully in first 5 ms
    \item Stays charged for remaining 45 ms of pulse
    \item Then discharges fully when pulse goes low
    \item Output: Clean square wave
\end{itemize}

\vspace{0.15cm}

\textbf{Case 2: Pulse Width $\approx$ 5$\tau$ (Matched Pulses)}

\textbf{What happens:}
\begin{itemize}
    \item Pulse duration approximately equals 5 time constants
    \item Capacitor just barely reaches full charge/discharge
    \item Output shows exponential curves
    \item Visible charging/discharging slopes
    \item Output resembles classic RC waveform
\end{itemize}

\textbf{Example:}
\begin{itemize}
    \item $\tau = 10$ ms, Pulse width = 50 ms (5$\tau$)
    \item Capacitor charges to 99\% during high pulse
    \item Then immediately starts discharging
    \item Output: Exponential rise and fall visible
\end{itemize}

\vspace{0.15cm}

\textbf{Case 3: Pulse Width $\approx$ 2$\tau$ (Medium Pulses)}

\textbf{What happens:}
\begin{itemize}
    \item Pulse duration about 2 time constants
    \item Capacitor only charges to ~86\% during high
    \item Capacitor only discharges to ~13\% during low
    \item Output voltage range reduced compared to input
    \item Clear exponential charging/discharging curves
\end{itemize}

\textbf{Example:}
\begin{itemize}
    \item Input: 0V to 5V square wave
    \item $\tau = 10$ ms, Pulse width = 20 ms (2$\tau$)
    \item Capacitor charges to: $5 \times 0.865 = 4.33$V
    \item Then discharges to: $4.33 \times 0.135 = 0.58$V
    \item Output swings: 0.58V to 4.33V (reduced from 0V-5V)
\end{itemize}

\vspace{0.15cm}

\textbf{Case 4: Pulse Width $\ll$ 5$\tau$ (Short Pulses)}

\textbf{What happens:}
\begin{itemize}
    \item Pulse duration much shorter than 5 time constants
    \item Capacitor doesn't have time to charge or discharge significantly
    \item Output voltage stays relatively constant
    \item Only small ripple around average value
    \item Acts as filter - removes fast changes
\end{itemize}

\textbf{Example:}
\begin{itemize}
    \item Input: 0V to 5V square wave, 50\% duty cycle
    \item $\tau = 100$ ms, Pulse width = 1 ms (0.01$\tau$)
    \item Capacitor barely charges/discharges
    \item Output: Approximately 2.5V DC with tiny ripple
    \item Square wave "smoothed" to nearly constant voltage
\end{itemize}

\vspace{0.15cm}

\textbf{RC Low-Pass Filter:}

\textbf{What is a low-pass filter?}
\begin{itemize}
    \item Passes low frequencies (lets them through)
    \item Blocks (attenuates) high frequencies
    \item RC circuit acts as low-pass filter naturally
    \item Output taken across capacitor
\end{itemize}

\textbf{How it works:}
\begin{itemize}
    \item \textbf{Low frequencies:} Long period, capacitor has time to charge/discharge, follows input
    \item \textbf{High frequencies:} Short period, capacitor can't respond fast enough, output attenuated
    \item Cutoff frequency ($f_c$): Frequency where signal reduced to 70.7\% (-3 dB)
\end{itemize}

\textbf{Cutoff frequency formula:}
\begin{itemize}
    \item $f_c = \frac{1}{2\pi RC}$ (in Hz)
    \item Where R in $\Omega$, C in F
    \item This is the -3 dB point
    \item Below $f_c$: Most signal passes
    \item Above $f_c$: Signal increasingly attenuated
\end{itemize}

\vspace{0.15cm}

\textbf{Filter Response Characteristics:}

\textbf{Frequency below cutoff ($f < f_c$):}
\begin{itemize}
    \item Signal passes with minimal attenuation
    \item Output amplitude $\approx$ input amplitude
    \item Capacitor reactance high compared to signal period
\end{itemize}

\textbf{Frequency at cutoff ($f = f_c$):}
\begin{itemize}
    \item Signal amplitude reduced to 70.7\% (0.707)
    \item This is -3 dB attenuation
    \item Power reduced to half (-3 dB = half power)
    \item Capacitive reactance equals resistance: $X_C = R$
\end{itemize}

\textbf{Frequency above cutoff ($f > f_c$):}
\begin{itemize}
    \item Signal increasingly attenuated
    \item Higher frequency $\rightarrow$ more attenuation
    \item Roll-off: -20 dB per decade (10$\times$ frequency)
    \item At 10$\times$$f_c$: Output is 1/10th of input
\end{itemize}

\vspace{0.15cm}

\textbf{Practical Filtering Applications:}

\textbf{Removing high-frequency noise:}
\begin{itemize}
    \item Desired signal: Low frequency (e.g., audio, DC with ripple)
    \item Unwanted noise: High frequency (e.g., switching noise, RF interference)
    \item Design filter with $f_c$ above desired signal, below noise
    \item RC filter passes signal, blocks noise
\end{itemize}

\textbf{Example - Power supply filtering:}
\begin{itemize}
    \item DC power with 100 kHz switching noise
    \item Want to remove noise, keep DC
    \item Choose $f_c = 1$ kHz (well below noise frequency)
    \item If R = 10 $\Omega$, need: $C = \frac{1}{2\pi f_c R} = \frac{1}{2\pi \times 1000 \times 10} \approx 16$ $\mu$F
    \item 100 kHz noise attenuated by factor of 100 (-40 dB)
\end{itemize}

\textbf{Example - Audio filtering (removing bass):}
\begin{itemize}
    \item Remove frequencies below 200 Hz
    \item Use high-pass filter (output across R, not C!)
    \item Same $f_c$ formula applies
    \item Choose C = 1 $\mu$F, calculate R: $R = \frac{1}{2\pi f_c C} = \frac{1}{2\pi \times 200 \times 10^{-6}} \approx 796$ $\Omega$
\end{itemize}

\vspace{0.15cm}

\textbf{Duty Cycle Effects:}

\textbf{50\% duty cycle (symmetric square wave):}
\begin{itemize}
    \item Equal high and low times
    \item If pulse width $\ll$ $\tau$: Output settles to average (half of input swing)
    \item If pulse width $\approx$ $\tau$: Output oscillates symmetrically
\end{itemize}

\textbf{Non-50\% duty cycle:}
\begin{itemize}
    \item Unequal charge/discharge times
    \item Output DC level shifts
    \item Longer high time $\rightarrow$ higher average output voltage
    \item Longer low time $\rightarrow$ lower average output voltage
\end{itemize}

\vspace{0.15cm}

\textbf{Designing RC Circuits for Specific Response:}

\textbf{Want output to follow input (minimal filtering):}
\begin{itemize}
    \item Choose $\tau \ll$ pulse width
    \item Rule of thumb: $\tau < \frac{1}{10}$ of shortest pulse
    \item Ensures full charge/discharge
\end{itemize}

\textbf{Want output smoothed (heavy filtering):}
\begin{itemize}
    \item Choose $\tau \gg$ pulse width
    \item Rule of thumb: $\tau > 10 \times$ pulse width
    \item Capacitor can't respond to changes
\end{itemize}

\textbf{Want visible charging curves (educational):}
\begin{itemize}
    \item Choose $\tau \approx$ pulse width
    \item See exponential charging/discharging
    \item Useful for learning/demonstration
\end{itemize}
\end{detailbox}

\vspace{0.2cm}

\noindent\textbf{\color{accentcolor} Practical Examples \& Numerical Calculations}
\begin{examplebox}
\textbf{Example 1: Cutoff Frequency Calculation}

Given: R = 1 k$\Omega$, C = 100 nF

Calculate cutoff frequency:
\begin{itemize}
    \item $f_c = \frac{1}{2\pi RC} = \frac{1}{2\pi \times 1000 \times 100 \times 10^{-9}}$
    \item $f_c = \frac{1}{2\pi \times 10^{-4}} = \frac{1}{6.28 \times 10^{-4}} \approx 1592$ Hz $\approx 1.6$ kHz
\end{itemize}

Response at different frequencies:
\begin{itemize}
    \item At 100 Hz ($< f_c$): Signal passes (minimal attenuation)
    \item At 1592 Hz ($= f_c$): Signal reduced to 70.7\%
    \item At 16 kHz ($10 \times f_c$): Signal reduced to ~10\%
\end{itemize}

\vspace{0.15cm}

\textbf{Example 2: Designing Filter for Noise Removal}

Requirement: Remove 50 kHz switching noise from DC signal

Choose cutoff well below noise: $f_c = 5$ kHz (10$\times$ lower than noise)

Select R = 100 $\Omega$ (low to avoid voltage drop)

Calculate required C:
\begin{itemize}
    \item $C = \frac{1}{2\pi f_c R} = \frac{1}{2\pi \times 5000 \times 100}$
    \item $C = \frac{1}{3.14 \times 10^6} \approx 0.318$ $\mu$F
    \item Use standard value: 0.33 $\mu$F or 330 nF
\end{itemize}

Verify noise attenuation at 50 kHz:
\begin{itemize}
    \item Noise is 10$\times$ above cutoff
    \item Attenuation: Approximately 1/10 or -20 dB
    \item If noise was 1V, output noise $\approx$ 0.1V
\end{itemize}

\vspace{0.15cm}

\textbf{Example 3: Square Wave Response}

Given: R = 10 k$\Omega$, C = 22 $\mu$F, Input = 0-5V square wave at 10 Hz (50\% duty)

Time constant: $\tau = 10,000 \times 22 \times 10^{-6} = 0.22$ s = 220 ms

Square wave period: $T = \frac{1}{10} = 0.1$ s = 100 ms

Pulse width (50\% duty): 50 ms

Compare pulse width to time constant:
\begin{itemize}
    \item Pulse width = 50 ms
    \item $\tau$ = 220 ms
    \item Pulse width $< \tau$ (actually 0.23$\tau$)
\end{itemize}

Charging during 50 ms pulse:
\begin{itemize}
    \item Time = 0.23$\tau$ (not even 1$\tau$)
    \item Voltage reaches: $5 \times (1 - e^{-0.23}) = 5 \times 0.205 \approx 1.03$V
    \item Only charges to 20.5\% of 5V
\end{itemize}

Discharging during 50 ms low:
\begin{itemize}
    \item Starting from 1.03V
    \item Discharges for 0.23$\tau$
    \item Voltage drops to: $1.03 \times e^{-0.23} = 1.03 \times 0.795 \approx 0.82$V
\end{itemize}

Result:
\begin{itemize}
    \item Output oscillates between 0.82V and 1.03V (small ripple!)
    \item Input was 0-5V, but output only varies by 0.21V
    \item Heavily filtered/smoothed
\end{itemize}

\vspace{0.15cm}

\textbf{Example 4: Fast Square Wave (High Frequency)}

Given: Same circuit (R = 10 k$\Omega$, C = 22 $\mu$F, $\tau$ = 220 ms)

Input: 0-5V square wave at 100 Hz (period = 10 ms, pulse width = 5 ms)

Compare: Pulse width = 5 ms = 0.023$\tau$ (very short!)

Charging during 5 ms:
\begin{itemize}
    \item $V_C = 5 \times (1 - e^{-0.023}) = 5 \times 0.023 \approx 0.11$V
    \item Barely charges at all!
\end{itemize}

Result:
\begin{itemize}
    \item Output voltage nearly constant
    \item Tiny ripple around 2.5V (average of 0-5V)
    \item 100 Hz signal completely filtered out
    \item Acts as smoothing capacitor
\end{itemize}

\vspace{0.15cm}

\textbf{Example 5: Matching Time Constant to Pulse}

Requirement: Want to see clear RC charging curve on oscilloscope

Input: 1 kHz square wave (period = 1 ms, pulse width = 0.5 ms)

Design for 5$\tau$ = pulse width:
\begin{itemize}
    \item $5\tau = 0.5$ ms
    \item $\tau = 0.1$ ms = 100 $\mu$s
\end{itemize}

Choose C = 10 nF, calculate R:
\begin{itemize}
    \item $R = \frac{\tau}{C} = \frac{100 \times 10^{-6}}{10 \times 10^{-9}} = 10,000$ $\Omega$ = 10 k$\Omega$
\end{itemize}

Verification:
\begin{itemize}
    \item $\tau = 10,000 \times 10 \times 10^{-9} = 100$ $\mu$s $\checkmark$
    \item During 0.5 ms pulse, capacitor charges for 5$\tau$ $\rightarrow$ reaches 99\%
    \item Clear exponential curve visible on scope
\end{itemize}

\vspace{0.15cm}

\textbf{Example 6: Audio Low-Pass Filter}

Application: Remove high-frequency hiss above 5 kHz

Choose cutoff: $f_c = 5$ kHz

Select C = 47 nF (standard audio capacitor value)

Calculate R:
\begin{itemize}
    \item $R = \frac{1}{2\pi f_c C} = \frac{1}{2\pi \times 5000 \times 47 \times 10^{-9}}$
    \item $R = \frac{1}{1.476 \times 10^{-3}} \approx 677$ $\Omega$
    \item Use standard value: 680 $\Omega$
\end{itemize}

Performance:
\begin{itemize}
    \item Below 5 kHz: Audio passes clearly
    \item At 5 kHz: -3 dB (barely noticeable)
    \item At 50 kHz: -20 dB (1/10th amplitude, noise greatly reduced)
\end{itemize}

\vspace{0.15cm}

\textbf{Example 7: Power Supply Ripple Filtering}

Given: 12V DC with 120 Hz ripple (from full-wave rectifier)

Want: Reduce ripple below 100 mV

Choose cutoff: $f_c = 12$ Hz (10$\times$ below ripple frequency)

If R = 10 $\Omega$ (low resistance to minimize voltage drop):
\begin{itemize}
    \item $C = \frac{1}{2\pi \times 12 \times 10} = \frac{1}{754} \approx 1327$ $\mu$F
    \item Use standard value: 1500 $\mu$F or 2200 $\mu$F electrolytic
\end{itemize}

Attenuation at 120 Hz:
\begin{itemize}
    \item 120 Hz is 10$\times$ above cutoff
    \item Attenuation: ~1/10 (-20 dB)
    \item If original ripple was 1V, filtered ripple $\approx$ 100 mV $\checkmark$
\end{itemize}
\end{examplebox}

\vspace{0.2cm}

\noindent\textbf{\color{accentcolor} Key Points (Interview Focus)}
\begin{keypointsbox}
\begin{enumerate}
    \item \textbf{Square Wave Response:} Depends on pulse width vs. time constant ($\tau$)
    
    \item \textbf{Pulse $\gg$ 5$\tau$:} Full charge/discharge, output follows input
    
    \item \textbf{Pulse $\ll$ 5$\tau$:} Partial charge/discharge, output smoothed
    
    \item \textbf{Low-Pass Filter:} RC circuit naturally filters high frequencies
    
    \item \textbf{Cutoff Frequency:} $f_c = \frac{1}{2\pi RC}$ (-3 dB point, 70.7\% amplitude)
    
    \item \textbf{Filter Response:}
    \begin{itemize}
        \item Below $f_c$: Signal passes
        \item At $f_c$: -3 dB (70.7\%)
        \item Above $f_c$: -20 dB/decade roll-off
    \end{itemize}
    
    \item \textbf{Applications:} Noise filtering, smoothing, signal conditioning
    
    \item \textbf{Design Trade-off:} Larger $\tau$ = better filtering but slower response
\end{enumerate}

\vspace{0.2cm}

\textbf{Interview Questions:}
\begin{itemize}
    \item \textbf{Q:} What happens when square wave pulse is much shorter than time constant? \\
    \textit{A:} Capacitor doesn't have time to charge significantly. Output becomes smooth DC with tiny ripple (filtering effect).
    
    \item \textbf{Q:} What is cutoff frequency of RC low-pass filter? \\
    \textit{A:} $f_c = \frac{1}{2\pi RC}$. Frequency where signal is attenuated to 70.7\% (-3 dB).
    
    \item \textbf{Q:} How does RC circuit filter high frequencies? \\
    \textit{A:} At high frequencies, capacitor reactance is low, so signal shorted to ground through capacitor. Low frequencies have high reactance, so signal passes.
    
    \item \textbf{Q:} What is roll-off rate of RC filter? \\
    \textit{A:} -20 dB per decade. At 10$\times$ cutoff frequency, signal is 1/10th amplitude.
    
    \item \textbf{Q:} How to design filter to remove 50 kHz noise? \\
    \textit{A:} Choose cutoff well below 50 kHz (e.g., 5 kHz), then calculate R and C using $f_c = \frac{1}{2\pi RC}$.
\end{itemize}

\vspace{0.2cm}

\textbf{Formulas:}
\begin{itemize}
    \item \textbf{Cutoff Frequency:} $f_c = \frac{1}{2\pi RC}$
    \item \textbf{Component from Cutoff:} $C = \frac{1}{2\pi f_c R}$ or $R = \frac{1}{2\pi f_c C}$
    \item \textbf{Time Constant:} $\tau = RC = \frac{1}{2\pi f_c}$
\end{itemize}

\vspace{0.2cm}

\textbf{Applications:}
\begin{itemize}
    \item Power supply noise filtering
    \item Audio tone controls (bass/treble)
    \item Anti-aliasing filters (before ADC)
    \item Smoothing rectified AC to DC
    \item Debouncing switches (removing bounce noise)
\end{itemize}
\end{keypointsbox}

% --------------------------------------------------------------------
\subsection{Application Examples: Timing Circuits and Comparators}

\noindent\textbf{\color{accentcolor} TL;DR (The Gist)}
\begin{tldrbox}
\begin{itemize}
    \item \textbf{Time Delay Circuit:} RC charges, CMOS buffer switches at threshold $\rightarrow$ pulse delay
    \item \textbf{Timer with Comparator:} Button charges cap, comparator output high until cap discharges below reference
    \item \textbf{Comparator:} Compares two voltages, output goes high/low based on which input is higher
    \item Real-world applications: Delays, timers, pulse shaping, automatic control
\end{itemize}
\end{tldrbox}

\vspace{0.2cm}

\noindent\textbf{\color{accentcolor} Detailed Explanation}
\begin{detailbox}
\textbf{Application 1: Time Delay Circuit with CMOS Buffer}

\vspace{0.15cm}

\textbf{Circuit description:}
\begin{itemize}
    \item Input: Pulse (step voltage)
    \item RC charging circuit
    \item CMOS buffer (digital IC) connected after RC
    \item Output: Delayed pulse
\end{itemize}

\textbf{What is a CMOS buffer?}
\begin{itemize}
    \item Digital component (pair of transistors)
    \item High input impedance (draws very little current)
    \item Low output impedance (can drive loads)
    \item Acts as current amplifier (voltage stays same)
    \item Requires external power supply (active device)
    \item Symbol: Triangle
\end{itemize}

\textbf{CMOS buffer switching behavior:}
\begin{itemize}
    \item Has a threshold voltage (typically ~50\% of supply)
    \item When input below threshold: Output = LOW (0V)
    \item When input above threshold: Output = HIGH ($V_{supply}$)
    \item Switches quickly between states
    \item Gradual input $\rightarrow$ sharp output transition
\end{itemize}

\vspace{0.15cm}

\textbf{How the time delay works:}

\textbf{Point A (input):}
\begin{itemize}
    \item Sharp rising edge (pulse)
    \item Goes from 0V to $V_{supply}$ instantly
\end{itemize}

\textbf{Point B (after RC):}
\begin{itemize}
    \item Capacitor charging voltage
    \item Exponential rise from 0V to $V_{supply}$
    \item Takes 5$\tau$ to fully charge
    \item Smooth, curved waveform
\end{itemize}

\textbf{Point C (CMOS buffer output):}
\begin{itemize}
    \item Stays LOW until Point B reaches threshold
    \item When threshold crossed: Switches HIGH abruptly
    \item Output pulse delayed by time it takes to reach threshold
    \item Sharp edges (even though input was gradual)
\end{itemize}

\textbf{Calculating the delay:}
\begin{itemize}
    \item If threshold = 50\% of $V_{supply}$
    \item Capacitor reaches 50\% at approximately 0.7$\tau$
    \item Delay time $\approx 0.7 \times R \times C$
    \item Can adjust delay by changing R or C
\end{itemize}

\textbf{Why use this circuit?}
\begin{itemize}
    \item Creates time delay between events
    \item Allows one action to complete before another starts
    \item Pulse shaping (converts slow rise to sharp edge)
    \item Clock delay in digital circuits
\end{itemize}

\vspace{0.15cm}

\textbf{Application 2: One-Shot Timer with Comparator}

\vspace{0.15cm}

\textbf{What is a comparator?}
\begin{itemize}
    \item Integrated circuit (IC)
    \item Two inputs: (+) non-inverting, (-) inverting
    \item One output
    \item Powered by $V+$ (positive supply) and ground
    \item Symbol: Triangle with + and - inputs
\end{itemize}

\textbf{How comparator works:}
\begin{itemize}
    \item Compares voltage at (+) input to voltage at (-) input
    \item If (+) $>$ (-): Output = HIGH ($V+$)
    \item If (+) $<$ (-): Output = LOW (ground)
    \item Switches very quickly
    \item Draws negligible current from inputs (high impedance)
    \item Can source/sink ~20 mA at output
\end{itemize}

\vspace{0.15cm}

\textbf{One-shot timer circuit description:}

\textbf{Components:}
\begin{itemize}
    \item Voltage divider (two resistors): Sets reference voltage
    \item Capacitor: Stores charge for timing
    \item Resistor in series with cap: Controls discharge rate
    \item Push button: Triggers timer
    \item Comparator: Monitors capacitor voltage
\end{itemize}

\textbf{Reference voltage (at - input):}
\begin{itemize}
    \item Created by voltage divider
    \item Constant voltage (e.g., 37\% of $V+$ = 1.8V for 5V supply)
    \item Stays fixed throughout operation
    \item This is the comparison threshold
\end{itemize}

\vspace{0.15cm}

\textbf{Operation sequence:}

\textbf{1. Initial state (idle):}
\begin{itemize}
    \item Capacitor fully discharged (0V)
    \item (+) input at 0V
    \item (-) input at reference (1.8V)
    \item Since (+) $<$ (-): Output = LOW (ground)
\end{itemize}

\textbf{2. Button pressed momentarily:}
\begin{itemize}
    \item Capacitor charges quickly to $V+$ (5V) through button
    \item Button released
    \item (+) input now at 5V
    \item (-) input still at 1.8V
    \item Since (+) $>$ (-): Output switches to HIGH (5V)
\end{itemize}

\textbf{3. Timing phase (button released):}
\begin{itemize}
    \item Capacitor discharges through resistor
    \item Exponential decay: $V_C(t) = 5 \times e^{-t/\tau}$
    \item Time constant: $\tau = R \times C$
    \item Output stays HIGH while $V_C > 1.8$V
\end{itemize}

\textbf{4. Timeout (end of timing):}
\begin{itemize}
    \item Capacitor voltage drops below 1.8V (reference)
    \item (+) input now $<$ (-) input
    \item Comparator output switches back to LOW
    \item Circuit ready for next trigger
\end{itemize}

\vspace{0.15cm}

\textbf{Calculating the timeout duration:}

\textbf{Voltage equation during discharge:}
\begin{itemize}
    \item $V_C(t) = V_{initial} \times e^{-t/\tau}$
    \item Where $V_{initial} = 5$V, $V_{reference} = 1.8$V
\end{itemize}

\textbf{Find time when $V_C$ = 1.8V:}
\begin{itemize}
    \item $1.8 = 5 \times e^{-t/\tau}$
    \item $\frac{1.8}{5} = e^{-t/\tau}$
    \item $0.36 = e^{-t/\tau}$
    \item Taking natural log: $\ln(0.36) = -\frac{t}{\tau}$
    \item $-1.02 = -\frac{t}{\tau}$
    \item $t = 1.02\tau$
\end{itemize}

\textbf{Timeout = 1.02$\tau$ = 1.02 $\times$ R $\times$ C}

\textbf{Example:}
\begin{itemize}
    \item Want 1-minute (60 second) timeout
    \item Choose C = 1000 $\mu$F
    \item $R = \frac{60}{1.02 \times 0.001} = 58,800$ $\Omega$ $\approx$ 56 k$\Omega$ or 62 k$\Omega$
\end{itemize}

\vspace{0.15cm}

\textbf{Why different voltages give different timeout formulas?}

\textbf{For 37\% reference (1 time constant):}
\begin{itemize}
    \item $V_{ref} = 0.37 \times V_{initial}$
    \item Timeout exactly = 1$\tau$ (by definition of time constant!)
\end{itemize}

\textbf{For 36\% reference (like 1.8V/5V):}
\begin{itemize}
    \item $V_{ref} = 0.36 \times V_{initial}$
    \item Timeout $\approx$ 1.02$\tau$ (slightly longer)
\end{itemize}

\textbf{For 13.5\% reference:}
\begin{itemize}
    \item Timeout = 2$\tau$
\end{itemize}

\textbf{For 5\% reference:}
\begin{itemize}
    \item Timeout = 3$\tau$
\end{itemize}

\vspace{0.15cm}

\textbf{Practical considerations:}

\textbf{Advantages of this timer:}
\begin{itemize}
    \item Simple circuit (few components)
    \item Easily adjustable (change R or C)
    \item Repeatable (consistent timing)
    \item One-shot operation (single pulse per trigger)
\end{itemize}

\textbf{Limitations:}
\begin{itemize}
    \item Not precise (component tolerances $\pm$10-20\%)
    \item Temperature affects R and C values
    \item Comparator has small offset voltage (error)
    \item For precision timing, use crystal oscillator or timer IC (555)
\end{itemize}

\textbf{Improvements:}
\begin{itemize}
    \item Use precision resistors and capacitors ($\pm$1\%)
    \item Temperature-stable components
    \item Add potentiometer for fine adjustment
    \item Use precision comparator (lower offset)
\end{itemize}

\vspace{0.15cm}

\textbf{Real-World Applications:}

\textbf{Time delay circuits:}
\begin{itemize}
    \item Turn on light 5 seconds after switch pressed
    \item Delay relay activation
    \item Staircase timer (light stays on for fixed time)
    \item Camera flash recovery time
\end{itemize}

\textbf{One-shot timers:}
\begin{itemize}
    \item Automatic shutoff (e.g., bathroom fan runs for 10 min after switch off)
    \item Timeout alarms
    \item Monostable circuits
    \item Pulse width generation
\end{itemize}

\textbf{Pulse shaping:}
\begin{itemize}
    \item Convert slow-rising signal to sharp edge
    \item Debounce mechanical switches
    \item Generate clock signals
    \item Digital signal conditioning
\end{itemize}

\vspace{0.15cm}

\textbf{Advanced Concept: Why Not Use Tricks Too Often?}

\textbf{The statement from source:}
\begin{itemize}
    \item "You try not too often to rely on tricks like this"
    \item Referring to RC delay tricks
\end{itemize}

\textbf{Why be cautious?}
\begin{itemize}
    \item Component tolerances cause timing variation
    \item Temperature changes affect timing
    \item Aging changes component values
    \item Not as reliable as crystal-based timing
    \item Hard to make very precise or very long delays
\end{itemize}

\textbf{When RC timing IS appropriate:}
\begin{itemize}
    \item Short delays (microseconds to seconds)
    \item Precision not critical ($\pm$10-20\% acceptable)
    \item Low-cost requirement
    \item Simple implementation needed
    \item Occasional/non-critical use
\end{itemize}

\textbf{When to use better alternatives:}
\begin{itemize}
    \item Need precision timing: Use crystal oscillator + counter
    \item Long delays (minutes/hours): Use timer IC (e.g., 555) or microcontroller
    \item Critical timing: Use dedicated timer chip
    \item Programmable delays: Use microcontroller
\end{itemize}
\end{detailbox}

\vspace{0.2cm}

\noindent\textbf{\color{accentcolor} Practical Examples \& Numerical Calculations}
\begin{examplebox}
\textbf{Example 1: CMOS Buffer Time Delay}

Circuit: R = 10 k$\Omega$, C = 100 $\mu$F, Buffer threshold = 2.5V (50\% of 5V)

Time constant: $\tau = 10,000 \times 100 \times 10^{-6} = 1$ second

Find delay time (time to reach 2.5V):

Using charging equation: $V_C(t) = 5(1 - e^{-t/\tau})$

Set $V_C = 2.5$V:
\begin{itemize}
    \item $2.5 = 5(1 - e^{-t/1})$
    \item $0.5 = 1 - e^{-t}$
    \item $e^{-t} = 0.5$
    \item $-t = \ln(0.5) = -0.693$
    \item $t = 0.693$ seconds $\approx$ 0.7$\tau$
\end{itemize}

Delay $\approx$ 0.7 seconds

\vspace{0.15cm}

\textbf{Example 2: Adjusting Delay Time}

Want 5-second delay, buffer threshold still 50\%

From Example 1: Delay = 0.7$\tau$

Required: $0.7\tau = 5$ seconds

Solve for $\tau$: $\tau = \frac{5}{0.7} = 7.14$ seconds

Choose C = 470 $\mu$F, find R:
\begin{itemize}
    \item $R = \frac{\tau}{C} = \frac{7.14}{470 \times 10^{-6}} = 15,190$ $\Omega$
    \item Use standard value: 15 k$\Omega$
\end{itemize}

Verification:
\begin{itemize}
    \item $\tau = 15,000 \times 470 \times 10^{-6} = 7.05$ s
    \item Delay = $0.7 \times 7.05 = 4.94$ s $\approx$ 5 s $\checkmark$
\end{itemize}

\vspace{0.15cm}

\textbf{Example 3: One-Shot Timer (1 Minute)}

Requirement: Output HIGH for 60 seconds after button press

Reference voltage: 37\% of $V+$ (for simple 1$\tau$ timeout)

Design:
\begin{itemize}
    \item Timeout = 1$\tau$ = 60 seconds
    \item Choose C = 1000 $\mu$F (large, but reasonable)
    \item $R = \frac{60}{0.001} = 60,000$ $\Omega$ = 60 k$\Omega$
\end{itemize}

Voltage divider for 37\% reference (assuming 5V supply):
\begin{itemize}
    \item $V_{ref} = 0.37 \times 5 = 1.85$V
    \item If $R_1$ = 10 k$\Omega$ (top), $R_2$ = ? (bottom)
    \item $\frac{R_2}{R_1 + R_2} = 0.37$
    \item $R_2 = \frac{0.37 \times 10k}{1 - 0.37} = 5.87$ k$\Omega$
    \item Use 5.6 k$\Omega$ or 6.2 k$\Omega$
\end{itemize}

Operation:
\begin{itemize}
    \item Button pressed: Cap charges to 5V, output goes HIGH
    \item Cap discharges: After 60s reaches 1.85V, output goes LOW
\end{itemize}

\vspace{0.15cm}

\textbf{Example 4: Timer with Different Reference}

Given: R = 47 k$\Omega$, C = 1000 $\mu$F, $V+$ = 5V

Voltage divider creates: $V_{ref}$ = 1.8V (36\% of 5V)

Calculate timeout:

Using discharge equation: $1.8 = 5 \times e^{-t/\tau}$

Solve:
\begin{itemize}
    \item $\frac{1.8}{5} = 0.36 = e^{-t/\tau}$
    \item $\ln(0.36) = -1.02 = -\frac{t}{\tau}$
    \item $t = 1.02\tau$
\end{itemize}

Time constant: $\tau = 47,000 \times 0.001 = 47$ seconds

Timeout: $t = 1.02 \times 47 = 47.94$ seconds $\approx$ 48 seconds

\vspace{0.15cm}

\textbf{Example 5: Comparator Voltage Calculation}

Comparator circuit with $V+ = 12$V

Voltage divider: $R_1$ = 10 k$\Omega$, $R_2$ = 4.7 k$\Omega$

Calculate reference voltage:
\begin{itemize}
    \item $V_{ref} = V+ \times \frac{R_2}{R_1 + R_2}$
    \item $V_{ref} = 12 \times \frac{4700}{10000 + 4700}$
    \item $V_{ref} = 12 \times \frac{4700}{14700} = 12 \times 0.32 = 3.84$V
\end{itemize}

Comparator behavior:
\begin{itemize}
    \item When (+) input $> 3.84$V: Output = 12V (HIGH)
    \item When (+) input $< 3.84$V: Output = 0V (LOW)
\end{itemize}

\vspace{0.15cm}

\textbf{Example 6: Designing for Specific Timeout with Custom Reference}

Requirement: 30-second timeout, reference at 20\% of $V+$

Find timeout multiplier for 20\% (0.2):
\begin{itemize}
    \item $0.2 = e^{-t/\tau}$
    \item $\ln(0.2) = -1.61 = -\frac{t}{\tau}$
    \item $t = 1.61\tau$
\end{itemize}

For 30-second timeout:
\begin{itemize}
    \item $1.61\tau = 30$
    \item $\tau = \frac{30}{1.61} = 18.6$ seconds
\end{itemize}

Choose C = 470 $\mu$F:
\begin{itemize}
    \item $R = \frac{18.6}{470 \times 10^{-6}} = 39,574$ $\Omega$
    \item Use 39 k$\Omega$ or 43 k$\Omega$
\end{itemize}

\vspace{0.15cm}

\textbf{Example 7: Comparing Different Reference Voltages}

Same RC: $\tau$ = 10 seconds

Different reference voltages:

50\% reference:
\begin{itemize}
    \item Timeout = $0.69\tau = 6.9$ seconds
\end{itemize}

37\% reference:
\begin{itemize}
    \item Timeout = $1.0\tau = 10$ seconds
\end{itemize}

20\% reference:
\begin{itemize}
    \item Timeout = $1.61\tau = 16.1$ seconds
\end{itemize}

10\% reference:
\begin{itemize}
    \item Timeout = $2.3\tau = 23$ seconds
\end{itemize}

Lower reference $\rightarrow$ longer timeout (capacitor takes longer to discharge to lower voltage)
\end{examplebox}

\vspace{0.2cm}

\noindent\textbf{\color{accentcolor} Key Points (Interview Focus)}
\begin{keypointsbox}
\begin{enumerate}
    \item \textbf{CMOS Buffer:} High input Z, low output Z, switches at threshold voltage
    
    \item \textbf{Time Delay:} RC charges, buffer switches when threshold reached
    
    \item \textbf{Delay Calculation:} For 50\% threshold, delay $\approx 0.7\tau$
    
    \item \textbf{Comparator:} Compares (+) and (-) inputs, output HIGH if (+)$>$(-)
    
    \item \textbf{One-Shot Timer:} Button charges cap, discharges through R, output HIGH until below reference
    
    \item \textbf{Timeout Formula:} Depends on reference voltage percentage
    \begin{itemize}
        \item 37\% reference: Timeout = 1$\tau$
        \item 50\% reference: Timeout = 0.69$\tau$
        \item Custom: Solve $V_{ref}/V_{initial} = e^{-t/\tau}$
    \end{itemize}
    
    \item \textbf{Voltage Divider:} Creates fixed reference for comparator (-) input
    
    \item \textbf{Limitations:} Component tolerance, temperature, not for precision timing
\end{enumerate}

\vspace{0.2cm}

\textbf{Interview Questions:}
\begin{itemize}
    \item \textbf{Q:} How does CMOS buffer create sharp output from gradual input? \\
    \textit{A:} Buffer has threshold. Below threshold = LOW, above = HIGH. Switches quickly even if input rises slowly.
    
    \item \textbf{Q:} How does comparator work? \\
    \textit{A:} Compares two inputs. If (+) input greater than (-) input, output goes HIGH. Otherwise LOW.
    
    \item \textbf{Q:} How to create 1-minute timer with RC and comparator? \\
    \textit{A:} Set $\tau = RC = 60$s (for 37\% reference). Button charges cap, then discharges through R. Output HIGH until voltage drops below reference.
    
    \item \textbf{Q:} Why set reference at 37\%? \\
    \textit{A:} 37\% is the voltage remaining after 1$\tau$ during discharge. Makes timeout exactly equal to time constant.
    
    \item \textbf{Q:} What's the purpose of voltage divider in timer circuit? \\
    \textit{A:} Creates fixed reference voltage for comparator (-) input. Determines when timer expires.
    
    \item \textbf{Q:} Can RC timing be very precise? \\
    \textit{A:} No. Component tolerances ($\pm$10-20\%), temperature effects, aging cause variation. Use for non-critical timing only.
\end{itemize}

\vspace{0.2cm}

\textbf{Formulas:}
\begin{itemize}
    \item \textbf{Delay (50\% threshold):} $t_{delay} \approx 0.69\tau = 0.69RC$
    \item \textbf{Timeout (37\% reference):} $t_{timeout} = 1.0\tau = RC$
    \item \textbf{General Timeout:} Solve $\frac{V_{ref}}{V_{initial}} = e^{-t/\tau}$ $\rightarrow$ $t = -\tau \ln\left(\frac{V_{ref}}{V_{initial}}\right)$
    \item \textbf{Voltage Divider:} $V_{ref} = V+ \times \frac{R_2}{R_1 + R_2}$
\end{itemize}

\vspace{0.2cm}

\textbf{Applications:}
\begin{itemize}
    \item Automatic shutoff timers (lights, fans)
    \item Delay circuits (relay activation)
    \item Pulse shaping (convert slow to fast edge)
    \item Debouncing (remove switch bounce)
    \item Monostable multivibrators
    \item Timeout alarms
\end{itemize}

\vspace{0.2cm}

\textbf{Component Notes:}
\begin{itemize}
    \item \textbf{CMOS Buffer:} Examples: 74HC04, CD4050, 74HCT14 (Schmitt trigger)
    \item \textbf{Comparator:} Examples: LM311, LM393, LM339
    \item Comparator $\neq$ Op-amp (though op-amp can work as comparator in simple cases)
    \item Op-amp covered in detail later in course
\end{itemize}
\end{keypointsbox}

