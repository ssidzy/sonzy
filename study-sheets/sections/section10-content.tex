% ====================================================================
% SECTION 10: INDUCTORS
% ====================================================================

\section*{\LARGE\color{headercolor} Section 10 -- Inductors}
\addcontentsline{toc}{section}{Section 10: Inductors}

% --------------------------------------------------------------------
\subsection{Inductor -- Introduction}

\noindent\textbf{\color{accentcolor} TL;DR (The Gist)}
\begin{tldrbox}
\begin{itemize}
    \item \textbf{Construction:} Coil of wire (solenoid) that creates magnetic field
    \item \textbf{Inductance:} L in Henries (H), typically $\mu$H or mH
    \item \textbf{Opposes current changes:} $V = L\frac{dI}{dt}$ (voltage proportional to rate of current change)
    \item \textbf{Dual of capacitor:} Stores energy in magnetic field, opposes current changes (C opposes voltage changes)
\end{itemize}
\end{tldrbox}

\vspace{0.2cm}

\noindent\textbf{\color{accentcolor} Detailed Explanation}
\begin{detailbox}
\textbf{What is an Inductor?}

\vspace{0.15cm}

\textbf{Physical construction:}
\begin{itemize}
    \item Simple length of wire coiled up
    \item Coil shape called solenoid
    \item Current through wire creates magnetic field
    \item Coil amplifies magnetic field strength
    \item Symbol shows coil loops (lines = core material)
\end{itemize}

\textbf{Why coil shape?}
\begin{itemize}
    \item Straight wire: Weak magnetic field (negligible unless mega-amps)
    \item Coiled wire: Magnetic fields from each loop add up
    \item Result: Much stronger total magnetic field
    \item More turns = stronger field
\end{itemize}

\vspace{0.15cm}

\textbf{Inductance -- The Key Property:}

\textbf{Self-induction mechanism:}
\begin{enumerate}
    \item Voltage applied $\rightarrow$ current flows $\rightarrow$ magnetic field created
    \item Changing magnetic field $\rightarrow$ induces voltage back (Lenz's Law)
    \item Induced voltage (back EMF) opposes the change
    \item Result: Current cannot change rapidly!
\end{enumerate}

\textbf{Inductor equation:}
\begin{itemize}
    \item $V_L = L \frac{dI}{dt}$
    \item Voltage across inductor proportional to rate of current change
    \item Fast current change $\rightarrow$ large voltage
    \item Steady current (dI/dt=0) $\rightarrow$ zero voltage
\end{itemize}

\vspace{0.15cm}

\textbf{Inductor vs. Capacitor (Dual Relationship):}

\begin{center}
\begin{tabular}{|l|l|l|}
\hline
\textbf{Property} & \textbf{Capacitor} & \textbf{Inductor} \\
\hline
Stores energy in & Electric field & Magnetic field \\
\hline
Opposes changes in & Voltage & Current \\
\hline
Key equation & $I = C\frac{dV}{dt}$ & $V = L\frac{dI}{dt}$ \\
\hline
Energy formula & $E = \frac{1}{2}CV^2$ & $E = \frac{1}{2}LI^2$ \\
\hline
DC behavior & Blocks (open) & Passes (wire) \\
\hline
AC behavior & Passes (low $X_C$) & Blocks (high $X_L$) \\
\hline
Phase shift & I leads V by 90° & V leads I by 90° \\
\hline
\end{tabular}
\end{center}

\vspace{0.15cm}

\textbf{Behavior in Circuits:}

\textbf{DC circuit (constant current):}
\begin{itemize}
    \item Initially: High voltage, zero current (opposes change)
    \item Gradually: Current rises exponentially
    \item Eventually: Acts like wire (zero resistance)
    \item Final state: $V_L = 0$, current limited only by R
\end{itemize}

\textbf{AC circuit (changing current):}
\begin{itemize}
    \item Continuously opposes current changes
    \item Creates inductive reactance $X_L = 2\pi fL$
    \item Higher frequency $\rightarrow$ higher opposition
    \item Useful in filters, transformers, RF circuits
\end{itemize}

\vspace{0.15cm}

\textbf{Inductance Unit:}

\textbf{Henry (H):}
\begin{itemize}
    \item Named after Joseph Henry
    \item 1 H = 1 V·s/A (volt-second per ampere)
    \item Henry is very large unit
    \item Practical: mH ($10^{-3}$ H), $\mu$H ($10^{-6}$ H), nH ($10^{-9}$ H)
\end{itemize}

\textbf{Typical ranges:}
\begin{itemize}
    \item nH: RF circuits, chip inductors
    \item $\mu$H: Switching converters, filters (1-1000 $\mu$H)
    \item mH: Audio equipment, power supplies
    \item H: Large transformers, electromagnets
\end{itemize}

\vspace{0.15cm}

\textbf{Where Inductors are Used:}

\textbf{Applications:}
\begin{itemize}
    \item Transformers (voltage conversion)
    \item Filters (with capacitors)
    \item Energy storage (switching regulators)
    \item RF circuits (tuning, matching)
    \item EMI suppression (chokes)
    \item Motors and solenoids
\end{itemize}
\end{detailbox}

\vspace{0.2cm}

\noindent\textbf{\color{accentcolor} Practical Examples \& Numerical Calculations}
\begin{examplebox}
\textbf{Example 1: Voltage Across Inductor During Current Change}

Given: L = 10 mH, current rises from 0 to 1A in 0.1 seconds (linear)

Calculate voltage:
\begin{itemize}
    \item $\frac{dI}{dt} = \frac{1A - 0A}{0.1s} = 10$ A/s
    \item $V_L = L \frac{dI}{dt} = 0.01 \times 10 = 0.1$~V = 100 mV
\end{itemize}

If same change in 0.01s (10$\times$ faster):
\begin{itemize}
    \item $\frac{dI}{dt} = 100$ A/s
    \item $V_L = 0.01 \times 100 = 1$~V (10$\times$ higher!)
\end{itemize}

\vspace{0.15cm}

\textbf{Example 2: Steady Current (DC)}

Given: L = 50 $\mu$H, constant DC current I = 2A

Calculate voltage:
\begin{itemize}
    \item $\frac{dI}{dt} = 0$ (steady current, no change)
    \item $V_L = L \times 0 = 0$~V
    \item Inductor acts like wire in DC steady state!
\end{itemize}

\vspace{0.15cm}

\textbf{Example 3: Energy Stored in Inductor}

Given: L = 100 mH, I = 500 mA

Energy formula: $E = \frac{1}{2}LI^2$

Calculate:
\begin{itemize}
    \item $E = \frac{1}{2} \times 0.1 \times (0.5)^2$
    \item $E = 0.05 \times 0.25 = 0.0125$~J = 12.5 mJ
\end{itemize}

Double current to 1A:
\begin{itemize}
    \item $E = \frac{1}{2} \times 0.1 \times 1^2 = 0.05$~J = 50 mJ
    \item Energy quadruples (E $\propto$ I²)!
\end{itemize}

\vspace{0.15cm}

\textbf{Example 4: Comparison with Capacitor}

Capacitor: C = 100 $\mu$F, V = 5V
\begin{itemize}
    \item Energy: $E_C = \frac{1}{2} \times 0.0001 \times 25 = 1.25$ mJ
    \item Stores energy in electric field between plates
\end{itemize}

Inductor: L = 100 mH, I = 50 mA
\begin{itemize}
    \item Energy: $E_L = \frac{1}{2} \times 0.1 \times 0.0025 = 0.125$ mJ
    \item Stores energy in magnetic field around coil
\end{itemize}

Different energy storage mechanism, similar component size!
\end{examplebox}

\vspace{0.2cm}

\noindent\textbf{\color{accentcolor} Key Points (Interview Focus)}
\begin{keypointsbox}
\begin{enumerate}
    \item \textbf{Construction:} Coil of wire (solenoid), creates magnetic field
    
    \item \textbf{Opposes current changes:} $V_L = L\frac{dI}{dt}$ (fundamental equation)
    
    \item \textbf{DC behavior:} Eventually acts like wire ($V_L=0$ at steady state)
    
    \item \textbf{AC behavior:} Continuously opposes $\rightarrow$ creates reactance
    
    \item \textbf{Energy storage:} $E = \frac{1}{2}LI^2$ in magnetic field
    
    \item \textbf{Unit:} Henry (H), typically $\mu$H or mH in practice
    
    \item \textbf{Dual of capacitor:}
    \begin{itemize}
        \item Capacitor: Opposes V changes, stores in E-field
        \item Inductor: Opposes I changes, stores in B-field
    \end{itemize}
\end{enumerate}

\vspace{0.2cm}

\textbf{Interview Questions:}
\begin{itemize}
    \item \textbf{Q:} What is an inductor? \\
    \textit{A:} Coil of wire that stores energy in magnetic field and opposes changes in current.
    
    \item \textbf{Q:} Why does inductor oppose current changes? \\
    \textit{A:} Changing current creates changing magnetic field, which induces voltage (back EMF) that opposes the change (Lenz's Law).
    
    \item \textbf{Q:} How does inductor behave in DC vs AC? \\
    \textit{A:} DC steady state: acts like wire (zero V). AC: opposes with reactance, blocks high frequencies.
    
    \item \textbf{Q:} What's the difference between inductor and capacitor? \\
    \textit{A:} Inductor opposes current changes/stores in magnetic field. Capacitor opposes voltage changes/stores in electric field.
\end{itemize}

\vspace{0.2cm}

\textbf{Formulas:}
\begin{itemize}
    \item \textbf{Voltage:} $V_L = L\frac{dI}{dt}$
    \item \textbf{Energy:} $E = \frac{1}{2}LI^2$
    \item \textbf{Reactance:} $X_L = 2\pi fL$ (covered in detail later)
\end{itemize}
\end{keypointsbox}

% --------------------------------------------------------------------
% NOTE: Topics 2-13 consolidated for efficiency while maintaining comprehensive coverage
% Following topics build systematically: construction$\rightarrow$behavior$\rightarrow$applications
% --------------------------------------------------------------------

\subsection{Types of Inductors \& Construction}

\noindent\textbf{\color{accentcolor} TL;DR (The Gist)}
\begin{tldrbox}
\begin{itemize}
    \item \textbf{Inductance formula:} $L = \frac{\mu N^2 A}{l}$ (permeability, turns, area, length)
    \item \textbf{Air core:} Low L ($\mu$H range), RF circuits, high frequency
    \item \textbf{Iron core:} High L (mH-H), transformers, audio, low frequency
    \item \textbf{Ferrite core:} Most common, adjustable permeability, gray/black color
\end{itemize}
\end{tldrbox}

\vspace{0.2cm}

\noindent\textbf{\color{accentcolor} Detailed Explanation}
\begin{detailbox}
\textbf{Inductance Formula:}

$L = \frac{\mu N^2 A}{l}$ where:
\begin{itemize}
    \item L = inductance (H)
    \item $\mu = \mu_0 \mu_r$ (permeability) - how easily magnetic field forms
    \item N = number of turns
    \item A = cross-sectional area (m²)
    \item l = length of coil (m)
\end{itemize}

\textbf{Increasing inductance:}
\begin{itemize}
    \item More turns (N²!) $\rightarrow$ much higher L
    \item Larger area $\rightarrow$ higher L
    \item Shorter length $\rightarrow$ higher L  
    \item Higher permeability core $\rightarrow$ higher L
\end{itemize}

\vspace{0.15cm}

\textbf{Air Core Inductors:}
\begin{itemize}
    \item Core material: Air ($\mu_r \approx 1$, low permeability)
    \item Low inductance (typically <5 $\mu$H)
    \item Fast current rise time
    \item High frequency capable (MHz-GHz)
    \item Applications: RF circuits, oscillators, antennas
    \item No core losses, no saturation
\end{itemize}

\textbf{Iron Core Inductors:}
\begin{itemize}
    \item Core material: Iron ($\mu_r = 200$-5000)
    \item High inductance (mH to H range)
    \item Large, heavy construction
    \item Low frequency use (50/60 Hz power)
    \item Applications: Transformers, audio equipment, filters
    \item Core losses at high frequency (eddy currents, hysteresis)
\end{itemize}

\textbf{Ferrite Core Inductors:}
\begin{itemize}
    \item Core: Iron oxide powder + epoxy resin
    \item Gray/black color, brittle material
    \item Most widely used type
    \item Permeability controllable (ratio of ferrite to epoxy)
    \item Medium inductance ($\mu$H to mH)
    \item Applications: Switching supplies, EMI filters, chokes
    \item Lower losses than iron at higher frequencies
\end{itemize}
\end{detailbox}

\vspace{0.2cm}

\noindent\textbf{\color{accentcolor} Key Points (Interview Focus)}
\begin{keypointsbox}
\begin{enumerate}
    \item \textbf{Formula:} $L = \frac{\mu N^2 A}{l}$ - inductance proportional to turns squared
    \item \textbf{Air core:} Low L, RF/high frequency
    \item \textbf{Iron core:} High L, power/audio/transformers
    \item \textbf{Ferrite core:} Most common, adjustable, versatile
    \item \textbf{Core material:} Dramatically affects inductance ($\mu_r$ factor)
\end{enumerate}

\textbf{Interview Q:}
\textit{Q: Why use core materials?} \\
A: Increase permeability ($\mu$) $\rightarrow$ much higher inductance in same physical size.
\end{keypointsbox}

% --------------------------------------------------------------------
\subsection{Inductors in Series and Parallel}

\noindent\textbf{\color{accentcolor} TL;DR (The Gist)}
\begin{tldrbox}
\begin{itemize}
    \item \textbf{Series:} $L_{total} = L_1 + L_2 + L_3 + ...$ (like resistors)
    \item \textbf{Parallel:} $\frac{1}{L_{total}} = \frac{1}{L_1} + \frac{1}{L_2} + ...$ (opposite of capacitors!)
    \item Inductors behave exactly opposite to capacitors in series/parallel
\end{itemize}
\end{tldrbox}

\vspace{0.2cm}

\noindent\textbf{\color{accentcolor} Practical Examples}
\begin{examplebox}
\textbf{Series:} 10 $\mu$H + 15 $\mu$H = 25 $\mu$H

\textbf{Parallel:} Two 10 $\mu$H $\rightarrow$ $L = \frac{10 \times 10}{10 + 10} = 5$ $\mu$H
\end{examplebox}

% --------------------------------------------------------------------
\subsection{Inductor Behavior in DC Circuit \& RL Time Constant}

\noindent\textbf{\color{accentcolor} TL;DR (The Gist)}
\begin{tldrbox}
\begin{itemize}
    \item \textbf{Initially:} $V_L$ high, I=0 (opposes sudden current)
    \item \textbf{Exponential rise:} $I(t) = \frac{V}{R}(1 - e^{-t/\tau})$ where $\tau = \frac{L}{R}$
    \item \textbf{After 5$\tau$:} Acts like wire, I = V/R
    \item Opposite behavior to capacitor (which starts with high I, zero V)
\end{itemize}
\end{tldrbox}

\vspace{0.2cm}

\noindent\textbf{\color{accentcolor} Detailed Explanation}
\begin{detailbox}
\textbf{RL Time Constant:}

$\tau = \frac{L}{R}$ (in seconds)

\textbf{Current rise (charging):}
\begin{itemize}
    \item At t=0: I=0, $V_L = V_{supply}$ (maximum)
    \item At t=1$\tau$: I=63\% of final
    \item At t=5$\tau$: I=99\% of final $\approx$ V/R
\end{itemize}

\textbf{Current decay (discharging):}
\begin{itemize}
    \item At t=0: $I=I_0$, $V_L = -V$ (reverse polarity!)
    \item At t=1$\tau$: I=37\% remains
    \item At t=5$\tau$: $I\approx 0$
    \item Direction unchanged (unlike capacitor)
\end{itemize}

\textbf{Energy storage:} $E = \frac{1}{2}LI^2$ in magnetic field
\end{detailbox}

\vspace{0.2cm}

\noindent\textbf{\color{accentcolor} Key Points (Interview Focus)}
\begin{keypointsbox}
\begin{itemize}
    \item \textbf{Time constant:} $\tau = \frac{L}{R}$ (larger L or smaller R $\rightarrow$ slower)
    \item \textbf{Opposes changes:} Resists sudden current changes
    \item \textbf{Steady DC:} Acts like wire ($V_L=0$)
    \item \textbf{vs. Capacitor:}
    \begin{itemize}
        \item Cap: $\tau = RC$, high I initially
        \item Ind: $\tau = L/R$, high V initially
    \end{itemize}
\end{itemize}

\textbf{Interview Q:}
\textit{Q: What happens when switch closes in RL circuit?} \\
A: Current rises exponentially from 0 to V/R over ~5$\tau$. Inductor opposes rapid change.
\end{keypointsbox}

% --------------------------------------------------------------------
\subsection{Back EMF and Protection (Flyback Diode)}

\noindent\textbf{\color{accentcolor} TL;DR (The Gist)}
\begin{tldrbox}
\begin{itemize}
    \item \textbf{Back EMF:} When current interrupted $\rightarrow$ magnetic field collapses $\rightarrow$ huge voltage spike
    \item \textbf{Danger:} Can destroy transistors/components (hundreds of volts!)
    \item \textbf{Solution:} Flyback diode in parallel (reverse biased normally)
    \item Critical for motors, relays, solenoids
\end{itemize}
\end{tldrbox}

\vspace{0.2cm}

\noindent\textbf{\color{accentcolor} Detailed Explanation}
\begin{detailbox}
\textbf{Back EMF Mechanism:}

\begin{enumerate}
    \item Current flowing $\rightarrow$ magnetic field built up
    \item Switch opens $\rightarrow$ current stops rapidly
    \item dI/dt very large (fast change!)
    \item $V_L = L\frac{dI}{dt}$ $\rightarrow$ huge voltage (kV possible!)
    \item Polarity reversed (opposes current decrease)
\end{enumerate}

\textbf{Flyback Diode Protection:}

\textbf{Connection:} Diode in parallel with inductor/motor, reverse biased

\textbf{Normal operation (switch on):}
\begin{itemize}
    \item Diode reverse biased $\rightarrow$ no effect
    \item Current flows through inductor normally
\end{itemize}

\textbf{Switch off (back EMF):}
\begin{itemize}
    \item Inductor polarity reverses
    \item Diode becomes forward biased
    \item Provides path for inductor current
    \item Energy dissipates safely through diode
    \item Voltage limited to ~0.7V (diode drop)
\end{itemize}

\textbf{Applications:}
\begin{itemize}
    \item DC motors (essential!)
    \item Relays and solenoids
    \item Inductive loads driven by transistors
    \item Any switching of inductive load
\end{itemize}
\end{detailbox}

\vspace{0.2cm}

\noindent\textbf{\color{accentcolor} Key Points (Interview Focus)}
\begin{keypointsbox}
\begin{enumerate}
    \item \textbf{Back EMF:} Collapsing field $\rightarrow$ large dI/dt $\rightarrow$ huge voltage spike
    \item \textbf{Danger:} Can exceed transistor breakdown voltage
    \item \textbf{Flyback diode:} Reverse biased normally, conducts during spike
    \item \textbf{Must use:} With motors, relays, any inductive switching
    \item \textbf{Diode polarity:} Cathode to +supply, anode to ground (reverse of normal)
\end{enumerate}

\textbf{Interview Q:}
\textit{Q: Why does turning off motor destroy transistor?} \\
A: Motor inductance creates huge back EMF spike when current interrupted. Flyback diode prevents this.
\end{keypointsbox}

% --------------------------------------------------------------------
\subsection{Inductor in AC Circuit (Phase Relationship)}

\noindent\textbf{\color{accentcolor} TL;DR (The Gist)}
\begin{tldrbox}
\begin{itemize}
    \item \textbf{Phase shift:} Voltage leads current by 90° (ELI mnemonic)
    \item \textbf{Opposite of capacitor:} Cap has I leads V, Inductor has V leads I
    \item \textbf{Reactive component:} Stores/releases energy each cycle
    \item Energy oscillates in magnetic field
\end{itemize}
\end{tldrbox}

\vspace{0.2cm}

\noindent\textbf{\color{accentcolor} Detailed Explanation}
\begin{detailbox}
\textbf{90° Phase Shift Explanation:}

\textbf{ELI the ICE man mnemonic:}
\begin{itemize}
    \item \textbf{ELI:} In inductors (L), voltage (E) leads current (I)
    \item \textbf{ICE:} In capacitors (C), current (I) leads voltage (E)
\end{itemize}

\textbf{Why voltage leads in inductor?}
\begin{itemize}
    \item $V_L = L\frac{dI}{dt}$ (voltage depends on rate of change)
    \item When I crosses zero: dI/dt is maximum $\rightarrow$ $V_L$ is maximum
    \item When I is at peak: dI/dt is zero $\rightarrow$ $V_L$ is zero
    \item Result: $V_L$ peaks 90° before I peaks
\end{itemize}

\textbf{Comparison table:}

\begin{center}
\begin{tabular}{|l|l|l|}
\hline
\textbf{Property} & \textbf{Resistor} & \textbf{Inductor} \\
\hline
Phase shift & 0° (V and I together) & 90° (V leads I) \\
Energy & Dissipated (heat) & Stored/released (no loss) \\
Frequency effect & None & Higher f $\rightarrow$ more opposition \\
\hline
\end{tabular}
\end{center}
\end{detailbox}

\vspace{0.2cm}

\noindent\textbf{\color{accentcolor} Key Points (Interview Focus)}
\begin{keypointsbox}
\begin{enumerate}
    \item \textbf{Phase:} Voltage leads current by 90° (ELI mnemonic)
    \item \textbf{Reactive:} Stores energy, doesn't dissipate
    \item \textbf{vs. Capacitor:} Opposite phase (ICE vs ELI)
    \item \textbf{Application:} Phase shift circuits, power factor correction
\end{enumerate}

\textbf{Interview Q:}
\textit{Q: What does "reactive" mean for inductor?} \\
A: Opposes current changes, stores energy in magnetic field, creates 90° phase shift. No real power dissipation.
\end{keypointsbox}

% --------------------------------------------------------------------
\subsection{Inductive Reactance $X_L = 2\pi fL$}

\noindent\textbf{\color{accentcolor} TL;DR (The Gist)}
\begin{tldrbox}
\begin{itemize}
    \item \textbf{Formula:} $X_L = 2\pi fL$ (ohms)
    \item \textbf{Direct relationship:} Higher f $\rightarrow$ higher $X_L$ (blocks HF)
    \item \textbf{DC (f=0):} $X_L = 0$ (wire)
    \item \textbf{Opposite of capacitor:} $X_C = \frac{1}{2\pi fC}$ decreases with f
\end{itemize}
\end{tldrbox}

\vspace{0.2cm}

\noindent\textbf{\color{accentcolor} Detailed Explanation}
\begin{detailbox}
\textbf{Inductive Reactance:}

$X_L = 2\pi fL$ where:
\begin{itemize}
    \item $X_L$ = inductive reactance ($\Omega$)
    \item f = frequency (Hz)
    \item L = inductance (H)
\end{itemize}

\textbf{Frequency dependence:}
\begin{itemize}
    \item \textbf{Low frequency:} Low $X_L$ $\rightarrow$ passes easily
    \item \textbf{High frequency:} High $X_L$ $\rightarrow$ blocks
    \item \textbf{DC (f=0):} $X_L = 0$ $\rightarrow$ acts like wire
    \item \textbf{f$\rightarrow$$\infty$:} $X_L \rightarrow \infty$ $\rightarrow$ open circuit
\end{itemize}

\textbf{Comparison with capacitor:}

\begin{center}
\begin{tabular}{|l|l|l|}
\hline
\textbf{Frequency} & \textbf{$X_C$} & \textbf{$X_L$} \\
\hline
DC (0 Hz) & $\infty$ (blocks) & 0 (passes) \\
Low freq & High (blocks) & Low (passes) \\
High freq & Low (passes) & High (blocks) \\
\hline
\end{tabular}
\end{center}

\textbf{Impedance in RL circuit:}

$Z = \sqrt{R^2 + X_L^2}$ (NOT $R + X_L$!)

\textbf{Current calculation:}

$I = \frac{V_{rms}}{Z}$
\end{detailbox}

\vspace{0.2cm}

\noindent\textbf{\color{accentcolor} Practical Examples}
\begin{examplebox}
\textbf{Example 1:} L = 30 mH at 10 kHz

$X_L = 2\pi \times 10000 \times 0.03 = 1885$ $\Omega$

At 1 kHz: $X_L = 188$ $\Omega$ (10$\times$ lower frequency $\rightarrow$ 10$\times$ lower reactance)

\vspace{0.15cm}

\textbf{Example 2:} L = 400 mH, R = 200 $\Omega$, f = 200 Hz

$X_L = 2\pi \times 200 \times 0.4 = 502.4$ $\Omega$

$Z = \sqrt{200^2 + 502.4^2} = \sqrt{40000 + 252405} = 540.8$ $\Omega$

If V = 5V: $I_{rms} = \frac{5}{540.8} = 9.2$ mA
\end{examplebox}

\vspace{0.2cm}

\noindent\textbf{\color{accentcolor} Key Points (Interview Focus)}
\begin{keypointsbox}
\begin{enumerate}
    \item \textbf{Formula:} $X_L = 2\pi fL$ (directly proportional to f and L)
    \item \textbf{Frequency effect:} Higher f $\rightarrow$ higher $X_L$ $\rightarrow$ more opposition
    \item \textbf{DC behavior:} f=0 $\rightarrow$ $X_L=0$ $\rightarrow$ inductor is wire
    \item \textbf{vs. Capacitor:} Opposite frequency dependence
    \item \textbf{Impedance:} $Z = \sqrt{R^2 + X_L^2}$ (vector addition)
\end{enumerate}

\textbf{Interview Q:}
\textit{Q: How does inductor behave at very low vs very high frequency?} \\
A: Low freq (DC): $X_L\approx 0$, acts like wire. High freq: $X_L$ very high, blocks AC.

\textbf{Formulas:}
\begin{itemize}
    \item $X_L = 2\pi fL = \omega L$
    \item $Z = \sqrt{R^2 + X_L^2}$
    \item $I = \frac{V}{Z}$
\end{itemize}
\end{keypointsbox}

% --------------------------------------------------------------------
\subsection{Series RLC Resonance}

\noindent\textbf{\color{accentcolor} TL;DR (The Gist)}
\begin{tldrbox}
\begin{itemize}
    \item \textbf{Resonance:} $X_L = X_C$ (cancel out!), Z = R only (minimum)
    \item \textbf{Resonant frequency:} $f_r = \frac{1}{2\pi\sqrt{LC}}$
    \item \textbf{At resonance:} Huge current/voltage, energy oscillates L$\leftrightarrow$C
    \item \textbf{Danger:} Can destroy components if unintended!
\end{itemize}
\end{tldrbox}

\vspace{0.2cm}

\noindent\textbf{\color{accentcolor} Detailed Explanation}
\begin{detailbox}
\textbf{Resonance Condition:}

At specific frequency, $X_L = X_C$:

$2\pi f_r L = \frac{1}{2\pi f_r C}$

Solving: $f_r = \frac{1}{2\pi\sqrt{LC}}$

\textbf{At resonance:}
\begin{itemize}
    \item Reactances cancel: $X_L - X_C = 0$
    \item Impedance: $Z = R$ (minimum!)
    \item Current maximum: $I = \frac{V}{R}$
    \item Circuit acts purely resistive
    \item Huge voltages across L and C (Q factor amplification)
\end{itemize}

\textbf{Energy oscillation:}
\begin{itemize}
    \item Energy transfers L $\leftrightarrow$ C each half cycle
    \item Magnetic field (L) $\leftrightarrow$ Electric field (C)
    \item Harmonic oscillator
    \item Only R dissipates energy
\end{itemize}

\textbf{Applications:}
\begin{itemize}
    \item Radio tuning (select frequency)
    \item Filters (bandpass)
    \item Oscillators
    \item Matching networks
\end{itemize}

\textbf{Danger:}
\begin{itemize}
    \item Accidental resonance in circuits
    \item Even PCB traces have small L, stray C
    \item Can cause oscillations, component damage
    \item Must consider in high-frequency design
\end{itemize}
\end{detailbox}

\vspace{0.2cm}

\noindent\textbf{\color{accentcolor} Practical Examples}
\begin{examplebox}
\textbf{Example:} L = 100 mH, C = 100 $\mu$F

Resonant frequency:
\begin{itemize}
    \item $f_r = \frac{1}{2\pi\sqrt{0.1 \times 0.0001}}$
    \item $f_r = \frac{1}{2\pi\sqrt{10^{-5}}} = \frac{1}{2\pi \times 0.00316}$
    \item $f_r = \frac{1}{0.01987} = 50.3$ Hz
\end{itemize}

At 50.3 Hz: $X_L = X_C$, they cancel, Z = R only!
\end{examplebox}

\vspace{0.2cm}

\noindent\textbf{\color{accentcolor} Key Points (Interview Focus)}
\begin{keypointsbox}
\begin{enumerate}
    \item \textbf{Resonance:} $f_r = \frac{1}{2\pi\sqrt{LC}}$ where $X_L = X_C$
    \item \textbf{Impedance:} Minimum (Z=R), maximum current
    \item \textbf{Energy:} Oscillates between L and C fields
    \item \textbf{Applications:} Tuning, filters, oscillators
    \item \textbf{Caution:} Unintended resonance can damage circuits
\end{enumerate}

\textbf{Interview Q:}
\textit{Q: What happens at resonance in RLC circuit?} \\
A: $X_L=X_C$ cancel, Z=R minimum, huge current flows, energy oscillates between inductor and capacitor.
\end{keypointsbox}

% --------------------------------------------------------------------
\subsection{Transformers (Step-Up, Step-Down, Isolation)}

\noindent\textbf{\color{accentcolor} TL;DR (The Gist)}
\begin{tldrbox}
\begin{itemize}
    \item \textbf{Construction:} Two inductors on shared iron/ferrite core
    \item \textbf{Voltage ratio:} $\frac{V_s}{V_p} = \frac{N_s}{N_p}$ (turns ratio)
    \item \textbf{Current ratio:} $\frac{I_s}{I_p} = \frac{N_p}{N_s}$ (inverse!)
    \item \textbf{Power conserved:} $V_pI_p \approx V_sI_s$ (ideal)
    \item \textbf{AC only:} DC doesn't work (no changing field)
\end{itemize}
\end{tldrbox}

\vspace{0.2cm}

\noindent\textbf{\color{accentcolor} Detailed Explanation}
\begin{detailbox}
\textbf{Transformer Principle:}

\begin{enumerate}
    \item AC current in primary $\rightarrow$ changing magnetic field in core
    \item Changing field links to secondary coil
    \item Induces voltage in secondary (Faraday's Law)
    \item Energy transferred via magnetic field (no electrical connection!)
\end{enumerate}

\textbf{Key Equations:}

\textbf{Voltage transformation:}
\begin{itemize}
    \item $\frac{V_{secondary}}{V_{primary}} = \frac{N_{secondary}}{N_{primary}}$
    \item More turns $\rightarrow$ higher voltage
    \item Fewer turns $\rightarrow$ lower voltage
\end{itemize}

\textbf{Current transformation:}
\begin{itemize}
    \item $\frac{I_{secondary}}{I_{primary}} = \frac{N_{primary}}{N_{secondary}}$
    \item Inverse relationship!
    \item Higher voltage $\rightarrow$ lower current (and vice versa)
\end{itemize}

\textbf{Power conservation:}
\begin{itemize}
    \item Ideal: $P_{out} = P_{in}$
    \item $V_s I_s = V_p I_p$
    \item If V doubles, I halves
    \item Real: 90-99\% efficient (core losses, copper losses)
\end{itemize}

\vspace{0.15cm}

\textbf{Three Types:}

\textbf{1. Step-Down Transformer:}
\begin{itemize}
    \item $N_s < N_p$ (fewer secondary turns)
    \item $V_s < V_p$ (lower output voltage)
    \item $I_s > I_p$ (higher output current)
    \item Example: 120V AC $\rightarrow$ 12V AC (10:1 ratio)
    \item Use: Power supplies, voltage reduction
\end{itemize}

\textbf{2. Step-Up Transformer:}
\begin{itemize}
    \item $N_s > N_p$ (more secondary turns)
    \item $V_s > V_p$ (higher output voltage)
    \item $I_s < I_p$ (lower output current)
    \item Example: 120V $\rightarrow$ 10kV (transmission lines)
    \item Use: Power transmission, voltage multiplication
\end{itemize}

\textbf{3. Isolation Transformer:}
\begin{itemize}
    \item $N_s = N_p$ (same turns)
    \item $V_s = V_p$ (same voltage)
    \item $I_s = I_p$ (same current)
    \item Purpose: Galvanic isolation, safety, break ground loops
    \item No electrical connection between primary/secondary
    \item Prevents DC, capacitive coupling
\end{itemize}

\vspace{0.15cm}

\textbf{Why AC Only?}

\textbf{AC (works):}
\begin{itemize}
    \item Constantly changing current
    \item Constantly changing magnetic field
    \item Changing field induces voltage
\end{itemize}

\textbf{DC (doesn't work):}
\begin{itemize}
    \item Constant current
    \item Static magnetic field (after initial transient)
    \item No change $\rightarrow$ no induced voltage
    \item Secondary voltage = 0
\end{itemize}
\end{detailbox}

\vspace{0.2cm}

\noindent\textbf{\color{accentcolor} Practical Examples}
\begin{examplebox}
\textbf{Example 1: Step-Down}

Primary: 120V, 100 turns \\
Secondary: 10 turns

Voltage ratio:
\begin{itemize}
    \item $\frac{V_s}{120} = \frac{10}{100} = 0.1$
    \item $V_s = 12$~V
\end{itemize}

If load draws 2A from secondary:
\begin{itemize}
    \item $\frac{I_s}{I_p} = \frac{N_p}{N_s} = \frac{100}{10} = 10$
    \item $I_p = \frac{2}{10} = 0.2$~A
\end{itemize}

Power: $P_s = 12 \times 2 = 24$~W, $P_p = 120 \times 0.2 = 24$~W $\checkmark$

\vspace{0.15cm}

\textbf{Example 2: Step-Up}

Primary: 230V, 50 turns \\
Secondary: 500 turns (10:1 ratio)

Voltage: $V_s = 230 \times \frac{500}{50} = 2300$~V

If primary draws 10A:
\begin{itemize}
    \item $I_s = 10 \times \frac{50}{500} = 1$~A
    \item Higher voltage, lower current!
\end{itemize}
\end{examplebox}

\vspace{0.2cm}

\noindent\textbf{\color{accentcolor} Key Points (Interview Focus)}
\begin{keypointsbox}
\begin{enumerate}
    \item \textbf{Voltage:} $\frac{V_s}{V_p} = \frac{N_s}{N_p}$ (proportional to turns)
    
    \item \textbf{Current:} $\frac{I_s}{I_p} = \frac{N_p}{N_s}$ (inverse!)
    
    \item \textbf{Power:} Conserved (ideal): $V_pI_p = V_sI_s$
    
    \item \textbf{Step-down:} $N_s < N_p$, lower V, higher I
    
    \item \textbf{Step-up:} $N_s > N_p$, higher V, lower I
    
    \item \textbf{Isolation:} $N_s = N_p$, same V and I, breaks ground
    
    \item \textbf{AC only:} Requires changing field to induce voltage
    
    \item \textbf{Symbol:} Two inductors facing each other with core lines
\end{enumerate}

\vspace{0.2cm}

\textbf{Interview Questions:}
\begin{itemize}
    \item \textbf{Q:} How does transformer work? \\
    \textit{A:} AC in primary creates changing magnetic field in core. This induces voltage in secondary via electromagnetic induction.
    
    \item \textbf{Q:} Why doesn't transformer work with DC? \\
    \textit{A:} DC creates static magnetic field (no change). No changing field = no induced voltage.
    
    \item \textbf{Q:} If transformer steps up voltage, what happens to current? \\
    \textit{A:} Current decreases (inverse ratio). Power is conserved: higher V $\times$ lower I = same power.
    
    \item \textbf{Q:} What's the purpose of isolation transformer? \\
    \textit{A:} Breaks electrical connection while transferring power. Prevents ground loops, improves safety, blocks DC.
\end{itemize}

\vspace{0.2cm}

\textbf{Applications:}
\begin{itemize}
    \item Power supplies (step-down 120V$\rightarrow$12V)
    \item Power transmission (step-up to kV range)
    \item Audio equipment (impedance matching)
    \item Isolation (safety, medical equipment)
    \item Voltage multiplication (high voltage generation)
\end{itemize}
\end{keypointsbox}

% --------------------------------------------------------------------
\subsection{Long-Distance Power Transmission}

\noindent\textbf{\color{accentcolor} TL;DR (The Gist)}
\begin{tldrbox}
\begin{itemize}
    \item \textbf{Problem:} Power loss in wires = $P_{loss} = I^2R$
    \item \textbf{Solution:} High voltage $\rightarrow$ low current for same power
    \item \textbf{Example:} 100kW at 1kV loses 10kW, at 10kV loses only 0.1kW (100$\times$ better!)
    \item Transformers enable efficient long-distance transmission
\end{itemize}
\end{tldrbox}

\vspace{0.2cm}

\noindent\textbf{\color{accentcolor} Detailed Explanation}
\begin{detailbox}
\textbf{Power Transmission Problem:}

\textbf{Power loss in conductors:}
\begin{itemize}
    \item $P_{loss} = I^2 R_{wire}$
    \item Loss proportional to current squared!
    \item Long wires have significant resistance
    \item High current $\rightarrow$ enormous losses
\end{itemize}

\textbf{Solution -- High Voltage Transmission:}

For same power: $P = VI$

\textbf{Low voltage transmission:}
\begin{itemize}
    \item High current needed
    \item $I^2R$ losses huge
    \item Example: 100kW at 1kV $\rightarrow$ 100A $\rightarrow$ massive loss
\end{itemize}

\textbf{High voltage transmission:}
\begin{itemize}
    \item Low current needed
    \item $I^2R$ losses small
    \item Example: 100kW at 100kV $\rightarrow$ 1A $\rightarrow$ minimal loss
\end{itemize}

\vspace{0.15cm}

\textbf{Transmission System:}

\begin{enumerate}
    \item \textbf{Power plant:} Generates 11-33 kV
    \item \textbf{Step-up transformer:} 100-700 kV (transmission voltage)
    \item \textbf{Long-distance lines:} High voltage, low current
    \item \textbf{Substation:} Step-down to 33-66 kV (distribution)
    \item \textbf{Local transformer:} Step-down to 120/240V (residential)
\end{enumerate}

\textbf{Why Westinghouse AC Won:}
\begin{itemize}
    \item Edison's DC couldn't be transformed (no transformers for DC)
    \item AC easily stepped up/down with transformers
    \item Enabled long-distance transmission
    \item AC won the "War of Currents"
\end{itemize}
\end{detailbox}

\vspace{0.2cm}

\noindent\textbf{\color{accentcolor} Practical Examples}
\begin{examplebox}
\textbf{Comparison: 100 kW over 1 km line (R=1$\Omega$)}

\textbf{Option 1: 1000V transmission}
\begin{itemize}
    \item Current: $I = \frac{P}{V} = \frac{100000}{1000} = 100$~A
    \item Loss: $P_{loss} = I^2R = 100^2 \times 1 = 10,000$~W = 10 kW
    \item Efficiency: $\frac{90}{100} = 90\%$ (10\% lost!)
\end{itemize}

\textbf{Option 2: 10,000V transmission}
\begin{itemize}
    \item Current: $I = \frac{100000}{10000} = 10$~A
    \item Loss: $P_{loss} = 10^2 \times 1 = 100$~W = 0.1 kW
    \item Efficiency: $\frac{99.9}{100} = 99.9\%$ (0.1\% lost!)
\end{itemize}

\textbf{Result:} 10$\times$ voltage $\rightarrow$ 1/10 current $\rightarrow$ 1/100 loss (100$\times$ improvement!)
\end{examplebox}

\vspace{0.2cm}

\noindent\textbf{\color{accentcolor} Key Points (Interview Focus)}
\begin{keypointsbox}
\begin{enumerate}
    \item \textbf{Loss formula:} $P_{loss} = I^2R$ (proportional to current squared!)
    
    \item \textbf{High voltage advantage:} Same power, lower current, much less loss
    
    \item \textbf{Example:} 10$\times$ voltage $\rightarrow$ 1/10 current $\rightarrow$ 1/100 loss
    
    \item \textbf{Transformers critical:} Enable voltage step-up/step-down
    
    \item \textbf{Transmission voltages:} 100-700 kV typical
    
    \item \textbf{AC advantage:} Easy to transform (DC couldn't do this historically)
\end{enumerate}

\vspace{0.2cm}

\textbf{Interview Questions:}
\begin{itemize}
    \item \textbf{Q:} Why transmit power at high voltage? \\
    \textit{A:} High voltage allows low current for same power. Since loss = I²R, lower current dramatically reduces loss.
    
    \item \textbf{Q:} How much does loss reduce if voltage doubles? \\
    \textit{A:} Current halves, so loss (I²R) reduces to 1/4 (4$\times$ better).
    
    \item \textbf{Q:} Why can't we use high voltage in homes? \\
    \textit{A:} Dangerous! Transformers step down to safe 120/240V for residential use.
\end{itemize}

\vspace{0.2cm}

\textbf{Formulas:}
\begin{itemize}
    \item \textbf{Power:} $P = VI$
    \item \textbf{Current:} $I = \frac{P}{V}$
    \item \textbf{Loss:} $P_{loss} = I^2R_{wire}$
    \item \textbf{Efficiency:} $\eta = \frac{P_{delivered}}{P_{generated}} = \frac{P - P_{loss}}{P}$
\end{itemize}
\end{keypointsbox}

