% ====================================================================
% SECTION 07: Alternating Current (AC)
% ====================================================================

\section{Section 07 -- Alternating Current (AC)}

% --------------------------------------------------------------------
\subsection{Why We Get AC Signal from the Wall Socket}

\noindent\textbf{\color{accentcolor} TL;DR (The Gist)}
\begin{tldrbox}
\begin{itemize}
    \item \textbf{AC for transmission:} Much more efficient than DC for long distances
    \item \textbf{Transformers:} Can easily step up/down AC voltage (not DC)
    \item High voltage transmission $\rightarrow$ Lower current $\rightarrow$ Smaller cables $\rightarrow$ Less power loss
    \item Example: AC system has 10$\times$ less power loss than DC for same power delivery
\end{itemize}
\end{tldrbox}

\vspace{0.2cm}

\noindent\textbf{\color{accentcolor} Detailed Explanation}
\begin{detailbox}
\textbf{Historical Context - The War of Currents:}

\vspace{0.15cm}

\textbf{DC Discovered First:}

\textit{Natural sources of DC:}
\begin{itemize}
    \item \textbf{Lightning:} High-voltage DC discharge
    \item \textbf{Static electricity:} Spark when touching charged material
    \item \textbf{Electric eels:} Produce 50-200V DC at 30A when threatened
    \item \textbf{Solar storms:} Charged particles interacting with Earth's magnetic field
\end{itemize}

\textit{Early investigations:}
\begin{itemize}
    \item DC voltage found in nature first
    \item Scientists studied DC electricity initially
    \item Batteries produce DC voltage
    \item Early electrical systems used DC
\end{itemize}

\vspace{0.15cm}

\textbf{The Late 1880s - Battle Between AC and DC:}

\textbf{Edison's DC System:}

\textit{Limitations of DC:}
\begin{itemize}
    \item DC voltage cannot be easily converted to higher voltages
    \item Even today, DC-DC conversion is complex (requires switching converters)
    \item No simple way to step up/down DC voltage
\end{itemize}

\textit{Edison's solution:}
\begin{itemize}
    \item Small local power plants for each neighborhood
    \item Three-wire distribution: +110V, -110V, Ground (neutral)
    \item Lights/devices connected between $\pm$110V and ground
    \item 110V chosen to allow for voltage drop in transmission
\end{itemize}

\textit{Critical limitation:}
\begin{itemize}
    \item Power plants needed within \textbf{1 mile of end user}
    \item Made rural electrification extremely difficult/impossible
    \item Required many small power plants
    \item Not economically viable for widespread use
\end{itemize}

\vspace{0.15cm}

\textbf{Westinghouse and Tesla's AC System:}

\textit{Key advantage - Transformers:}
\begin{itemize}
    \item \textbf{Transformers work only with AC} (not DC)
    \item Can easily step voltage up to thousands of volts
    \item Can step voltage back down to usable levels
    \item Inexpensive and reliable technology
\end{itemize}

\textit{Why transformers revolutionized power distribution:}
\begin{itemize}
    \item High voltage transmission = low current
    \item Low current = smaller wires (less copper)
    \item Smaller wires = lower cost
    \item Less power loss in transmission
    \item Power plants can be far from users
\end{itemize}

\vspace{0.15cm}

\textbf{Modern Reality:}

\textit{Today's situation:}
\begin{itemize}
    \item Every home/business wired for AC
    \item All appliances plugged into AC outlets
    \item Wall socket provides AC voltage (varies by region)
    \item North America: 120V RMS, 60Hz
    \item Europe/Asia: 230V RMS, 50Hz
\end{itemize}

\textit{But most devices use DC internally!}
\begin{itemize}
    \item TV, computer, phone chargers all convert AC$\rightarrow$DC
    \item Use \textbf{rectifiers} (AC to DC converters)
    \item Almost all electronics operate on DC internally
    \item AC used only for distribution efficiency
\end{itemize}

\vspace{0.15cm}

\textbf{Why AC is More Efficient - Detailed Analysis:}

\textbf{Scenario Setup:}
\begin{itemize}
    \item Two houses, each 1,000 feet from power plant
    \item Each house demands: 100A at 480V
    \item Power required: $P = 480 \times 100 = 48{,}000W = 48kW$
    \item One system DC, one system AC
\end{itemize}

\vspace{0.15cm}

\textbf{DC System Analysis:}

\textbf{Cable requirements:}
\begin{itemize}
    \item Must carry 100A over 1,000 feet
    \item Large diameter cable needed (low resistance)
    \item Resistance inversely proportional to cross-section area
    \item $R = \rho \times \frac{L}{A}$ (resistivity $\times$ length / area)
    \item Typical: 0.15$\Omega$ per 1,000 feet for 100A conductor
\end{itemize}

\textbf{Voltage drop calculation:}
\begin{align*}
    V_{drop} &= I \times R_{cable} \\
    &= 100A \times 0.15\Omega \\
    &= 15V
\end{align*}

\textbf{Generator must supply:}
\begin{equation*}
    V_{gen} = V_{house} + V_{drop} = 480V + 15V = 495V
\end{equation*}

\textbf{Power loss in cable:}
\begin{align*}
    P_{loss} &= V_{drop} \times I \\
    &= 15V \times 100A \\
    &= \boxed{1,500W}
\end{align*}

\textbf{Efficiency:}
\begin{equation*}
    \eta = \frac{48{,}000}{48{,}000 + 1{,}500} \times 100\% = 97\%
\end{equation*}

\vspace{0.15cm}

\textbf{AC System Analysis (with Transformers):}

\textbf{Transformer at power plant:}
\begin{itemize}
    \item Step up voltage: 480V $\rightarrow$ 4,800V (10$\times$ increase)
    \item Step down current: 100A $\rightarrow$ 10A (10$\times$ decrease)
    \item Power conservation: $P_{in} = P_{out}$ (transformer is passive)
    \item $480V \times 100A = 4{,}800V \times 10A = 48{,}000W$ $\checkmark$
\end{itemize}

\textit{Key principle:}
\begin{itemize}
    \item Transformer increases voltage at expense of current
    \item NOT a power generator (passive component)
    \item Output power = Input power (minus small losses)
    \item $V_{out} \times I_{out} = V_{in} \times I_{in}$
\end{itemize}

\textbf{Cable requirements:}
\begin{itemize}
    \item Only 10A current (10$\times$ less than DC)
    \item Smaller diameter cable sufficient
    \item Higher resistance acceptable (less current)
    \item Typical: 1.5$\Omega$ per 1,000 feet for 10A conductor
\end{itemize}

\textbf{Voltage drop:}
\begin{align*}
    V_{drop} &= I \times R_{cable} \\
    &= 10A \times 1.5\Omega \\
    &= 15V \text{ (same as DC!)}
\end{align*}

\textbf{Power loss in cable:}
\begin{align*}
    P_{loss} &= V_{drop} \times I \\
    &= 15V \times 10A \\
    &= \boxed{150W}
\end{align*}

\textbf{Transformer at house:}
\begin{itemize}
    \item Step down voltage: 4,800V $\rightarrow$ 480V
    \item Step up current: 10A $\rightarrow$ 100A
    \item Delivers required power to house
\end{itemize}

\vspace{0.15cm}

\textbf{Comparison - AC vs DC:}

\begin{center}
\begin{tabular}{|l|c|c|}
\hline
\textbf{Parameter} & \textbf{DC System} & \textbf{AC System} \\
\hline
Transmission voltage & 480V & 4,800V \\
Transmission current & 100A & 10A \\
Cable resistance & 0.15$\Omega$ & 1.5$\Omega$ \\
Voltage drop & 15V & 15V \\
Power loss & 1,500W & 150W \\
Cable size & Large (100A) & Small (10A) \\
Efficiency & 97\% & 99.7\% \\
\hline
\end{tabular}
\end{center}

\textbf{Result:}
\begin{itemize}
    \item AC system: \textbf{10$\times$ less power loss} than DC
    \item AC uses smaller, cheaper cables
    \item AC allows long-distance transmission
    \item DC requires power plant within 1 mile
    \item AC wins for power distribution!
\end{itemize}

\vspace{0.15cm}

\textbf{Why Voltage Matters for Transmission:}

\textit{Power loss formula:}
\begin{equation*}
    P_{loss} = I^2 R_{cable}
\end{equation*}

\textit{Key insight:}
\begin{itemize}
    \item Power loss proportional to $I^2$ (square of current)
    \item Halving current reduces loss by 4$\times$
    \item 10$\times$ less current = 100$\times$ less loss (but cable R also changes)
    \item High voltage = low current = low loss
\end{itemize}

\textit{Why high voltage helps:}
\begin{itemize}
    \item For same power: $P = V \times I$
    \item Higher V $\rightarrow$ lower I (inversely related)
    \item Lower I $\rightarrow$ much less $I^2R$ loss
    \item Can use smaller cable (cheaper)
\end{itemize}

\vspace{0.15cm}

\textbf{Real-World AC Transmission:}

\textit{Typical voltage levels:}
\begin{itemize}
    \item Power plant generation: 10-25kV
    \item Long-distance transmission: 110-765kV (very high!)
    \item Regional distribution: 10-35kV
    \item Local distribution: 4-15kV
    \item Residential service: 120/240V (North America), 230V (Europe)
\end{itemize}

\textit{Multiple transformer stages:}
\begin{enumerate}
    \item Step up at power plant (to hundreds of kV)
    \item Transmission over long distances
    \item Step down at substations (to tens of kV)
    \item Step down at local transformers (to household voltage)
\end{enumerate}
\end{detailbox}

\vspace{0.2cm}

\noindent\textbf{\color{accentcolor} Practical Example \& Numerical}
\begin{examplebox}
\textbf{Example 1: DC vs AC Power Loss Calculation}

\textit{Given:} Transmit 50kW over 2,000 feet

\textbf{DC System (480V):}
\begin{align*}
    I &= \frac{P}{V} = \frac{50{,}000}{480} = 104.2A \\
    R_{cable} &= 0.15\Omega/1000ft \times 2 = 0.3\Omega \\
    P_{loss} &= I^2 R = (104.2)^2 \times 0.3 = 3{,}258W
\end{align*}

\textbf{AC System (4,800V with transformer):}
\begin{align*}
    I &= \frac{50{,}000}{4{,}800} = 10.42A \\
    R_{cable} &= 1.5\Omega/1000ft \times 2 = 3\Omega \\
    P_{loss} &= I^2 R = (10.42)^2 \times 3 = 326W
\end{align*}

\textbf{Comparison:}
\begin{equation*}
    \frac{P_{loss(DC)}}{P_{loss(AC)}} = \frac{3{,}258}{326} = \boxed{10\times}
\end{equation*}

AC has 10$\times$ less power loss!

\vspace{0.2cm}

\textbf{Example 2: Why Double Voltage Helps So Much}

\textit{Same power (10kW) transmitted at different voltages:}

\textbf{At 100V:}
\begin{align*}
    I &= \frac{10{,}000}{100} = 100A \\
    P_{loss} &= I^2 R = (100)^2 \times 1 = 10{,}000W
\end{align*}

\textbf{At 200V (doubled):}
\begin{align*}
    I &= \frac{10{,}000}{200} = 50A \\
    P_{loss} &= I^2 R = (50)^2 \times 1 = 2{,}500W
\end{align*}

\textbf{At 1,000V (10$\times$):}
\begin{align*}
    I &= \frac{10{,}000}{1{,}000} = 10A \\
    P_{loss} &= I^2 R = (10)^2 \times 1 = 100W
\end{align*}

Doubling voltage reduces loss by 4$\times$; 10$\times$ voltage reduces loss by 100$\times$!

\vspace{0.2cm}

\textbf{Example 3: Transformer Power Conservation}

\textit{Step-up transformer:}
\begin{itemize}
    \item Input: 120V, 10A
    \item Turns ratio: 1:10 (step up 10$\times$)
\end{itemize}

\textbf{Input power:}
\begin{equation*}
    P_{in} = 120V \times 10A = 1{,}200W
\end{equation*}

\textbf{Output (assuming ideal transformer):}
\begin{align*}
    V_{out} &= 120V \times 10 = 1{,}200V \\
    I_{out} &= 10A / 10 = 1A \\
    P_{out} &= 1{,}200V \times 1A = 1{,}200W
\end{align*}

Power conserved: $P_{in} = P_{out}$ $\checkmark$

\vspace{0.2cm}

\textbf{Example 4: Regional AC Voltage Differences}

\textbf{North America:}
\begin{itemize}
    \item Voltage: 120V RMS (single phase)
    \item Also 240V available (split phase)
    \item Frequency: 60Hz
    \item Outlets: Type A/B (two or three prong)
\end{itemize}

\textbf{Europe/Asia/Africa/Australia:}
\begin{itemize}
    \item Voltage: 230V RMS (single phase)
    \item Frequency: 50Hz
    \item Outlets: Various types (C, D, E, F, G, I, etc.)
\end{itemize}

\textbf{Why different voltages?}
\begin{itemize}
    \item Historical decisions (Edison vs European systems)
    \item 230V more efficient for high-power appliances
    \item 120V considered safer (lower shock risk)
    \item Both work well for modern electronics
\end{itemize}

\vspace{0.2cm}

\textbf{Example 5: Why Most Devices Need DC Internally}

\textit{Devices that convert AC to DC:}
\begin{itemize}
    \item \textbf{Laptop:} AC adapter converts 120V AC $\rightarrow$ 19V DC
    \item \textbf{Phone charger:} 120V AC $\rightarrow$ 5V DC (USB)
    \item \textbf{TV:} Internal power supply converts AC $\rightarrow$ multiple DC voltages
    \item \textbf{LED bulbs:} Rectifier + regulator converts AC $\rightarrow$ DC for LEDs
    \item \textbf{Computer:} PSU converts 120V AC $\rightarrow$ 12V, 5V, 3.3V DC rails
\end{itemize}

\textit{Why DC needed:}
\begin{itemize}
    \item Semiconductors (transistors, ICs) require DC
    \item Logic circuits need stable voltage
    \item Microprocessors operate on DC only
    \item LEDs are DC devices
    \item Motors in fans/drives use DC (or rectified AC)
\end{itemize}

\vspace{0.2cm}

\textbf{Example 6: Cable Resistance and Diameter}

\textit{Resistance formula:}
\begin{equation*}
    R = \rho \frac{L}{A}
\end{equation*}

Where:
\begin{itemize}
    \item $\rho$ = resistivity (material property)
    \item $L$ = length
    \item $A$ = cross-sectional area
\end{itemize}

\textbf{For circular wire:} $A = \pi r^2 = \pi (d/2)^2$

\textit{Key insight:}
\begin{itemize}
    \item Doubling diameter $\rightarrow$ 4$\times$ area $\rightarrow$ 1/4 resistance
    \item Larger diameter = lower resistance
    \item Lower resistance = less voltage drop
    \item But larger cable costs more (more copper)
\end{itemize}

\textbf{Trade-off:}
\begin{itemize}
    \item High current $\rightarrow$ need large cable (expensive)
    \item Low current $\rightarrow$ small cable works (cheaper)
    \item AC with transformers $\rightarrow$ low current $\rightarrow$ cheap cables!
\end{itemize}
\end{examplebox}

\vspace{0.2cm}

\noindent\textbf{\color{accentcolor} Key Points (Interview Focus)}
\begin{keypointsbox}
\begin{enumerate}
    \item \textbf{AC for transmission:} 10$\times$ more efficient than DC for long distances
    \item \textbf{Transformers:} Easily step AC voltage up/down (impossible with DC)
    \item \textbf{High voltage advantage:} Less current $\rightarrow$ less $I^2R$ loss
    \item \textbf{Edison's DC:} Limited to 1 mile radius (impractical)
    \item \textbf{Tesla/Westinghouse AC:} Enabled long-distance power distribution
    \item \textbf{Modern reality:} AC for distribution, DC for device operation
    \item \textbf{Power conservation:} Transformer changes V and I but not P
    \item \textbf{Cable size:} High voltage = low current = smaller/cheaper cables
\end{enumerate}

\textbf{Interview Questions:}
\begin{itemize}
    \item \textbf{Q:} Why use AC for power distribution instead of DC? \\
    \textit{A:} Transformers can easily change AC voltage; high voltage transmission has much lower losses.
    
    \item \textbf{Q:} What was the main limitation of Edison's DC system? \\
    \textit{A:} Power plants had to be within 1 mile of users; impractical for rural areas.
    
    \item \textbf{Q:} Why does high voltage reduce transmission losses? \\
    \textit{A:} Higher voltage = lower current for same power; loss is $I^2R$, so lower I drastically reduces loss.
    
    \item \textbf{Q:} Do transformers create power? \\
    \textit{A:} No - they're passive; output power = input power (minus small losses).
    
    \item \textbf{Q:} If transformer steps up voltage 10$\times$, what happens to current? \\
    \textit{A:} Current steps down 10$\times$ (power conservation: $V_{in}I_{in} = V_{out}I_{out}$).
    
    \item \textbf{Q:} Do most electronics use AC or DC internally? \\
    \textit{A:} DC - semiconductors, ICs, LEDs all require DC; AC only used for distribution.
\end{itemize}

\textbf{AC vs DC Comparison:}
\begin{itemize}
    \item \textbf{Transmission:} AC wins (transformers, efficiency)
    \item \textbf{Storage:} DC wins (batteries are DC)
    \item \textbf{Electronics:} DC wins (semiconductors need DC)
    \item \textbf{Long distance:} AC wins (10-100$\times$ less loss)
    \item \textbf{Simplicity:} DC wins (no frequency, no transformers needed)
    \item \textbf{Cost:} AC wins (smaller cables, fewer power plants)
\end{itemize}

\textbf{Applications:}
\begin{itemize}
    \item AC: Power grids, household distribution, large motors
    \item DC: Batteries, electronics, solar panels, electric vehicles
    \item Hybrid: Modern HVDC transmission (AC$\rightarrow$DC$\rightarrow$AC for very long distances)
\end{itemize}

\textbf{Common Misconceptions:}
\begin{itemize}
    \item "AC is always better than DC" - No, each has advantages
    \item "Transformers amplify power" - No, passive devices (power in = power out)
    \item "Higher voltage always dangerous" - Danger depends on current through body
    \item "All household devices run on AC" - No, most convert to DC internally
\end{itemize}
\end{keypointsbox}

% --------------------------------------------------------------------
\subsection{AC - Important Characteristics}

\noindent\textbf{\color{accentcolor} TL;DR (The Gist)}
\begin{tldrbox}
\begin{itemize}
    \item \textbf{AC signal:} Sinusoidal waveform, alternates positive/negative
    \item \textbf{Frequency (f):} Cycles per second (Hz); $f = 1/T$
    \item \textbf{Period (T):} Time for one complete cycle (seconds); $T = 1/f$
    \item \textbf{Amplitude/Peak:} Maximum voltage; Peak-to-peak = $2 \times$ amplitude
\end{itemize}
\end{tldrbox}

\vspace{0.2cm}

\noindent\textbf{\color{accentcolor} Detailed Explanation}
\begin{detailbox}
\textbf{AC Signal Representation:}

\vspace{0.15cm}

\textbf{Graphical Form - Sinusoidal Waveform:}

\textit{What is AC?}
\begin{itemize}
    \item \textbf{Alternating Current:} Current flows one way, then reverses
    \item Voltage continuously changes: positive $\rightarrow$ zero $\rightarrow$ negative $\rightarrow$ zero
    \item Represented by sine wave (sinusoid)
    \item Most common and natural AC waveform
\end{itemize}

\textit{How AC flows:}
\begin{itemize}
    \item When voltage positive: Current flows in one direction
    \item When voltage negative: Current flows in opposite direction
    \item Voltage and current go "hand-in-hand"
    \item Direction reverses every half cycle
\end{itemize}

\vspace{0.15cm}

\textbf{Zero Crossing:}

\textit{Definition:}
\begin{itemize}
    \item Point where waveform crosses zero volts
    \item Transitions between positive and negative
    \item Two zero crossings per cycle (up and down)
    \item Critical for timing and synchronization
\end{itemize}

\textit{Significance:}
\begin{itemize}
    \item Marks direction reversal
    \item Used for frequency measurement
    \item Important for phase control circuits
    \item Zero-crossing detection common in AC applications
\end{itemize}

\vspace{0.15cm}

\textbf{Frequency (f) - Cycles Per Second:}

\textbf{Definition:}
\begin{itemize}
    \item Number of complete cycles in one second
    \item Unit: \textbf{Hertz (Hz)}
    \item 1 Hz = 1 cycle per second
    \item Determines how fast signal oscillates
\end{itemize}

\textbf{Mathematical expression:}
\begin{equation*}
    f = \frac{1}{T}
\end{equation*}

Where $T$ = period (time for one cycle)

\vspace{0.15cm}

\textbf{Common AC Frequencies:}

\textbf{Power line frequencies:}
\begin{itemize}
    \item \textbf{60 Hz:} North America, parts of South America, Japan
    \item \textbf{50 Hz:} Europe, Asia, Africa, Australia
    \item These are standard worldwide power frequencies
    \item Choice historical (no technical advantage either way)
\end{itemize}

\textbf{Other frequency ranges:}
\begin{itemize}
    \item Audio: 20 Hz - 20 kHz (human hearing range)
    \item Radio AM: 530 kHz - 1.7 MHz
    \item Radio FM: 88 MHz - 108 MHz
    \item WiFi: 2.4 GHz, 5 GHz
    \item Microwave: 2.45 GHz (microwave ovens)
\end{itemize}

\vspace{0.15cm}

\textbf{Frequency Effect on Waveform:}

\textbf{Low frequency (e.g., 1 Hz):}
\begin{itemize}
    \item Slow oscillation (1 cycle per second)
    \item Long time for complete cycle
    \item Waveform appears "stretched out"
    \item Easy to see individual cycles
\end{itemize}

\textbf{Medium frequency (e.g., 60 Hz):}
\begin{itemize}
    \item 60 complete cycles per second
    \item Each cycle takes 16.67 ms
    \item Typical power line frequency
    \item Causes flicker in some lights (perceptible)
\end{itemize}

\textbf{High frequency (e.g., 1 kHz):}
\begin{itemize}
    \item 1,000 cycles per second
    \item Each cycle only 1 ms
    \item Waveform appears "compressed"
    \item Many oscillations in short time
\end{itemize}

\textit{General rule:}
\begin{itemize}
    \item Higher frequency $\rightarrow$ faster oscillation $\rightarrow$ shorter period
    \item Lower frequency $\rightarrow$ slower oscillation $\rightarrow$ longer period
    \item Frequency and period inversely related
\end{itemize}

\vspace{0.15cm}

\textbf{Period (T) - Time Per Cycle:}

\textbf{Definition:}
\begin{itemize}
    \item Time for signal to complete \textbf{one full cycle}
    \item Measured in seconds (or ms, $\mu$s)
    \item From zero $\rightarrow$ positive peak $\rightarrow$ zero $\rightarrow$ negative peak $\rightarrow$ zero
    \item Complete repetition of waveform pattern
\end{itemize}

\textbf{Mathematical expression:}
\begin{equation*}
    T = \frac{1}{f}
\end{equation*}

\textit{Relationship to frequency:}
\begin{itemize}
    \item Period and frequency are reciprocals
    \item If period increases $\rightarrow$ frequency decreases
    \item If frequency increases $\rightarrow$ period decreases
    \item \textbf{Inversely proportional}
\end{itemize}

\vspace{0.15cm}

\textbf{What Constitutes One Cycle:}

\textit{Complete cycle includes:}
\begin{enumerate}
    \item Start at zero volts
    \item Rise to positive peak
    \item Return to zero (first zero crossing)
    \item Fall to negative peak
    \item Return to zero (second zero crossing)
    \item Back to starting point
\end{enumerate}

\textit{Alternative description:}
\begin{itemize}
    \item One complete wavelength
    \item 360° or $2\pi$ radians
    \item Positive half-cycle + negative half-cycle
    \item Pattern that repeats continuously
\end{itemize}

\vspace{0.15cm}

\textbf{Amplitude - Maximum Voltage:}

\textbf{Definition:}
\begin{itemize}
    \item Maximum voltage reached by signal
    \item Distance from zero to peak (positive or negative)
    \item Also called \textbf{peak voltage} ($V_{peak}$)
    \item Measured in volts
\end{itemize}

\textit{Positive vs negative amplitude:}
\begin{itemize}
    \item Positive peak: Maximum positive voltage
    \item Negative peak: Maximum negative voltage
    \item For symmetrical AC: Both have same magnitude
    \item Sign indicates direction only (not "less than zero")
\end{itemize}

\vspace{0.15cm}

\textbf{Peak-to-Peak Voltage ($V_{p-p}$):}

\textbf{Definition:}
\begin{itemize}
    \item Total voltage swing from negative peak to positive peak
    \item Twice the amplitude (for symmetrical waveform)
    \item $V_{p-p} = 2 \times V_{peak}$
    \item Easy to measure on oscilloscope
\end{itemize}

\textit{Why peak-to-peak useful:}
\begin{itemize}
    \item Oscilloscope displays peak-to-peak naturally
    \item Shows total voltage excursion
    \item Useful for component voltage rating checks
    \item Common measurement in practice
\end{itemize}

\textbf{Example:}
\begin{itemize}
    \item If $V_{peak} = 170V$
    \item Then $V_{p-p} = 2 \times 170 = 340V$
    \item Signal swings from -170V to +170V
\end{itemize}

\vspace{0.15cm}

\textbf{Important Distinctions:}

\begin{center}
\begin{tabular}{|l|l|}
\hline
\textbf{Term} & \textbf{Meaning} \\
\hline
Amplitude & Maximum value from zero (same as peak) \\
Peak voltage & Maximum voltage (positive or negative) \\
Peak-to-peak & Total swing (2$\times$ amplitude) \\
RMS voltage & Effective value (covered next topic) \\
\hline
\end{tabular}
\end{center}

\vspace{0.15cm}

\textbf{AC Waveform Properties Summary:}

\textbf{Continuous variation:}
\begin{itemize}
    \item Voltage never constant (always changing)
    \item Smoothly transitions through zero
    \item Sinusoidal shape most common
    \item Periodic (repeats regularly)
\end{itemize}

\textbf{Symmetrical (typical):}
\begin{itemize}
    \item Positive half matches negative half
    \item Average voltage = 0 (positive cancels negative)
    \item Peak positive = Peak negative (magnitude)
    \item Center line at 0V
\end{itemize}

\textbf{Key parameters:}
\begin{itemize}
    \item \textbf{Frequency:} How fast it oscillates
    \item \textbf{Period:} How long one cycle takes
    \item \textbf{Amplitude/Peak:} How high it goes
    \item \textbf{Peak-to-peak:} Total voltage range
\end{itemize}

\vspace{0.15cm}

\textbf{Frequency Prefixes:}

\textit{Common in electronics:}
\begin{itemize}
    \item Hz (Hertz): Base unit
    \item kHz (kilohertz): $10^3$ Hz = 1,000 Hz
    \item MHz (megahertz): $10^6$ Hz = 1,000,000 Hz
    \item GHz (gigahertz): $10^9$ Hz = 1,000,000,000 Hz
\end{itemize}
\end{detailbox}

\vspace{0.2cm}

\noindent\textbf{\color{accentcolor} Practical Example \& Numerical}
\begin{examplebox}
\textbf{Example 1: 60 Hz Power Line}

\textit{North American household power:}

\textbf{Frequency:}
\begin{equation*}
    f = 60\text{ Hz}
\end{equation*}

\textbf{Period:}
\begin{equation*}
    T = \frac{1}{f} = \frac{1}{60} = 0.01667\text{ s} = \boxed{16.67\text{ ms}}
\end{equation*}

\textbf{Interpretation:}
\begin{itemize}
    \item One complete cycle takes 16.67 milliseconds
    \item 60 cycles occur in one second
    \item Voltage crosses zero 120 times per second (2 per cycle)
\end{itemize}

\vspace{0.2cm}

\textbf{Example 2: 50 Hz European Power}

\textbf{Frequency:}
\begin{equation*}
    f = 50\text{ Hz}
\end{equation*}

\textbf{Period:}
\begin{equation*}
    T = \frac{1}{50} = 0.02\text{ s} = \boxed{20\text{ ms}}
\end{equation*}

\textbf{Comparison to 60 Hz:}
\begin{itemize}
    \item 50 Hz period: 20 ms
    \item 60 Hz period: 16.67 ms
    \item 50 Hz is 20\% slower oscillation
    \item Both work fine for power distribution
\end{itemize}

\vspace{0.2cm}

\textbf{Example 3: Audio Frequency - Middle A Note}

\textit{Musical note A above middle C:}

\textbf{Frequency:}
\begin{equation*}
    f = 440\text{ Hz}
\end{equation*}

\textbf{Period:}
\begin{equation*}
    T = \frac{1}{440} = 0.002273\text{ s} = \boxed{2.273\text{ ms}}
\end{equation*}

\textbf{Meaning:}
\begin{itemize}
    \item Sound wave vibrates 440 times per second
    \item Each vibration takes 2.273 ms
    \item Human ear perceives this as the note "A"
\end{itemize}

\vspace{0.2cm}

\textbf{Example 4: FM Radio Station}

\textit{FM radio at 100.7 MHz:}

\textbf{Frequency:}
\begin{equation*}
    f = 100.7\text{ MHz} = 100{,}700{,}000\text{ Hz}
\end{equation*}

\textbf{Period:}
\begin{align*}
    T &= \frac{1}{100{,}700{,}000} \\
    &= 9.93 \times 10^{-9}\text{ s} \\
    &= \boxed{9.93\text{ ns (nanoseconds)}}
\end{align*}

\textbf{Interpretation:}
\begin{itemize}
    \item Extremely fast oscillation
    \item Over 100 million cycles per second
    \item Each cycle only ~10 nanoseconds
    \item Radio waves travel at speed of light
\end{itemize}

\vspace{0.2cm}

\textbf{Example 5: Peak and Peak-to-Peak}

\textit{Household AC voltage (North America):}

\textbf{Peak voltage:}
\begin{equation*}
    V_{peak} = 170\text{ V}
\end{equation*}

\textbf{Peak-to-peak voltage:}
\begin{equation*}
    V_{p-p} = 2 \times V_{peak} = 2 \times 170 = \boxed{340\text{ V}}
\end{equation*}

\textbf{Meaning:}
\begin{itemize}
    \item Voltage swings from -170V to +170V
    \item Total excursion: 340V
    \item This is for 120V RMS (covered next topic)
\end{itemize}

\vspace{0.2cm}

\textbf{Example 6: Comparing Three Frequencies}

\textit{Same amplitude, different frequencies:}

\textbf{Signal A: 1 Hz}
\begin{itemize}
    \item Period: $T = 1/1 = 1$ second
    \item Very slow oscillation
    \item One cycle per second
\end{itemize}

\textbf{Signal B: 10 Hz}
\begin{itemize}
    \item Period: $T = 1/10 = 0.1$ second = 100 ms
    \item 10 times faster than Signal A
    \item 10 cycles per second
\end{itemize}

\textbf{Signal C: 100 Hz}
\begin{itemize}
    \item Period: $T = 1/100 = 0.01$ second = 10 ms
    \item 100 times faster than Signal A
    \item 100 cycles per second
\end{itemize}

\textbf{Observation:}
\begin{itemize}
    \item Higher frequency $\rightarrow$ more compressed waveform
    \item Lower frequency $\rightarrow$ more stretched waveform
    \item All have same amplitude (height)
    \item Differ only in oscillation rate
\end{itemize}

\vspace{0.2cm}

\textbf{Example 7: Frequency Unit Conversions}

\textbf{Convert 5 kHz to Hz:}
\begin{equation*}
    5\text{ kHz} = 5 \times 10^3 = 5{,}000\text{ Hz}
\end{equation*}

\textbf{Convert 2.4 GHz to MHz:}
\begin{equation*}
    2.4\text{ GHz} = 2.4 \times 10^3 = 2{,}400\text{ MHz}
\end{equation*}

\textbf{Convert 500 kHz to MHz:}
\begin{equation*}
    500\text{ kHz} = 500 / 1000 = 0.5\text{ MHz}
\end{equation*}

\textbf{Find period of 20 kHz:}
\begin{equation*}
    T = \frac{1}{20{,}000} = 0.00005\text{ s} = 50\text{ $\mu$s}
\end{equation*}

\vspace{0.2cm}

\textbf{Example 8: Oscilloscope Measurement}

\textit{Oscilloscope display shows:}
\begin{itemize}
    \item Peak-to-peak voltage: 10V
    \item 4 complete cycles in 20 ms
\end{itemize}

\textbf{Find amplitude:}
\begin{equation*}
    V_{peak} = \frac{V_{p-p}}{2} = \frac{10}{2} = \boxed{5\text{ V}}
\end{equation*}

\textbf{Find period:}
\begin{equation*}
    T = \frac{20\text{ ms}}{4\text{ cycles}} = \boxed{5\text{ ms per cycle}}
\end{equation*}

\textbf{Find frequency:}
\begin{equation*}
    f = \frac{1}{T} = \frac{1}{0.005} = \boxed{200\text{ Hz}}
\end{equation*}
\end{examplebox}

\vspace{0.2cm}

\noindent\textbf{\color{accentcolor} Key Points (Interview Focus)}
\begin{keypointsbox}
\begin{enumerate}
    \item \textbf{AC waveform:} Sinusoidal, alternates positive/negative continuously
    \item \textbf{Frequency (f):} Cycles per second (Hz); $f = 1/T$
    \item \textbf{Period (T):} Time per cycle (seconds); $T = 1/f$
    \item \textbf{Amplitude/Peak:} Maximum voltage from zero
    \item \textbf{Peak-to-peak:} $V_{p-p} = 2 \times V_{peak}$ (total swing)
    \item \textbf{Zero crossing:} Points where voltage = 0 (direction reversal)
    \item \textbf{One cycle:} Zero $\rightarrow$ peak+ $\rightarrow$ zero $\rightarrow$ peak- $\rightarrow$ zero
    \item \textbf{Inverse relationship:} Higher f $\rightarrow$ shorter T; Lower f $\rightarrow$ longer T
\end{enumerate}

\textbf{Interview Questions:}
\begin{itemize}
    \item \textbf{Q:} What is frequency? \\
    \textit{A:} Number of complete cycles per second, measured in Hertz (Hz).
    
    \item \textbf{Q:} What is the period of 60 Hz AC? \\
    \textit{A:} $T = 1/60 = 16.67$ ms.
    
    \item \textbf{Q:} If peak voltage is 100V, what is peak-to-peak? \\
    \textit{A:} $V_{p-p} = 2 \times 100 = 200V$.
    
    \item \textbf{Q:} What is one complete AC cycle? \\
    \textit{A:} Zero $\rightarrow$ positive peak $\rightarrow$ zero $\rightarrow$ negative peak $\rightarrow$ zero (complete waveform repetition).
    
    \item \textbf{Q:} Relationship between frequency and period? \\
    \textit{A:} Inversely proportional: $f = 1/T$ and $T = 1/f$.
    
    \item \textbf{Q:} How many zero crossings per cycle? \\
    \textit{A:} Two - one going positive, one going negative.
    
    \item \textbf{Q:} What's the difference between amplitude and peak-to-peak? \\
    \textit{A:} Amplitude is max from zero; peak-to-peak is total swing (2$\times$ amplitude).
\end{itemize}

\textbf{Common Frequencies:}
\begin{itemize}
    \item 50 Hz: European power line
    \item 60 Hz: North American power line
    \item 20 Hz - 20 kHz: Audio (human hearing)
    \item 530 kHz - 1.7 MHz: AM radio
    \item 88 MHz - 108 MHz: FM radio
    \item 2.4 GHz, 5 GHz: WiFi
\end{itemize}

\textbf{Formulas:}
\begin{itemize}
    \item $f = 1/T$ (frequency from period)
    \item $T = 1/f$ (period from frequency)
    \item $V_{p-p} = 2 \times V_{peak}$ (peak-to-peak voltage)
    \item $V_{peak} = V_{p-p} / 2$ (peak voltage)
\end{itemize}

\textbf{Applications:}
\begin{itemize}
    \item Power distribution (50/60 Hz)
    \item Audio signals (20 Hz - 20 kHz)
    \item Radio communications (kHz to GHz)
    \item Oscilloscope measurements
    \item Signal processing
    \item Frequency synthesis
\end{itemize}

\textbf{Common Mistakes:}
\begin{itemize}
    \item Confusing amplitude with peak-to-peak
    \item Forgetting frequency and period are reciprocals
    \item Thinking negative voltage is "less than zero" (it's just direction)
    \item Confusing cycle count with zero-crossing count
\end{itemize}
\end{keypointsbox}

% --------------------------------------------------------------------
\subsection{Root Mean Square Voltage (V\textsubscript{rms})}

\noindent\textbf{\color{accentcolor} TL;DR (The Gist)}
\begin{tldrbox}
\begin{itemize}
    \item \textbf{RMS voltage:} Effective AC voltage = equivalent DC for same power
    \item \textbf{Formula:} $V_{rms} = 0.707 \times V_{peak}$ or $V_{peak} = 1.414 \times V_{rms}$
    \item \textbf{Meters show RMS:} Multimeter displays RMS, not peak
    \item 120V AC means 120V RMS (peak is actually 170V)
\end{itemize}
\end{tldrbox}

\vspace{0.2cm}

\noindent\textbf{\color{accentcolor} Detailed Explanation}
\begin{detailbox}
\textbf{The Problem with AC Voltage Values:}

\vspace{0.15cm}

\textbf{AC is constantly changing:}
\begin{itemize}
    \item Voltage goes from zero $\rightarrow$ peak+ $\rightarrow$ zero $\rightarrow$ peak- $\rightarrow$ zero
    \item Always varying (never constant)
    \item Most of the time, voltage is \textbf{less than peak}
    \item Peak voltage not good measure of "real effect"
\end{itemize}

\textbf{Average doesn't work either:}
\begin{itemize}
    \item Positive half-cycle and negative half-cycle
    \item Positive values exactly cancel negative values
    \item Mathematical average of sine wave = \textbf{0 volts}
    \item Useless for comparing to DC!
\end{itemize}

\textit{Need better measure:}
\begin{itemize}
    \item Represents "effective" power delivery
    \item Allows comparison with DC
    \item Practical and meaningful
    \item Solution: \textbf{Root Mean Square (RMS)}
\end{itemize}

\vspace{0.15cm}

\textbf{What is RMS Voltage?}

\textbf{Definition:}
\begin{itemize}
    \item \textbf{Effective value} of varying voltage/current
    \item Equivalent steady DC value giving \textbf{same power}
    \item Represents heating effect or power delivery capability
    \item The voltage that "matters" for real work
\end{itemize}

\textbf{Physical meaning:}
\begin{itemize}
    \item AC at 120V RMS delivers same power as 120V DC
    \item Lights up bulb to same brightness
    \item Produces same heat in resistor
    \item Does same amount of work
\end{itemize}

\vspace{0.15cm}

\textbf{Mathematical Relationship (For Sine Waves):}

\textbf{RMS from Peak:}
\begin{equation*}
    \boxed{V_{rms} = 0.707 \times V_{peak}}
\end{equation*}

Or more precisely:
\begin{equation*}
    V_{rms} = \frac{V_{peak}}{\sqrt{2}} = \frac{V_{peak}}{1.414}
\end{equation*}

\textbf{Peak from RMS:}
\begin{equation*}
    \boxed{V_{peak} = 1.414 \times V_{rms}}
\end{equation*}

Or:
\begin{equation*}
    V_{peak} = \sqrt{2} \times V_{rms}
\end{equation*}

\textit{Important note:}
\begin{itemize}
    \item These factors (0.707 and 1.414) are \textbf{only for sine waves}
    \item Different waveforms (square, triangle) have different factors
    \item Sine wave most common in power systems
    \item Always assume sine wave unless stated otherwise
\end{itemize}

\vspace{0.15cm}

\textbf{Same Formulas for Current:}

\begin{align*}
    I_{rms} &= 0.707 \times I_{peak} \\
    I_{peak} &= 1.414 \times I_{rms}
\end{align*}

\vspace{0.15cm}

\textbf{RMS - The "Equivalent DC" Concept:}

\textbf{Example comparison:}

\textit{Circuit 1: DC powered}
\begin{itemize}
    \item 120V DC battery
    \item Light bulb
    \item Steady, constant brightness
\end{itemize}

\textit{Circuit 2: AC powered}
\begin{itemize}
    \item 170V peak AC source (120V RMS)
    \item Same light bulb
    \item Same brightness as DC circuit!
\end{itemize}

\textbf{Why same brightness?}
\begin{itemize}
    \item RMS voltage represents effective power
    \item 170V peak AC delivers same average power as 120V DC
    \item Brightness depends on power, not instantaneous voltage
    \item Bulb responds to average heating effect
\end{itemize}

\vspace{0.15cm}

\textbf{What Your Meter Shows:}

\textbf{Multimeter on AC mode:}
\begin{itemize}
    \item \textbf{Always displays RMS value}
    \item NOT peak voltage
    \item NOT average voltage
    \item NOT peak-to-peak
    \item Shows the "effective" AC voltage
\end{itemize}

\textbf{Example - North American outlet:}
\begin{itemize}
    \item Actual peak voltage: 170V
    \item Meter reading: 120V (RMS)
    \item Peak-to-peak: 340V
    \item But you see: 120V on meter
\end{itemize}

\textit{Why RMS on meters?}
\begin{itemize}
    \item Allows direct comparison with DC
    \item Represents actual power capability
    \item Industry standard
    \item More meaningful for practical use
\end{itemize}

\vspace{0.15cm}

\textbf{When AC Voltage is Specified:}

\textbf{Convention:}
\begin{itemize}
    \item "120V AC" means 120V \textbf{RMS} (unless stated otherwise)
    \item "230V AC" means 230V RMS
    \item If peak voltage meant, clearly stated as "peak"
    \item Example: "170V peak" or "120V RMS"
\end{itemize}

\textbf{Always assume RMS for AC specifications:}
\begin{itemize}
    \item Wall outlet ratings
    \item Power supply specifications
    \item Component voltage ratings
    \item Electrical code requirements
    \item Everyday usage
\end{itemize}

\vspace{0.15cm}

\textbf{Why "Root Mean Square"?}

\textit{The name comes from the calculation method:}

\begin{enumerate}
    \item \textbf{Square} all instantaneous values (makes negative positive)
    \item Find the \textbf{Mean} (average) of the squared values
    \item Take the square \textbf{Root} of the mean
\end{enumerate}

\textit{Mathematical formula (general):}
\begin{equation*}
    V_{rms} = \sqrt{\frac{1}{T} \int_0^T v^2(t) \, dt}
\end{equation*}

For sine wave, this simplifies to:
\begin{equation*}
    V_{rms} = \frac{V_{peak}}{\sqrt{2}} = 0.707 \times V_{peak}
\end{equation*}

\textit{Why squaring helps:}
\begin{itemize}
    \item Power proportional to $V^2$ (in resistor: $P = V^2/R$)
    \item Squaring accounts for power delivery
    \item Makes all values positive
    \item Taking root returns to voltage units
\end{itemize}

\vspace{0.15cm}

\textbf{RMS Represents Power Accurately:}

\textbf{Power in resistor with AC:}
\begin{equation*}
    P_{avg} = \frac{V_{rms}^2}{R} = I_{rms}^2 \times R = V_{rms} \times I_{rms}
\end{equation*}

\textit{Same formulas as DC!}
\begin{itemize}
    \item Using RMS values, power formulas identical to DC
    \item No need for special AC power formulas
    \item Direct substitution works
    \item This is why RMS is so useful!
\end{itemize}

\vspace{0.15cm}

\textbf{RMS vs Peak vs Peak-to-Peak Summary:}

\begin{center}
\begin{tabular}{|l|c|c|}
\hline
\textbf{Measurement} & \textbf{Formula} & \textbf{Example (120V RMS)} \\
\hline
RMS & $0.707 \times V_{peak}$ & 120V \\
Peak & $1.414 \times V_{rms}$ & 170V \\
Peak-to-Peak & $2 \times V_{peak}$ & 340V \\
Average & 0 (for sine wave) & 0V \\
\hline
\end{tabular}
\end{center}

\vspace{0.15cm}

\textbf{Different Waveforms Have Different Factors:}

\textbf{Sine wave:}
\begin{itemize}
    \item $V_{rms} = 0.707 \times V_{peak}$
    \item Most common (power lines)
\end{itemize}

\textbf{Square wave:}
\begin{itemize}
    \item $V_{rms} = V_{peak}$ (factor = 1.0)
    \item RMS equals peak!
\end{itemize}

\textbf{Triangle wave:}
\begin{itemize}
    \item $V_{rms} = 0.577 \times V_{peak}$
    \item Different factor
\end{itemize}

\textit{Standard assumption:}
\begin{itemize}
    \item Unless specified, assume sine wave
    \item Use 0.707 and 1.414 factors
    \item Power systems always sinusoidal
\end{itemize}
\end{detailbox}

\vspace{0.2cm}

\noindent\textbf{\color{accentcolor} Practical Example \& Numerical}
\begin{examplebox}
\textbf{Example 1: North American Household Outlet}

\textit{Wall socket rated "120V AC"}

\textbf{This means:}
\begin{equation*}
    V_{rms} = 120V
\end{equation*}

\textbf{Actual peak voltage:}
\begin{equation*}
    V_{peak} = 1.414 \times 120 = \boxed{169.7V \approx 170V}
\end{equation*}

\textbf{Peak-to-peak voltage:}
\begin{equation*}
    V_{p-p} = 2 \times 170 = \boxed{340V}
\end{equation*}

\textbf{What meter shows:} 120V (RMS)

\vspace{0.2cm}

\textbf{Example 2: European Household Outlet}

\textit{Wall socket rated "230V AC"}

\textbf{RMS voltage:}
\begin{equation*}
    V_{rms} = 230V
\end{equation*}

\textbf{Peak voltage:}
\begin{equation*}
    V_{peak} = 1.414 \times 230 = \boxed{325V}
\end{equation*}

\textbf{Peak-to-peak:}
\begin{equation*}
    V_{p-p} = 2 \times 325 = \boxed{650V}
\end{equation*}

\vspace{0.2cm}

\textbf{Example 3: Equivalent DC Comparison}

\textit{Two circuits with identical light bulbs:}

\textbf{Circuit A (DC):}
\begin{itemize}
    \item 120V DC battery
    \item Light bulb: 100W
    \item Brightness: Reference level
\end{itemize}

\textbf{Circuit B (AC):}
\begin{itemize}
    \item AC source: 170V peak (120V RMS)
    \item Same light bulb: 100W
    \item Brightness: \textbf{Identical to Circuit A!}
\end{itemize}

\textbf{Power calculation (both):}
\begin{align*}
    P &= \frac{V^2}{R} \\
    R &= \frac{120^2}{100} = 144\Omega \text{ (bulb resistance)}
\end{align*}

Circuit A: $P = 120^2 / 144 = 100W$

Circuit B: $P = V_{rms}^2 / R = 120^2 / 144 = 100W$

Same power $\rightarrow$ same brightness!

\vspace{0.2cm}

\textbf{Example 4: Finding RMS from Oscilloscope}

\textit{Oscilloscope shows:}
\begin{itemize}
    \item Peak voltage: 50V
    \item Sine wave
\end{itemize}

\textbf{Calculate RMS:}
\begin{equation*}
    V_{rms} = 0.707 \times 50 = \boxed{35.35V}
\end{equation*}

\textit{If multimeter measured same signal:}
\begin{itemize}
    \item Oscilloscope displays: 50V (peak)
    \item Multimeter displays: 35.35V (RMS)
    \item Both correct, different measurements!
\end{itemize}

\vspace{0.2cm}

\textbf{Example 5: Power Calculation with AC}

\textit{Given:}
\begin{itemize}
    \item AC voltage: 120V RMS
    \item Resistor: 10$\Omega$
\end{itemize}

\textbf{Current (RMS):}
\begin{equation*}
    I_{rms} = \frac{V_{rms}}{R} = \frac{120}{10} = 12A \text{ (RMS)}
\end{equation*}

\textbf{Power dissipated:}
\begin{align*}
    P &= V_{rms} \times I_{rms} \\
    &= 120 \times 12 = 1{,}440W
\end{align*}

Or using $P = V^2/R$:
\begin{equation*}
    P = \frac{120^2}{10} = \frac{14{,}400}{10} = 1{,}440W
\end{equation*}

Or using $P = I^2R$:
\begin{equation*}
    P = 12^2 \times 10 = 144 \times 10 = 1{,}440W
\end{equation*}

All formulas work with RMS values!

\vspace{0.2cm}

\textbf{Example 6: Peak Current from RMS}

\textit{Given:} Circuit draws 5A RMS

\textbf{Peak current:}
\begin{equation*}
    I_{peak} = 1.414 \times 5 = \boxed{7.07A}
\end{equation*}

\textbf{Implication:}
\begin{itemize}
    \item Circuit breaker must handle 7.07A peak
    \item But rated in RMS (5A)
    \item Component ratings usually specify RMS
    \item Peak current 41\% higher than RMS
\end{itemize}

\vspace{0.2cm}

\textbf{Example 7: Lower RMS Voltage, Lower Power}

\textit{Two AC sources powering identical bulbs:}

\textbf{Source A: 170V peak (120V RMS)}
\begin{itemize}
    \item Bulb power: $P = 120^2 / 144 = 100W$
    \item Warm-up time: 5 seconds
    \item Brightness: Reference
\end{itemize}

\textbf{Source B: 120V peak (85V RMS)}
\begin{itemize}
    \item Bulb power: $P = 85^2 / 144 = 50W$
    \item Warm-up time: Much longer
    \item Brightness: Dimmer (half power)
\end{itemize}

Lower RMS $\rightarrow$ lower power $\rightarrow$ less brightness!

\vspace{0.2cm}

\textbf{Example 8: Converting Between Values}

\textit{Given:} Peak-to-peak voltage = 200V (sine wave)

\textbf{Peak voltage:}
\begin{equation*}
    V_{peak} = \frac{V_{p-p}}{2} = \frac{200}{2} = 100V
\end{equation*}

\textbf{RMS voltage:}
\begin{equation*}
    V_{rms} = 0.707 \times 100 = \boxed{70.7V}
\end{equation*}

\textbf{Verification (RMS to peak to p-p):}
\begin{align*}
    V_{peak} &= 1.414 \times 70.7 = 100V \quad \checkmark \\
    V_{p-p} &= 2 \times 100 = 200V \quad \checkmark
\end{align*}
\end{examplebox}

\vspace{0.2cm}

\noindent\textbf{\color{accentcolor} Key Points (Interview Focus)}
\begin{keypointsbox}
\begin{enumerate}
    \item \textbf{RMS:} Effective AC value = equivalent DC for same power
    \item \textbf{Formula:} $V_{rms} = 0.707 \times V_{peak}$ (sine wave only)
    \item \textbf{Reverse:} $V_{peak} = 1.414 \times V_{rms}$
    \item \textbf{Meters:} Always show RMS on AC mode
    \item \textbf{Convention:} "120V AC" means 120V RMS (peak is 170V)
    \item \textbf{Power formulas:} Use RMS values ($P = V_{rms} I_{rms}$)
    \item \textbf{Not average:} Average of sine wave = 0 (useless)
    \item \textbf{Factors vary:} 0.707 only for sine waves; square wave = 1.0
\end{enumerate}

\textbf{Interview Questions:}
\begin{itemize}
    \item \textbf{Q:} What does RMS voltage mean? \\
    \textit{A:} Effective AC voltage - equivalent DC value delivering same power.
    
    \item \textbf{Q:} 120V AC outlet - what is peak voltage? \\
    \textit{A:} $V_{peak} = 1.414 \times 120 = 170V$.
    
    \item \textbf{Q:} Why not use average voltage for AC? \\
    \textit{A:} Average of sine wave = 0 (positive and negative cancel out).
    
    \item \textbf{Q:} What does multimeter show for AC? \\
    \textit{A:} RMS value (not peak, not average).
    
    \item \textbf{Q:} Convert 100V peak to RMS. \\
    \textit{A:} $V_{rms} = 0.707 \times 100 = 70.7V$.
    
    \item \textbf{Q:} Why is RMS useful? \\
    \textit{A:} Allows direct comparison with DC; power formulas work the same.
    
    \item \textbf{Q:} Do RMS formulas apply to square waves? \\
    \textit{A:} No - factors differ (0.707 only for sine waves).
\end{itemize}

\textbf{Key Formulas (Sine Wave):}
\begin{itemize}
    \item $V_{rms} = 0.707 \times V_{peak} = V_{peak}/\sqrt{2}$
    \item $V_{peak} = 1.414 \times V_{rms} = \sqrt{2} \times V_{rms}$
    \item $I_{rms} = 0.707 \times I_{peak}$
    \item $I_{peak} = 1.414 \times I_{rms}$
    \item $P = V_{rms} \times I_{rms}$ (same as DC formulas)
\end{itemize}

\textbf{Memory Aids:}
\begin{itemize}
    \item 0.707 $\approx$ 0.7 $\approx$ 70\% (RMS is ~70\% of peak)
    \item 1.414 $\approx$ 1.4 $\approx$ 140\% (peak is ~40\% higher than RMS)
    \item $\sqrt{2} = 1.414...$ (exact factor)
    \item $1/\sqrt{2} = 0.707...$ (exact factor)
\end{itemize}

\textbf{Common Values:}
\begin{itemize}
    \item 120V RMS $\rightarrow$ 170V peak (North America)
    \item 230V RMS $\rightarrow$ 325V peak (Europe/Asia)
    \item 240V RMS $\rightarrow$ 340V peak (Australia, parts of Asia)
\end{itemize}

\textbf{Applications:}
\begin{itemize}
    \item Power supply specifications
    \item Component voltage ratings
    \item Meter readings interpretation
    \item Power calculations
    \item Circuit design and safety
\end{itemize}

\textbf{Common Mistakes:}
\begin{itemize}
    \item Using peak voltage in power calculations (use RMS!)
    \item Thinking multimeter shows peak (it shows RMS)
    \item Applying sine wave factors to non-sinusoidal signals
    \item Confusing RMS with average (average of sine = 0)
\end{itemize}
\end{keypointsbox}

